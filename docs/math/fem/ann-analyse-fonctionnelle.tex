\chapter{Outils d'analyse fonctionnelle}
%
%
\section{Quelques rappels}
%
\subsection{Normes et produits scalaires}
%
\noindent
Soit $E$ un espace vectoriel.\\
%
\begin{definition}
  \label{def:7}
  $\|.\|$ : $E \rightarrow \RR$ est une {\bf norme} sur $E$ ssi elle v\'erifie :
  \begin{description}
  \item[$\qquad$(N1)] $\left( \| x \| = 0 \right)  \Longrightarrow (x=0)$
  \item[$\qquad$(N2)] $\forall\, \lambda\in\RR,\; \forall x\in E, \quad \| \lambda x \|  = |\lambda| \; \| x \| $
  \item[$\qquad$(N3)] $\forall\,  x,y \in E, \quad \| x+ y \| \le \|x \| + \|y\|\qquad$ (in\'egalit\'e triangulaire)
    \\
  \end{description}
\end{definition}


%
%
\begin{example}
Pour $E=\RR^n$ et $x=(x_1,\ldots,x_n) \in\RR^n$, on d\'efinit les normes
$$
\| x \|_1 = \sum_{i=1}^n |x_i|
\qquad
\| x \|_2 = \left( \sum_{i=1}^n x_i^2 \right)^{1/2}
\qquad
\| x \|_\infty = \sup_{i} |x_i|
$$
\end{example}

%

%
%
%
\begin{definition}
  On appelle {\bf produit scalaire} sur $E$ toute forme bilin\'eaire sym\'etrique d\'efinie positive.\\
  $\quad<.,.>$ : $E\times E \rightarrow \RR$ est donc un produit scalaire sur
  $E$ ssi il v\'erifie :
  \begin{description}
  \item[$\qquad$(S1)] $\forall\; x,y \in E, \quad <x,y> = <y,x>$
  \item[$\qquad$(S2)] $\forall\; x_1,x_2,y \in E, \quad <x_1+x_2,y> = <x_1,y>
    + <x_2,y> $
  \item[$\qquad$(S3)] $\forall\; x,y \in E, \, \forall\, \lambda\in\RR,\quad
    <\lambda x,y> = \lambda <x,y> $
  \item[$\qquad$(S4)] $\forall\;  x \in E, x\ne 0, \quad  <x,x>\; > 0 $\\
  \end{description}
  \label{def:8}
\end{definition}

%
A partir d'un produit scalaire, on peut d\'efinir une {\bf norme induite} : $ \| x \| = \sqrt{<x,x>} $\\
%
On a alors, d'apr\`es (N3), l'{\bf in\'egalit\'e de Cauchy-Schwarz} : $\ds{ | <x,y> | \le \| x \| \; \| y \| }$

%
%
{\bf Exemple :} Pour $E=\RR^n$, on d\'efinit le produit scalaire $\ds{<x,y> = \sum_{i=1}^n x_i \, y_i}$. Sa norme induite est $\| . \|_2$ d\'efinie pr\'ec\'edemment.

%
%
Un espace vectoriel muni d'une norme est appel\'e {\bf espace norm\'e}. \\
Un espace vectoriel muni d'un produit scalaire est appel\'e {\bf espace pr\'ehilbertien}. En particulier, c'est donc un espace norm\'e pour la norme induite.
%
%
\subsection{Suites de Cauchy - espaces complets}
%
%
\noindent
\begin{definition}
  \label{def:1}
  Soit $E$ un espace vectoriel et $(x_n)_n$ une suite de $E$. $(x_n)_n$ est une {\bf suite de Cauchy} ssi $\forall \varepsilon > 0,\;\; \exists N / \forall p>N, \forall q>N, \quad \|x_p - x_q \| < \varepsilon$

\end{definition}

%
Toute suite convergente est de Cauchy. La r\'eciproque est fausse.
%
\begin{definition}
  \label{def:2}
  Un espace vectoriel est {\bf complet} ssi toute suite de Cauchy y est convergente.
\end{definition}


%
\begin{definition}
  \label{def:3}
  Un espace norm\'e complet est un {\bf espace de Banach}.
\end{definition}

%
\begin{definition}
  \label{def:4}
  Un espace pr\'ehilbertien complet est un {\bf espace de Hilbert}.\\
\end{definition}

%
\begin{definition}
  \label{def:5}
  Un espace de Hilbert de dimension finie est appel\'e  {\bf espace euclidien}.
\end{definition}

%
%
%
\section{Espaces fonctionnels}
%
%
\noindent
\begin{definition}
  \label{def:6}
  Un {\bf espace fonctionnel} est un espace vectoriel dont les \'el\'ements
  sont des fonctions.\end{definition}

\begin{example}
  ${\cal C}^p([a;b])$ d\'esigne l'espace des fonctions d\'efinies sur
  l'intervalle $[a,b]$, dont toutes les d\'eriv\'ees jusqu'\`a l'ordre $p$
  existent et sont continues sur $[a,b]$.
\end{example}



%
Dans la suite, les fonctions seront d\'efinies sur un sous-ensemble de $\RR^n$ (le plus souvent un ouvert not\'e $\Omega$), \`a valeurs dans $\RR$ ou $\RR^p$.
%

\begin{example}
  La temp\'erature $T(x,y,z,t)$ en tout point d'un objet $\bar{\Omega}\subset \RR^3$ est une fonction de $ \bar{\Omega} \times \RR \longrightarrow \RR$.
\end{example}


%

%
Les normes usuelles les plus simples sur les espaces fonctionnels sont les
{\bf normes} $\bf L^p$ d\'efinies par :
$$
\| u \|_{L^p} = \left ( \int_{\Omega } |u|^p \right) ^{1/p} \quad ,\; p\in [1,+ \infty[ ,
\qquad
\hbox{et}\qquad
\| u \|_{L^\infty} = {\hbox{Sup}}_{\Omega } |u|
$$
%
%
Comme on va le voir, ces formes $L^p$ ne sont pas n\'ecessairement des
normes. Et lorsqu'elles le sont, les espaces fonctionnels munis de ces normes
ne sont pas n\'ecessairement des espaces de Banach. Par exemple, les formes
$L^\infty$ et $L^1$ sont bien des normes sur l'espace ${\cal C}^0([a;b])$, et
cet espace est complet si on le munit de la norme $L^\infty$, mais ne l'est
pas si on le munit de la norme $L^1$.
%
Pour cette raison, on va d\'efinir les espaces ${\cal L}^p(\Omega)$ ($p\in [1,+ \infty[$) par
$$
{\cal L}^p(\Omega) = \left\{  u : \Omega \rightarrow \RR, \hbox{ mesurable, et telle que } \int_\Omega |u|^p<\infty  \right\}
$$
%
( on rappelle qu'une fonction $u$ est mesurable ssi $\{ x /  |u(x)|<r \}$ est mesurable $\forall r>0$. )
%
Sur ces espaces ${\cal L}^p(\Omega)$, les formes $L^p$ ne sont pas des normes. En effet, $\| u \|_{L^p} = 0$ implique que $u$ est nulle presque partout dans ${\cal L}^p(\Omega)$, et non pas $u=0$. C'est pourquoi on va d\'efinir les {\bf espaces} $\bf L^p(\Omega)$ :
%
\begin{definition}
  $L^p(\Omega)$ est la classe d'\'equivalence des fonctions de ${\cal
  L}^p(\Omega)$ pour la relation d'\'equivalence ``\'egalit\'e presque
  partout''. Autrement dit, on confondra deux fonctions d\`es lors qu'elles
  sont \'egales presque partout, c'est \`a dire qu'elles ne diff\`erent que
  sur un ensemble de mesure nulle.\label{def:9}
\end{definition}

%
\begin{theorem}
  La forme $L^p$ est une norme sur $L^p(\Omega)$, et $L^p(\Omega)$ muni de la
  norme $L^p$ est un espace de Banach (c.a.d. est complet).\label{thr:1}
\end{theorem}

%
Un cas particulier tr\`es important est $p=2$. On obtient alors \boldmath
l'{\bf espace fonctionnel $L^2(\Omega)$}, \unboldmath c'est \`a dire l'espace
des fonctions de carr\'e sommable sur $\Omega$ (\`a la relation
d'\'equivalence ``\'egalit\'e presque partout'' pr\`es). A la norme $L^2$ :
$\| u \|_{L^2} = \left( \int_\Omega u^2 \right)^{1/2} $, on peut associer la
forme bilin\'eaire $(u,v)_{L^2} = \int_\Omega u\, v$. Il s'agit d'un produit
scalaire, dont d\'erive la norme $L^2$. D'o\`u :
%
\begin{theorem}
  $L^2(\Omega)$ est un espace de Hilbert.\label{thr:2}
\end{theorem}

%
%
%
%
\section{Notion de d\'eriv\'ee g\'en\'eralis\'ee}
\label{sec:notion-de-derivee}
%
%
\noindent
Nous venons de d\'efinir des espaces fonctionnels complets, ce qui sera un bon
cadre pour d\'emontrer l'existence et l'unicit\'e de solutions d'\'equations
aux d\'eriv\'ees partielles, comme on le verra plus loin notamment avec le
th\'eor\`eme de Lax-Milgram. Toutefois, on a vu que les \'el\'ements de ces
espaces $L^p$ ne sont pas n\'ecessairement des fonctions tr\`es
r\'eguli\`eres. D\`es lors, les d\'eriv\'ees partielles de telles fonctions ne
sont pas forc\'ement d\'efinies partout. Pour s'affranchir de ce probl\`eme,
on va \'etendre la notion de d\'erivation.
%
Le v\'eritable outil \`a introduire pour cela est la notion de {\bf
distribution}, due \`a L. Schwartz (1950). Par manque de temps dans ce cours,
on se contentera ici d'en donner une id\'ee tr\`es simplifi\'ee, avec la
notion de {\bf d\'eriv\'ee g\'en\'eralis\'ee}.  Cette derni\`ere a des
propri\'et\'es beaucoup plus limit\'ees que les distributions, mais permet de
``sentir" les aspects n\'ecessaires pour mener \`a la formulation
variationnelle.
%
%
Dans la suite, $\Omega$ sera un ouvert (pas n\'ecessairement born\'e) de $\RR^n$.
%
%
\subsection{Fonctions tests}
\label{sec:fonctions-tests}
%
%
\noindent
\begin{definition}
  Soit $\varphi : \Omega \rightarrow \RR$. On appelle {\bf support de $\bf
  \varphi$} l'adh\'erence de $\{ x \in \Omega / \varphi(x) \ne 0 \}$.\label{def:10}
\end{definition}

%
\begin{example}
  Pour $\Omega = ]-1,1[$, et $\varphi$ la fonction constante \'egale \`a 1,
  $\hbox{Supp}\, \varphi = [-1,1]$.
\end{example}

%
\begin{definition}
  On note ${\cal D}(\Omega)$ l'espace des fonctions de $\Omega$ vers $\RR$, de
  classe ${\cal C}^\infty$, et \`a support compact inclus dans $\Omega$.
  ${\cal D}(\Omega)$ est parfois appel\'e {\bf espace des fonctions-tests}.\label{def:11}
\end{definition}

 %
\begin{example}
  L'exemple le plus classique dans le cas 1-D est la fonction \be \varphi(x) =
  \left\{
    \begin{array}{ll}
      \ds{ e^{- \frac{1}{1-x^2}} } & \hbox{si } |x|<1\\
      0 &  \hbox{si } |x|\ge 1\\
    \end{array}
  \right.
  \label{eq:fonction-test1}
  \ee
%
$\varphi$ est une fonction de ${\cal D}(]a,b[)$ pour tous $a < -1 < 1 < b$.
\end{example}

%
Cet exemple s'\'etend ais\'ement au cas multi-dimensionnel ($n>1$). Soit $a\in\Omega$ et $r>0$ tel que la boule ferm\'ee de centre $a$ et de rayon $r$ soit incluse dans $\Omega$. On pose alors :
\be
 \varphi(x) = \left\{
 \begin{array}{ll}
 \ds{ e^{- \frac{1}{r^2-|x-a|^2}} } & \hbox{si } |x-a|<r\\
 0 &  \hbox{sinon }\\
 \end{array}
 \right.
\label{eq:fonction-test2}
\ee
%
$\varphi$ ainsi d\'efinie est \'el\'ement de  ${\cal D}(\Omega)$.
%
\begin{theorem}
  $\overline{{\cal D}(\Omega) } = L^2(\Omega)$\label{thr:4}
\end{theorem}

%
%
%
\subsection{D\'eriv\'ee g\'en\'eralis\'ee}
\label{sec:derivee-generalisee}
%
%
\noindent
Soit $u\in {\cal C}^1(\Omega)$ et $\varphi \in {\cal D}(\Omega)$. Par int\'egration par parties (\cf annexe \ref{sec:green}), on a :
$$
\int_\Omega \partial_i u\;  \varphi = - \int_\Omega u \; \partial_i\varphi + \int_{\partial \Omega} u \; \varphi \; {\bf e}_i.{\bf n}
$$
%
%
Ce dernier terme (int\'egrale sur le bord de $\Omega$) est nul car $\varphi$
est \`a support compact (donc nul sur $\partial \Omega$). Or $\int_\Omega u
\; \partial_i\varphi$ a un sens par exemple d\`es que $u\in L^2(\Omega)$. Donc
le terme $\int_\Omega \partial_i u\; \varphi$ a aussi du sens, sans que $u$ ne
soit n\'ecessairement de classe ${\cal C}^1$. Ceci permet de d\'efinir
$\partial_i u$ m\^eme dans ce cas.
%
\begin{definition}{cas 1-D}
  $\quad$ Soit $I$ un intervalle de \RR, pas forc\'ement born\'e. On
  dit que $u\in L^2(I)$ admet une {\bf d\'eriv\'ee g\'en\'eralis\'ee} dans
  $L^2(I)$ ssi $\exists u_1\in L^2(I)$ telle que $\forall \varphi\in {\cal
  D}(I), \quad \int_I u_1\;\varphi = - \int_I u \varphi'$\label{def:12}
\end{definition}

%
\begin{example}
  Soit $I=]a,b[$ un intervalle born\'e, et $c$ un point de $I$. On consid\`ere
  une fonction $u$ form\'ee de deux branches de classe ${\cal C}^1$, l'une sur
  $]a,c[$, l'autre sur $]c,b[$, et se raccordant de fa\c{c}on continue mais
  non d\'erivable en $c$. Alors $u$ admet une d\'eriv\'ee g\'en\'eralis\'ee
  d\'efinie par $u_1(x)=u'(x)\quad \forall x\ne c$. En effet :
$$
\forall \varphi\in {\cal D}(]a,b[)\qquad \int_a^b u \varphi' = \int_a^c +
\int_c^b = - \int_a^c u' \varphi - \int_c^b u'\varphi +
\underbrace{(u(c^-)-u(c^+))}_{=0} \, \varphi(c)
$$
par int\'egration par parties. La valeur $u_1(c)$ n'a pas d'importance: on a
de toute fa\c{c}on au final la m\^eme fonction de $L^2(I)$, puisqu'elle est
d\'efinie comme classe d'\'equivalence de la relation d'\'equivalence
``\'egalit\'e presque partout".
\end{example}

%o
%
\begin{definition}
  En it\'erant, on dit que $u$ admet une {\bf d\'eriv\'ee g\'en\'eralis\'ee
  d'ordre $\bf k$} dans $L^2(I)$, not\'ee $u_k$, ssi $\ds{\forall \varphi\in
  {\cal D}(I), \quad \int_I u_k\;\varphi = (- 1)^k \; \int_I u \varphi^{(k)}
  }$\label{def:13}
\end{definition}

%
Ces d\'efinitions s'\'etendent naturellement pour la d\'efinition de d\'eriv\'ees partielles g\'en\'eralis\'ees, dans le cas $n>1$.

%
\begin{theorem}
  Quand elle existe, la d\'eriv\'ee g\'en\'eralis\'ee est unique.\label{thr:5}
\end{theorem}

\begin{theorem}
  Quand $u$ est de classe ${\cal C}^1(\bar{\Omega})$, la d\'eriv\'ee
  g\'en\'eralis\'ee est \'egale \`a la d\'eriv\'ee classique.\label{thr:6}
\end{theorem}


%
%
%
%
\section{Espaces de Sobolev}
%
%
\subsection{Les espaces $H^m$}
\label{sec:sobolev}
\noindent

\begin{definition}
  $\ds{ H^1(\Omega) = \left\{ u \in L^2(\Omega)\; / \; \partial_i u \; \in
    L^2(\Omega), \quad 1 \le i \le n \right\} }$ o\`u $\partial_i u$ est
  d\'efinie au sens de la d\'eriv\'ee g\'en\'eralis\'ee.\label{def:14}
\end{definition}
%
$H^1(\Omega)$ est appel\'e {\bf espace de Sobolev d'ordre 1}.
%
\begin{definition}
  Pour tout entier $m\ge 1$,
$$
H^m(\Omega) = \left\{ u \in L^2(\Omega) \; / \; \partial^\alpha u \; \in
  L^2(\Omega) \quad \forall \alpha =(\alpha_1,\ldots,\alpha_n) \in \NN^n\hbox{
  tel que}\; |\alpha|= \alpha_1+\cdots+\alpha_n \le m \right\} $$\label{def:15}
\end{definition}
%
$H^m(\Omega)$ est appel\'e {\bf espace de Sobolev d'ordre $\bf m$}.
%
Par extension, on voit aussi que $H^0(\Omega)=L^2(\Omega)$.
Dans le cas de la dimension 1, on \'ecrit plus simplement pour $I$ ouvert de $\RR$ :
$$ H^m(I) =  \left\{ u \in L^2(I)  \; / \;   u', \ldots, u^{(m)} \in L^2(I) \right\} $$

%
\begin{theorem}
  $H^1(\Omega)$ est un espace de Hilbert pour le produit scalaire
$$
(u,v)_1 = \int_\Omega u \, v\, + \sum_{i=1}^n \; \int_\Omega \partial_i u
\; \partial_i v = (u,v)_0 + \sum_{i=1}^n (\partial_i u, \partial_i v )_0
$$
en notant $(.,.)_0$ le produit scalaire $L^2$. On notera $\|.\|_1$ la norme
associ\'ee \`a $(.,.)_1$.\label{thr:7}
\end{theorem}
%
On d\'efinit de m\^eme un produit scalaire et une norme sur $H^m(\Omega)$ par
$$
(u,v)_m =   \sum_{|\alpha| \le m} ( \partial^\alpha u , \partial^\alpha v )_0 \qquad
\hbox{ et }\qquad
\| u \|_m = (u,u)_m^{1/2}
$$
%
\begin{theorem}
  $H^m(\Omega)$ muni du produit scalaire $(.,.)_m$ est un espace de Hilbert.\label{thr:8}
\end{theorem}


%
\begin{theorem}
  Si $\Omega$ est un ouvert de $\RR^n$ de fronti\`ere $\partial\Omega$
  ``suffisamment r\'eguli\`ere" (par exemple ${\cal C}^1$), on a l'inclusion :
  $H^m(\Omega) \subset {\cal C}^k(\bar{\Omega})$ pour $\ds{ k < m-\frac{n}{2}
  }$\label{thr:9}
\end{theorem}

%
\begin{example}
  En particulier, on voit que pour un intervalle $I$ de $\RR$, on a $H^1(I)
  \subset {\cal C}^0(\bar{I})$, c'est \`a dire que, en 1-D, toute fonction
  $H^1$ est continue.

  L'exemple de $u(x) = x\, \sin\frac{1}{x}$ pour $x\in]0,1]$ et $u(0)=0$
  montre que la r\'eciproque est fausse.

  L'exemple de $u(x,y) = | \ln (x^2+y^2) |^k$ pour $0<k<1/2$ montre qu'en
  dimension sup\'erieure \`a 1 il existe des fonctions $H^1$ discontinues.
\end{example}

%
%
\subsection{Trace d'une fonction}
%
%
Pour pouvoir faire les int\'egrations par parties qui seront utiles par exemple pour la formulation variationnelle, il faut pouvoir d\'efinir le prolongement ({\em la trace}) d'une fonction sur le bord de l'ouvert $\Omega$.
%
\underline{Si $n=1$ (cas 1-D)} : on consid\`ere un intervalle ouvert $I=]a,b[$ born\'e. On a vu que  $H^1(I) \subset {\cal C}^0(\bar{I})$. Donc, pour $u\in H^1(I)$, $u$ est continue sur $[a,b]$, et $u(a)$ et $u(b)$ sont bien d\'efinies.
%
\underline{Si $n>1$} : on n'a plus $H^1(\Omega) \subset {\cal C}^0(\bar{\Omega})$. Comment alors d\'efinir la trace ? La d\'emarche est la suivante :
\begin{itemize}
\item On d\'efinit l'espace
$$
{\cal C}^1(\bar{\Omega}) = \left\{  \varphi : \Omega \rightarrow \RR \;/\;  \exists O \hbox{ ouvert contenant } \bar{\Omega},\; \exists \psi \in {\cal C}^1(O),\; \psi_{|\Omega} = \varphi \right\}
$$
Autrement dit, ${\cal C}^1(\bar{\Omega})$ est l'espace des fonctions ${\cal C}^1$ sur $\Omega$, prolongeables par continuit\'e sur $\partial\Omega$ et dont le gradient est lui-aussi prolongeable par continuit\'e. Il n'y a donc pas de probl\`eme pour d\'efinir la trace de telles fonctions.
%
\item On montre que, si $\Omega$ est un ouvert born\'e de fronti\`ere $\partial\Omega$ ``assez r\'eguli\`ere", alors ${\cal C}^1(\bar{\Omega})$ est dense dans $H^1(\Omega)$.
%
\item L'application lin\'eaire continue, qui \`a toute fonction $u$ de  ${\cal C}^1(\bar{\Omega})$  associe sa trace sur $\partial\Omega$, se prolonge alors en une application lin\'eaire continue de $H^1(\Omega)$ dans $L^2(\partial\Omega)$, not\'ee $\gamma_0$, qu'on appelle {\bf application trace}.
\boldmath
On dit que $\gamma_0(u)$ {\bf est la trace de $u$ sur }$\partial\Omega$.
\unboldmath
%
\end{itemize}
%
%
Pour une fonction $u$ de $H^1(\Omega)$ qui soit en m\^eme temps continue sur $\bar{\Omega}$, on a \'evidemment $\gamma_0(u) = u_{|\partial\Omega}$. C'est pourquoi on note souvent par abus simplement $u_{|\partial\Omega}$ plut\^ot que $\gamma_0(u)$.

%
%
On peut de fa\c{c}on analogue d\'efinir $\gamma_1$, application trace qui permet de prolonger la d\'efinition usuelle de la d\'eriv\'ee normale sur $\partial\Omega$.  Pour $u\in H^2(\Omega)$, on a $\partial_i u \in H^1(\Omega)$, $\forall i=1,\ldots,n$, et on peut donc d\'efinir $\gamma_0(\partial_i u)$. La fronti\`ere $\partial\Omega$ \'etant ``assez r\'eguli\`ere" (par exemple, id\'ealement, de classe ${\cal C}^1$), on peut d\'efinir la normale $n=\left(   \begin{array}{l}  n_1 \\ \vdots \\ n_n \end{array} \right)$ en tout point de $\partial\Omega$. On pose alors $\ds{\gamma_1(u) = \sum_{i=1}^n \gamma_0(\partial_i u) n_i}$. Cette application continue $\gamma_1$ de $H^2(\Omega)$ dans $L^2(\partial\Omega)$ permet donc bien de prolonger la d\'efinition usuelle de la d\'eriv\'ee normale. Dans le cas o\`u $u$ est une fonction  de $H^2(\Omega)$ qui soit en m\^eme temps dans  ${\cal C}^1(\bar{\Omega})$, la d\'eriv\'ee normale au sens usuel de $u$ existe, et $\gamma_1(u)$ lui est \'evidemment  \'egal. C'est pourquoi on note souvent, par abus, $\partial_n u$ plut\^ot que $\gamma_1(u)$.
%
%
\subsection{Espace $\bf H^1_0(\Omega)$}
\label{sec:H10}
%
%
\noindent
\begin{definition}
  Soit $\Omega$ ouvert de $\RR^n$. L'espace $H^1_0(\Omega)$ est d\'efini comme
  l'adh\'erence de ${\cal D}(\Omega)$ pour la norme $\|.\|_1$ de
  $H^1(\Omega)$. (on rappelle que ${\cal D}(\Omega)$ est l'espace des
  fonctions ${\cal C}^\infty$ sur $\Omega$ \`a support compact, encore
  appel\'e espace des fonctions tests)\label{def:16}
\end{definition}

%
\begin{theorem}
  Par construction $H^1_0(\Omega)$ est un espace complet. C'est un espace de
  Hilbert pour la norme $\|.\|_1$\label{thr:10}
\end{theorem}

%
\underline{Si $n=1$ (cas 1-D)} : on consid\`ere un intervalle ouvert $I=]a,b[$ born\'e. Alors $$H^1_0(]a,b[) = \left\{ u \in H^1(]a,b[),\; u(a)=u(b)=0 \right\}$$
%
\underline{Si $n>1$} :  Si $\Omega$ est un ouvert born\'e de fronti\`ere``assez r\'eguli\`ere" (par exemple ${\cal C}^1$ par morceaux), alors  $H^1_0(\Omega) = \ker \gamma_0$.
%
$H^1_0(\Omega)$ est donc le sous-espace des fonctions de  $H^1(\Omega)$ de trace nulle sur la fronti\`ere $\partial\Omega$.
%
\begin{definition}
  Pour toute fonction $u$ de $H^1(\Omega)$, on peut d\'efinir :
$$\ds{ |u|_1 = \left( \sum_{i=1}^n \| \partial_i u \|_0^2 \right)^{1/2}
= \left( \int_\Omega \sum_{i=1}^n \left( \partial_i u \right)^2 dx
\right)^{1/2} } \vspace*{5 mm}
$$\label{def:17}
\end{definition}

%
\begin{theorem}[\textbf{In\'egalit\'e de Poincar\'e}]
  \label{thr:11}
  Si $\Omega$ est born\'e dans au moins une direction, alors il existe une
  constante $C(\Omega)$ telle que $\forall u \in H^1_0(\Omega), \; \|u\|_0 \le
  C(\Omega)\; |u|_1$.
\end{theorem}

%
On en d\'eduit que $|.|_1$ est une norme sur $H^1_0(\Omega)$, \'equivalente \`a la norme $\|.\|_1$.
%

%
%
\begin{corollary}
  Le r\'esultat pr\'ec\'edent s'\'etend au cas o\`u l'on a une condition de
  Dirichlet nulle seulement sur une partie de $\partial\Omega$, si $\Omega$
  est connexe.
\end{corollary}

On suppose que $\Omega$ est un ouvert born\'e connexe, de fronti\`ere ${\cal
C}^1$ par morceaux. Soit $V=\left\{ v\in H^1(\Omega),\, v=0 \hbox{ sur
  }\Gamma_0 \right\}$ o\`u $\Gamma_0$ est une partie de $\partial\Omega$ de
mesure non-nulle. Alors il existe une constante $C(\Omega)$ telle que $\forall
u \in V, \; \|u\|_{0,V} \le C(\Omega)\; |u|_{1,V}$, o\`u $\|.\|_{0,V}$ et
$|.|_{1,V}$ d\'esignent les norme et semi-norme induites sur $V$.
%
On en d\'eduit que $|.|_{1,V}$ est une norme sur $V$, \'equivalente \`a la norme $\|.\|_{1,V}$.
%
%
\small
~\vspace*{5mm}\\
\subsection*{Exercices}
%
\begin{enumerate}
\item Montrer que les fonctions d\'efinies par (\ref{eq:fonction-test1}) et
  (\ref{eq:fonction-test2}) sont bien ${\cal C}^\infty$ \`a support compact.
\item Montrer que ${\cal C}^0([a,b])$ est un espace complet pour la norme $L^\infty$.
\item Montrer que ce n'est pas le cas pour la norme $L^1$ (exhiber une suite
  de Cauchy non convergente dans ${\cal C}^0([a,b])$).
%\item Montrer que $L^1( est le complété de C0 pour la norme L1
%Lp complété de C0 pour norme Lp ? est-ce vrai ??
%
\item D\'emontrer que, lorsqu'elle existe, la d\'eriv\'ee g\'en\'eralis\'ee est unique.
\item D\'emontrer que, pour une fonction de classe ${\cal C}^1$, la
  d\'eriv\'ee g\'en\'eralis\'ee est \'egale \`a la d\'eriv\'ee classique.
\item Soit une fonction de $[a,b]$ vers $\RR$, form\'ee de deux branches de
  classe ${\cal C}^1$ sur $[a,c[$ et $]c,b]$, et discontinue en $c$.  Montrer
  qu'elle n'admet pas de d\'eriv\'ee g\'en\'eralis\'ee. (il faudrait alors
  avoir recours \`a la notion de distribution pour d\'eriver cette fonction).
\item Montrer que $|.|_1$ est une norme sur $H^1_0(\Omega)$, \'equivalente \`a la norme $\|.\|_1$
\end{enumerate}

%
\normalsize

%%
%



%%% Local Variables:
%%% coding: utf-8
%%% mode: latex
%%% TeX-PDF-mode: t
%%% TeX-parse-self: t
%%% TeX-auto-save: t
%%% x-symbol-8bits: nil
%%% TeX-auto-regexp-list: TeX-auto-full-regexp-list
%%% TeX-master: "mef-intro"
%%% ispell-local-dictionary: "american"
%%% End:
