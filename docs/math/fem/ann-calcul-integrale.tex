\chapter{Calcul d'int\'egrales}
%
\section{Formules de Green}
\label{sec:green}
\noindent
%
%
Soit $\Omega$ un ouvert non-vide de $\RR^n$, de fronti\`ere not\'ee $\partial\Omega$. On note $n$ la normale locale sur $\partial\Omega$.\saut
%
On a les propri\'et\'es suivantes, appel\'ees formules de Green, qui sont en fait simplement des cas particuliers d'int\'egration par parties :
\be
\int_\Omega \frac{\partial u}{\partial x_k}\; v\; dx = - \int_\Omega u \; \frac{
\partial v}{\partial x_k} \; dx + \int_{\partial \Omega} u\, v \; (e_k.n)\; ds
\ee
%
o\`u $e_k$ est le vecteur unitaire dans la direction $x_k$.
%
%
\be
\int_\Omega \Delta u \; v\; dx = - \int_\Omega \nabla u \; \nabla v \; dx + \int_{\partial \Omega} \frac{\partial u}{\partial n}\, v \; ds
\ee
%
%
\be
\int_\Omega u \; \hbox{div} E \; dx = - \int_\Omega \nabla u \; E \; dx + \int_{\partial \Omega}  u \; (E.n)\; ds
\ee
%
%
\section{Changement de variable dans une int\'egrale}
\noindent
%
%
Soient $\hat{K}$ et $K$ deux ouverts de $\RR^n$. Soit $F$ un ${\cal C}^1$- diff\'eomorphisme de $\hat{K}$ dans $K$, c'est \`a dire une bijection de classe ${\cal C}^1$ dont la r\'eciproque est \'egalement de classe ${\cal C}^1$. On note $(e_1,\ldots,e_n)$ la base canonique de $\RR^n$ et 
$$
F : x=\sum_{i=1}^n x_i \, e_i \; \longrightarrow \; F(x) = \sum_{i=1}^n F_i(x_1,\ldots,x_n) \, e_i
$$
La matrice jacobienne de $F$ au point $x$, not\'ee $J_F(x)$ est la matrice $n\times n$ d\'efinie par
$$
\left( J_F(x) \right)_{ij} = \frac{\partial F_i}{\partial x_j}(x_1,\ldots,x_n)
\qquad 1\le i,j \le n
$$
%
%
On a alors la formule de changement de variable :
\be
\int_K u(x)\; dx = \int_{\hat{K}} u(F(\hat{x}))\; \left| \hbox{det } J_F(\hat{x}) \right| \; d\hat{x}
\ee
%
%
{\bf Remarque :} dans la m\'ethode des \'el\'ements finis, on aura souvent \`a calculer de tels changements de variables dans des int\'egrales du type $\ds{ \int_K Hu(x)\; dx }$, o\`u $H$ est un op\'erateur aux d\'eriv\'ees partielles (gradient, laplacien, \ldots). Il faudra alors faire attention au changement de variable dans l'op\'erateur lui-m\^eme. Par exemple dans $\RR^2$ :
%
\begin{eqnarray*}
\int_K (\nabla u(x))^2\; dx & = & {\ds \int_K \left[ \left(\frac{\partial u(x,y)}{\partial x} \right)^2 + \left(\frac{\partial u(x,y)}{\partial y} \right)^2 \right]\; dx\; dy }\\
%
& = & \ds{ \int_{\hat{K}} \left[ \left(\frac{\partial u(F(\hat{x},\hat{y}))}{\partial x}  \right)^2 + 
\left(\frac{\partial u(F(\hat{x},\hat{y}))}{\partial y} \right)^2 \right] \left| \hbox{det} J_F(\hat{x}) \right| \; \; d\hat{x}\, d\hat{y}
}\\
%
& = & \ds{ \int_{\hat{K}} \left[ \left(\frac{\partial u(F(\hat{x},\hat{y}))}{\partial
 \hat{x}} \;  \frac{\partial \hat{x}}{\partial x} + \frac{\partial u(F(\hat{x},\hat{y}))}{\partial \hat{y}} \; \frac{\partial \hat{y}}{\partial x} \right)^2  \right.
}\\
%
& & \qquad
 \ds{ \left. +
\left(\frac{\partial u(F(\hat{x},\hat{y}))}{\partial \hat{x}} \;  \frac{\partial \hat{x}}{\partial y} + \frac{\partial u(F(\hat{x},\hat{y}))}{\partial \hat{y}} \; \frac{\partial \hat{y}}{\partial y} \right)^2 \right] \left| \hbox{det} J_F(\hat{x}) \right| \; \; d\hat{x}\, d\hat{y}
}
\end{eqnarray*}
%
$$
\quad
= \int_{\hat{K}} \left[ \left(\frac{\partial \hat{u}(\hat{x},\hat{y})}{\partial
 \hat{x}} \;  \frac{\partial \hat{x}}{\partial x} + \frac{\partial \hat{u}(\hat{x},\hat{y})}{\partial \hat{y}} \; \frac{\partial \hat{y}}{\partial x} \right)^2 + 
\left(\frac{\partial \hat{u}(\hat{x},\hat{y})}{\partial \hat{x}} \;  \frac{\partial \hat{x}}{\partial y} + \frac{\partial \hat{u}(\hat{x},\hat{y})}{\partial \hat{y}} \; \frac{\partial \hat{y}}{\partial y} \right)^2 \right] \left| \hbox{det} J_F(\hat{x}) \right| \; \; d\hat{x}\, d\hat{y}
\vspace*{5 mm}
$$
%
Dans le cas d'une transformation $F$ affine, not\'ee :
$$
\left\{
\begin{array}{lll}
x & = & a\hat{x} + b\hat{y} + e\\
y & = & c\hat{x} + d\hat{y} + f
\end{array}
\right.
$$
on a :
$$
\hat{x} = \frac{d(x-e)-b(y-f)}{D},\qquad 
\hat{y} = \frac{-c(x-e)+a(y-f)}{D},\quad \hbox{ et }
\left| \hbox{det} J_F(\hat{x}) \right| = D = ad-bc
\vspace*{5 mm}
$$
%
Le calcul pr\'ec\'edent devient alors :
$$
\int_K (\nabla u(x))^2\; dx = \int_{\hat{K}} \left[ \left(\frac{\partial \hat{u}(\hat{x},\hat{y})}{\partial \hat{x}} \;  \frac{d}{D} + \frac{\partial \hat{u}(\hat{x},\hat{y})}{\partial \hat{y}} \; \frac{-c}{D} \right)^2 + 
\left(\frac{\partial \hat{u}(\hat{x},\hat{y})}{\partial \hat{x}} \;  \frac{-b}{D} + \frac{\partial \hat{u}(\hat{x},\hat{y})}{\partial \hat{y}} \; \frac{a}{D} \right)^2 \right] |D| \; \; d\hat{x}\, d\hat{y}
$$
%
$$
\qquad\qquad\quad\qquad = \frac{1}{|D|}\; \int_{\hat{K}} \left[ \left( d\, \frac{\partial \hat{u}(\hat{x},\hat{y})}{\partial \hat{x}} - c\, \frac{\partial \hat{u}(\hat{x},\hat{y})}{\partial \hat{y}} \right)^2 + 
\left(-b\, \frac{\partial \hat{u}(\hat{x},\hat{y})}{\partial \hat{x}} \; + a\, \frac{\partial \hat{u}(\hat{x},\hat{y})}{\partial \hat{y}} \right)^2 \right]  \; \; d\hat{x}\, d\hat{y}
$$





