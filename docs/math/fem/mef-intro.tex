\documentclass[a4paper,final,12pt]{report}

\usepackage{a4wide}
\usepackage[T1]{fontenc}
\usepackage[utf8x]{inputenc}
\usepackage{stmaryrd}
\usepackage{femmacros}
\usepackage{lipsum}
\usepackage{booktabs}
\usepackage{multido}
\newtheorem{definition}{Définition}[chapter]
%\newtheorem{proposition}{Proposition}[chapter]
\newtheorem{vocabulary}{Vocabulaire}[chapter]
%\newtheorem{theorem}{Théorème}[chapter]
\newtheorem{example}{Exemple}[chapter]
\newtheorem{corollary}{Corollaire}[chapter]
\newtheorem{remark}{Remarque}[chapter]
\newtheorem{problem}{Problème}[chapter]
\newtheorem{lemma}{Lemme}[chapter]

\newcounter{theorem}[chapter]
\newenvironment{theorem}
  {\par\noindent\theoremhang\par\nobreak\noindent
   \refstepcounter{theorem}\postdisplaypenalty=10000 %
   {\sffamily\bfseries\upshape Théorème \thetheorem}\ \ignorespaces}
  {\par\nobreak\noindent\theoremhung\par\addvspace{\topsep}}

\newcounter{proposition}[chapter]
\newenvironment{proposition}
  {\par\noindent\theoremhang\par\nobreak\noindent
   \refstepcounter{proposition}\postdisplaypenalty=10000 %
   {\sffamily\bfseries\upshape Proposition \theproposition}\ \ignorespaces}
  {\par\nobreak\noindent\theoremhung\par\addvspace{\topsep}}

\newenvironment{proof}
  {\par\addvspace{\topsep}\noindent{\sffamily\bfseries\upshape Preuve}\ \ignorespaces}
  {\par\nobreak}% no spacing here so that supposes you will always use {proof} inside {theorem} or else there might be some problems.

\newcommand{\theoremhang}{% top theorem decoration
  \begingroup%
    \setlength{\unitlength}{.005\linewidth}% \linewidth/200
    \begin{picture}(0,0)(1.5,0)%
      \linethickness{0.45pt} \color{black!50}%
      \put(-3,2){\line(1,0){206}}% Top line
      \multido{\iA=2+-1,\iB=50+-10}{5}{% Top hangs
        \color{black!\iB}%
        \put(-3,\iA){\line(0,-1){1}}% Top left hang
        \put(203,\iA){\line(0,-1){1}}% Top right hang
      }%
    \end{picture}%
  \endgroup%
}%

\newcommand{\theoremhung}{% bottom theorem decoration
  \begingroup%
    \setlength{\unitlength}{.005\linewidth}% \linewidth/200
    \begin{picture}(0,0)(1.5,0)%
      \linethickness{0.45pt} \color{black!50}%
      \put(-3,0){\line(1,0){206}}% Bottom line
      \multido{\iA=0+1,\iB=50+-10}{5}{% Bottom hangs
        \color{black!\iB}%
        \put(-3,\iA){\line(0,1){1}}% Bottom left hang
        \put(203,\iA){\line(0,1){1}}% Bottom right hang
      }%
    \end{picture}%
  \endgroup%
}%


\usepackage{filecontents}

\usepackage{indentfirst,graphicx,subfigure}
\usepackage{colortbl,xcolor}
\usepackage{url}
\usepackage[colorlinks,linkcolor=blue,urlcolor=blue]{hyperref}
\usepackage{ifthen}
\usepackage{listings}
\usepackage{tikz}
\usetikzlibrary{arrows,backgrounds,mindmap,positioning,shadows,shapes}
\usepackage{pgfplots,pgfplotstable}
\definecolor{lbcolor}{rgb}{0.95,0.95,0.95}
\definecolor{cblue}{rgb}{0.,0.0,0.6}
\definecolor{lblue}{rgb}{0.1,0.1,0.4}
\definecolor{ljk}{rgb}{0.50, 0.625, 0.70}


% Pretty-print listings
\lstset{
  language        = C++,
  texcl           = true,
  mathescape      = true,
  basicstyle      = \tt,
  keywordstyle    = \bf\color{red!50},
  commentstyle    = \it\tt\color{black!70},
  stringstyle     = \color{black!50},
  morekeywords    = {grad, gradt, dot, ddot, id, idt, avg, avg, jump, oplus, divt,
  elements,boundaryfaces,markedelements,markedfaces,
  ominus, normal, h, integrate, form1, form2, abs, on, unknown, comp, diff,
  div, mult, poly, gradient, span, submesh, trace_interpolator, dof_space,
  N, H},
  frame={top,bottom},
  backgroundcolor=\color{lbcolor},
}

\lstset{escapeinside={(*@}{@*)}}
\pgfqkeys{/pgfplots}{
%  cycle list name = black white,
  legend style = { at={(0.95,0.3)} },
  log basis x = 10
}

% Image path
\graphicspath{{../../Figures/}}

%
\newcommand{\setR}[1]{{\ensuremath{\mathbb{R}^{#1}}}\xspace}
\newcommand{\R}[1]{\mathbb R^{#1}}

%
% \newcommand{\definition}{{\bf D\'efinition : }}
% \newcommand{\theoreme}{{\bf Th\'eor\`eme : }}
% \newcommand{\exemple}{{\bf Exemple : }}
% \newcommand{\propriete}{{\bf Propri\'et\'e : }}
% \renewcommand{\labelitemi}{$\bullet$}
%
%
%
%
%
%\includeonly{ch-problemes-coercifs}
%
\begin{document}

\pagenumbering{roman}
\thispagestyle{empty}
~\vspace*{7 cm}\\
\begin{center}
{\LARGE\bf Notes de cours\vspace*{10 mm}\\ sur la m\'ethode des \'el\'ements
finis} \vspace*{8 cm}
\end{center}
%
%
{\bf
~\hspace*{\fill}Master 1 Calcul Scientifique et Sécurité Informatique\\
~\hspace*{\fill}Christophe Prud'homme, Éric Blayo\\
~\hspace*{\fill}2012-2013\\
}
%
\newpage
%
~\vspace*{7cm}\\
%

%%% Local Variables:
%%% coding: utf-8
%%% mode: latex
%%% TeX-PDF-mode: t
%%% TeX-parse-self: t
%%% TeX-auto-save: t
%%% x-symbol-8bits: nil
%%% TeX-auto-regexp-list: TeX-auto-full-regexp-list
%%% TeX-master: "mef-intro"
%%% ispell-local-dictionary: "american"
%%% End:



\tableofcontents
\clearpage
\pagenumbering{arabic}
\chapter{Introduction \`a la méthode des éléments finis}
%
%
\section{Formulation variationnelle}
%
\subsection{Exemple 1-D}
\label{sec:modele-1D}
%
\noindent
Soit \`a résoudre le problème
\be
({\cal P})\qquad
\left\{
\begin{array}{l}
-u"(x)+c(x)u(x) = f(x)\qquad a<x<b\\
u(a)=u(b)=0\\
\end{array}\right.
\label{eq:modele-1D}
\ee
où $f$ et $c$ sont des fonctions données continues sur $[a,b]$. On supposera de plus que la fonction $c$ est strictement positive sur $[a,b]$. Un tel problème est appelé {\it problème aux limites}.\saut
%
\begin{definition}
  Une {\bf solution classique} (ou {\bf solution forte}) de $({\cal P})$ est
  une fonction de ${\cal C}^2([a,b])$ telle que $u(a)=u(b)=0$ et $\forall x
  \in ]a,b[, \; -u"(x)+c(x)u(x) = f(x)$.
  \label{def:18}
\end{definition}

%
En faisant le produit scalaire $L^2(]a,b[)$ de l'équation différentielle avec une fonction-test $v \in {\cal D}(]a,b[)$ (c'est \`a dire en intégrant sur $[a,b]$), on a :
$$
- \int_a^b u"(x)v(x)\; dx + \int_a^b c(x)u(x)v(x)\; dx = \int_a^b f(x)v(x)\; dx
$$
soit, en intégrant par parties le premier terme :
$$
\int_a^b u'(x)v'(x)\; dx + \int_a^b c(x)u(x)v(x)\; dx = \int_a^b f(x)v(x)\; dx
$$
car $v(a)=v(b)=0$ puisque $v\in{\cal D}(]a,b[)$. Chaque terme de cette équation a en fait un sens dès lors que $v\in H^1_0(]a,b[)$. De plus, ${\cal D}(]a,b[)$ étant dense dans $H^1_0(]a,b[)$ (\cf \S \ref{sec:H10}), cette équation est vérifiée pour tout $v\in H^1_0(]a,b[)$. \par
%
%
On peut donc définir le nouveau problème :
\begin{equation}
({\cal Q})\quad
\left\{
\begin{array}{l}
\hbox{Trouver }u\in H^1_0(]a,b[) \hbox{ tel que }\\
\ds{ \int_a^b u'(x)v'(x)\; dx + \int_a^b c(x)u(x)v(x)\; dx = \int_a^b f(x)v(x)\; dx \quad\forall v\in H^1_0(]a,b[)}
\end{array}
\right.
\label{eq:FV}
\end{equation}
%
%
Ce problème est la {\bf formulation variationnelle} (ou {\bf formulation
faible}) du problème $({\cal P})$.  Toute solution de $({\cal Q})$ est appelée
{\bf solution faible}. Il est immédiat que toute solution forte de $({\cal
P})$ est aussi une solution faible.
%
%
%
\subsection{Exemple 2-D}
\label{sec:modele-2D}
%
\noindent
Soit $\Omega$ ouvert borné de $\RR^n$. On veut résoudre le problème
\be
({\cal P})\qquad
\left\{
\begin{array}{ll}
-\Delta u+u = f & \hbox{ dans }\Omega\\
\partial_n u=0& \hbox{ sur }\partial\Omega\\
\end{array}\right.
\label{eq:modele-2D}
\ee
%
Une solution classique de ce problème est une fonction de ${\cal C}^2(\bar{\Omega})$ vérifiant (\ref{eq:modele-2D}) en tout point de $\Omega$. Au passage, on voit que ceci impose que $f$ soit ${\cal C}^0(\bar{\Omega})$. Toute solution classique vérifie donc :
%
$$
\forall v\in {\cal C}^1(\bar{\Omega}) \quad \int_\Omega -\Delta u\; v + \int_\Omega uv = \int_\Omega fv
$$
%
soit par intégration par parties  : $\ds{  \int_\Omega \nabla u \cdot \nabla v + \int_\Omega uv = \int_\Omega fv
 }$ puisque $\partial_n u = 0 $ sur $\partial\Omega$.
%
%
Or, $\overline{{\cal C}^1(\bar{\Omega})} = H^1(\Omega)$. On peut donc définir le nouveau problème :
\be
({\cal Q})\quad
\left\{
\begin{array}{l}
\hbox{Trouver }u\in H^1(\Omega) \hbox{ tel que }\\
 \qquad \ds{ \int_\Omega \nabla u \cdot \nabla v + \int_\Omega u v = \int_\Omega f v \quad\forall v\in H^1(\Omega)}
\end{array}
\right.
\label{eq:FV2}
\ee
%
C'est la formulation variationnelle de $({\cal P})$. On voit aussi que ce problème est défini dès lors que $f\in L^2(\Omega)$.
%
%
\subsection{Formulation générale}
%
\noindent
Les exemples précédents montre que, d'une fa\c{c}on générale, la formulation variationnelle sera obtenue en faisant le produit scalaire $L^2(\Omega)$ de l'équation avec une fonction $v$ appartenant \`a un espace \`a préciser (c'est \`a dire en multipliant par $v$ et en intégrant sur $\Omega$), et en intégrant par parties les termes d'ordre les plus élevés en tenant compte des conditions aux limites du problème. On arrive alors \`a une formulation du type :
$$
\qquad\qquad
\hbox{Trouver }u\in V \hbox{ tel que }
a(u, v) =l(v) \quad\forall v\in V
$$
%
où $a(.,.)$ est une forme sur $V\times V$ (bilinéaire si l'EDP de départ est linéaire) et $l(.)$ est une forme sur $V$ (linéaire si les conditions aux limites de  l'EDP de départ le sont).
%
%
\section{Théorème de Lax-Milgram}
\subsection{Définitions et théorèmes}
\label{sec:lax-milgram}
%
%
\noindent
On va introduire ici un outil important pour assurer l'existence et l'unicité de solutions \`a la formulation variationnelle de problèmes aux limites de type elliptique.\saut
%
Soit $V$ un espace de Hilbert.\\
%
\begin{definition}
  Une {\bf forme linéaire} $l(u)$ sur $V$ est {\bf continue} ssi il existe une
  constante $K$ telle que $|l(u)| \le K\, \|u\| \quad \forall u \in V$.\label{def:19}
\end{definition}

%
\begin{definition}
  Une {\bf forme bilinéaire} $a(u,v)$ sur $V\times V$ est {\bf continue} ssi
  il existe une constante $M$ telle que $|a(u,v)| \le M\, \|u\| \|v\| \quad
  \forall (u,v) \in V\times V$.\label{def:20}
\end{definition}

%
%
\begin{definition}
  Une {\bf forme bilinéaire} $a(u,v)$ sur $V\times V$ est {\bf coercive} (ou
  {\bf V-elliptique}) ssi il existe une constante $\alpha >0$ telle que
  $a(u,u) \ge \alpha \, \|u\|^2, \; \forall u \in V$.
  \label{def:21}
\end{definition}

%
%
\begin{theorem}
  \label{thr:12}
  {\bf (Lax-Milgram) } Soit $V$ un espace de Hilbert. Soit $a$ une forme
  bilinéaire continue coercive sur $V$. Soit $l$ une forme linéaire
  continue sur $V$. Alors il existe un unique $u\in V$ tel que $a(u,v)=l(v),\;
  \forall v\in V$.

  De plus, l'application linéaire $l \rightarrow u$ est continue.

  \begin{proof}
    La démonstration générale de ce théorème peut être trouvée par exemple dans cite:[raviart-thomas-1983].
  \end{proof}
\end{theorem}



%
%
\begin{theorem}
  On prend les mêmes hypothèses que pour le théorème de Lax-Milgram,
  et on suppose de plus que $a$ est symétrique, c'est \`a dire que
  $a(u,v)=a(v,u)\quad\forall u,v$. On définit alors la fonctionnelle $
  J(v)=\frac{1}{2}\, a(v,v)-l(v)$, et on considère le problème de
  minimisation :
  $$
  \hbox{Trouver } u\in V \hbox{ tel que } J(u) = \min_{v\in V} J(v)
  $$
  %
  Alors ce problème admet une solution unique, qui est également la solution
  du problème variationnel précédent.\label{thr:13}
  \begin{proof}
    La démonstration de ce théorème vient du fait que $J$ est une fonctionnelle quadratique, et que l'on a $\nabla J[u](v) = a(u,v) - l(v)$.
  \end{proof}
\end{theorem}

\paragraph{Remarque}
\label{sec:remarque}

C'est de cette propriété que vient l'utilisation du terme ``variationnel", puisqu'elle montre le lien avec le ``calcul des variations".
%
%
%
%
\subsection{Retour \`a l'exemple 1-D}
\label{sec:modele-1D2}
%
\noindent
%
En reprenant l'exemple 1-D précédent, on peut  poser :
\be
a(u,v) = \int_a^b u'(x)v'(x)\; dx + \int_a^b c(x)u(x)v(x)\; dx
\ee
et
\be
l(v) = \int_a^b f(x)v(x)\; dx
\ee
%
$a$ ainsi définie est une forme bilinéaire symétrique continue coercive
sur $H^1_0(a,b) \times H^1_0(a,b)$, et $l$ est une forme linéaire continue
sur $H^1_0(a,b)$. Donc le problème (\ref{eq:FV}) admet une solution unique
d'après le théorème de Lax-Milgram.\saut
%
% %
Cherchons maintenant \`a interpréter cette solution $u$ de
(\ref{eq:FV}). Prenons $v=\varphi \in {\cal D}(]a,b[)$. Alors
$$
\int_a^b u'(x)\varphi'(x)\; dx + \int_a^b c(x)u(x)\varphi(x)\; dx = \int_a^b f(x)\varphi(x)\; dx
$$
soit, en intégrant par parties :
$$
- \int_a^b u"(x)\varphi(x)\; dx + \int_a^b c(x)u(x)\varphi(x)\; dx = \int_a^b f(x)\varphi(x)\; dx
$$
c'est \`a dire $(-u"+cu,\varphi)_0 = (f,\varphi)_0\; \forall \varphi \in {\cal
D}(]a,b[)$.  ${\cal D}(]a,b[)$ étant dense dans $L^2(]a,b[)$, on a donc :
$-u"+cu=f$ dans $L^2(]a,b[)$. $u$ étant dans $L^2(]a,b[)$, et $f$ et $c$
étant dans ${\cal C}^0([a,b])$, donc également dans $L^2(]a,b[)$, on  en
déduit que $u"=cu-f$ est aussi dans $L^2(]a,b[)$.

Puisque $u$ est dans $H^1_0(]a,b[)$ et que $u"$ est dans $L^2(]a,b[)$, on en
déduit que  $u$ est dans $H^2(]a,b[)$. Donc $u$ est dans ${\cal C}^1([a,b])$
(\cf \S \ref{sec:sobolev}).

De ce fait, $cu-f$, c'est \`a dire $u"$, est dans ${\cal C}^0([a,b])$. Donc
$u'$ est dans ${\cal C}^1([a,b])$, donc $u$ est dans ${\cal C}^2([a,b])$.

La solution faible $u$ est donc aussi solution forte du problème de
départ.\saut
%
%
En résumé :
\begin{itemize}
\item On est parti d'un problème $({\cal P})$ et on a introduit sa
  formulation variationnelle $({\cal Q})$.
\item On a montré l'existence et l'unicité d'une solution faible (en
  utilisant le théorème de Lax-Milgram). Toute solution forte étant
  aussi solution faible, ceci prouve qu'il y a au plus une solution forte pour
  $({\cal P})$.
\item On a prouvé que cette solution faible est bien une solution forte. Le
  problème de départ $({\cal P})$ a donc une solution unique.  \saut
\end{itemize}
%
L'intérêt de cette démarche est d'une part que la formulation
variationnelle se prête bien \`a l'étude de l'existence et de l'unicité
de solutions, et d'autre part que l'on travaille dans des espaces de Hilbert,
ce qui va permettre de faire de l'approximation interne.
%
%
\subsection{Équations elliptiques d'ordre 2}
\label{sec:elliptique}
%
\noindent
%
Soit $\Omega$ un ouvert borné de $\RR^n$, de frontière $\partial\Omega$ assez régulière.
Soient des fonctions $\alpha_{ij}$ ($1\le i,j \le n$) dans ${\cal C}^1(\bar{\Omega})$ et $\beta$ dans ${\cal C}^0(\bar{\Omega})$.
On considère le problème :
\be
({\cal P})\qquad
\left\{
\begin{array}{rl}
{\ds -\sum_{i,j=1}^n \partial_i (\alpha_{ij} \, \partial_j\, u) + \beta\, u = f } & \hbox{ dans }\Omega \\
u= 0 & \hbox{ sur }\Gamma_0 \\
{\ds \sum_{i,j=1}^n  \alpha_{ij} \, \partial_j  u\; n_i = g } & \hbox{ sur }\Gamma_1
%
\end{array}\right.
\label{eq:edp-elliptique}
\ee
%
où $\Gamma_0$ et $\Gamma_1$ forment une partition de $\partial\Omega$ (
$\Gamma_0 \cap\Gamma_1 = \emptyset$ et $\Gamma_0 \cup\Gamma_1
= \partial\Omega$).
%
Une solution classique de $({\cal P})$, sous l'hypothèse que $f\in{\cal C}^0(\bar{\Omega})$ et $g\in{\cal C}^0(\Gamma_1)$, sera une fonction de ${\cal C}^2(\bar{\Omega})$ vérifiant l'équation en chaque point de $\Omega$.\saut
%
%
La formulation variationnelle de $({\cal P})$ est obtenue par intégration
par parties. Elle s'écrit :
%
\be
({\cal Q})\quad
\left\{
\begin{array}{l}
\hbox{Trouver }u\in V \hbox{ tel que }\\
\qquad \ds{ \int_\Omega  \left(  \sum_{i,j=1}^n \alpha_{ij} \, \partial_j u\;  \, \partial_i v + \beta\,   u v \right) = \int_\Omega f v +  \int_{\Gamma_1} gv \qquad\forall v\in V}
\end{array}
\right.
\label{eq:FV3}
\ee
%
avec $\ds{ V = \left\{ v \in H^1(\Omega) \; , \; v=0 \hbox{ sur }\Gamma_0
\right\} }$. Cette formulation est en fait définie dès lors que $\beta$ et
les $\alpha_{ij}$ sont dans $L^\infty(\Omega)$, $f$ dans $L^2(\Omega)$ et $g$ dans
$L^2(\Gamma_1)$.
%
Posons
$$\ds{a(u,v) =  \int_\Omega   \left( \sum_{i,j=1}^n \alpha_{ij} \, \partial_j
  u\;   \partial_i v + \beta\,   u v \right) },
\quad \ds{l(v) = \int_\Omega f v +  \int_{\Gamma_1} gv.\qquad}$$
%
Il est immédiat que $a$ est une forme bilinéaire continue et $l$ une forme
linéaire continue sur $V$.


%
Si l'EDP de départ (\ref{eq:edp-elliptique}) vérifie les deux hypothèses d'ellipticité :
\begin{itemize}
\item il existe $\alpha >0$ tel que $\forall \xi=(\xi_1, \ldots , \xi_n)\in\RR^n$, $ {\ds \sum_{i,j=1}^n  \alpha_{ij}(x) \, \xi_i \, \xi_j  \ge \alpha \, \| \xi \|^2 }$ presque pour tout $x\in\Omega$
%
\item il existe $\beta_0$ tel que $\beta(x) \ge \beta_0$ presque pour tout $x\in\Omega$\\
\end{itemize}
%
alors $a$ est coercive :
\begin{itemize}
\item sur $H^1_0(\Omega)$ dès que $\ds{\alpha_0 >
  \frac{-\alpha}{C(\Omega)^2}}$ (et donc en particulier si $\beta\ge 0$) où
  $C(\Omega)$ est la constante de l'inégalité de Poincaré, voir le
  théorème~\ref{thr:11}.
\item sur $H^1(\Omega)$ si $\beta > 0$\\
\end{itemize}
%
%
Par application du théorème de Lax-Milgram, on a donc existence et unicité
d'une solution \`a la formulation variationnelle $({\cal Q})$ :
\begin{itemize}
\item  si $\Gamma_0 = \partial\Omega$ (c'est \`a dire $\Gamma_1=\emptyset$) et si  $\ds{\beta > \frac{-\alpha}{C(\Omega)^2}}$
%
\item si $\Gamma_1\ne \emptyset$ et si $\beta > 0$
\end{itemize}
%
%
%
\section{Approximation interne}
%
\subsection{Principe général}
%
\noindent
Soit $\Omega$ un domaine ouvert de $\RR^n$ ($n=1,2$ ou 3 en pratique), de
frontière $\partial\Omega$, et sur lequel on cherche \`a résoudre une
équation aux dérivées partielles, munie de conditions aux limites. En
écrivant la formulation variationnelle, on obtient un problème de la forme
%
$$
({\cal Q})\qquad \hbox{Trouver } u\in V \hbox{ tel que } a(u,v)=l(v), \quad\forall v\in V
$$
où $V$ est un espace de Hilbert. Sous réserve que l'équation de départ
ait de bonnes propriétés, c'est \`a dire par exemple qu'on soit dans les
hypothèses du théorème de Lax-Milgram, $({\cal Q})$ admet une solution
unique $u$.  Pour obtenir une approximation numérique de $u$, on va
maintenant remplacer l'espace $V$ qui est en général de dimension infinie
par un sous-espace $V_h$ de dimension finie, et on va chercher \`a résoudre
le problème approché
%
\begin{equation}
  \label{eq:6}
  ({\cal Q}_h)\qquad \hbox{Trouver } u_h\in V_h \hbox{ tel que } a(u_h,v_h)=l(v_h), \quad\forall v_h\in V_h
\end{equation}

%
$V_h$ étant de dimension finie, c'est un fermé de $V$. $V$ étant un
espace de Hilbert, $V_h$ l'est donc aussi. D'où l'existence et l'unicité
de $u_h$, \`a nouveau par exemple d'après le théorème de
Lax-Milgram.\saut
%
L'espace $V_h$ sera en pratique construit \`a partir d'un maillage du domaine
$\Omega$, l'indice $h$ désignant la ``taille typique'' des mailles. Lorsque
l'on construit des maillages de plus en plus fins, la suite de sous-espaces
$(V_h)_h$ formera une {\bf approximation interne} de $V$, c'est \`a dire que,
pour tout élément $\varphi$ de $V$, il existe une suite de $\varphi_h\in
V_h$ telle que $\|\varphi-\varphi_h\|\longrightarrow 0$ quand
$h\longrightarrow 0$. Cette méthode d'approximation interne est également
appelée {\bf méthode de Galerkin}.
%
%
\subsection{Interprétation de $u_h$}
%
\noindent
%
On a $a(u,v)=l(v), \forall v\in V$, donc en particulier $a(u,v_h)=l(v_h),
\forall v_h\in V_h$, car $V_h\subset V$. Par ailleurs, $a(u_h,v_h)=l(v_h),
\forall v_h\in V_h$. Par différence, on en déduit que
\begin{equation}
  a(u-u_h,v_h)=0,\quad \forall v_h\in V_h
  \label{eq:ortho}
\end{equation}
%
Dans le cas où $a(.,.)$ est symétrique, il s'agit d'un produit scalaire
sur $V$. $u_h$ peut alors être interprétée comme la projection
orthogonale de $u$ sur $V_h$ au sens de $a(.,.)$.
%
%
\subsection{Estimation d'erreur}
\label{sec:estim}
%
\noindent
%
On a :
$$
\begin{array}{ll}
a(u-u_h,u-u_h) & = a(u-u_h,u-v_h+v_h-u_h) \quad\forall v_h\in V_h\\
 & =a(u-u_h,u-v_h) + a(u-u_h,v_h-u_h)
\end{array}
$$
%
Or $v_h-u_h \in V_h$. Donc $a(u-u_h,v_h-u_h)=0$ d'après (\ref{eq:ortho}).
On a donc :
\begin{equation}
  a(u-u_h,u-u_h) = a(u-u_h,u-v_h) \quad\forall v_h\in V_h
 \label{eq:estim1}
\end{equation}
%
$a$ étant coercive, il existe $\alpha > 0$ tel que $a(u-u_h,u-u_h) \ge
\alpha \|u-u_h\|^2$, où $\|.\|$ est une norme sur $V$. Par ailleurs, $a$
étant continue, il existe $M > 0$ tel que $a(u-u_h,u-v_h)\le M \|u-u_h\| \,
\|u-v_h\|$. En réinjectant ces deux inégalités de part et d'autre de
(\ref{eq:estim1}) et en simplifiant par $\|u-u_h\|$ on obtient

\begin{equation}
  \|u-u_h\|
  \le \frac{M}{\alpha}\; \|u-v_h\| \quad \forall v_h\in V_h
  \label{eq:cea}
\end{equation}
%
c'est \`a dire

\begin{equation}
\|u-u_h\| \le \frac{M}{\alpha}\; d(u,V_h)
\label{eq:4}
\end{equation}


%
où $d$ est la distance induite par $\|.\|$.  Cette majoration est appelée
{\bf lemme de Céa}. Elle ramène l'étude de l'erreur d'approximation
$u-u_h$ \`a l'étude de l'erreur d'interpolation $d(u,V_h)$.
%
%
\section{Principe  général de la méthode des éléments finis}
\label{sec:general}
\noindent
%
%
La démarche générale de la méthode des éléments finis est la
suivante. On a une EDP \`a résoudre sur un domaine $\Omega$. On écrit la
formulation variationnelle de cette EDP, et on se ramène donc \`a un
problème du type
%
$$
({\cal Q})\qquad \hbox{Trouver } u\in V \hbox{ tel que } a(u,v)=l(v), \quad\forall v\in V
$$
%
On va chercher une approximation de $u$ par approximation interne. Pour cela,
on définit un maillage du domaine $\Omega$, gr\^ace auquel on va définir
un espace d'approximation $V_h$, s.e.v. de $V$ de dimension finie $N_h$ (par
exemple $V_h$ sera l'ensemble des fonctions continues sur $\Omega$ et affines
sur chaque maille).  Le problème approché est alors
$$
({\cal Q}_h)\qquad \hbox{Trouver } u_h\in V_h \hbox{ tel que } a(u_h,v_h)=l(v_h), \quad\forall v_h\in V_h
$$
%
Soit $(\varphi_1,\ldots,\varphi_{N_h})$ une base de $V_h$. En décomposant $u_h$ sur cette base sous la forme
\be
u_h = \sum_{i=1}^{N_h} \mu_i \; \varphi_i
\ee
le problème $({\cal Q}_h)$ devient
\be
\hbox{Trouver } \mu_1,\ldots,\mu_{N_h} \hbox{ tels que } \sum_{i=1}^{N_h} \mu_i \; a(\varphi_i,v_h)=l(v_h), \quad\forall v_h\in V_h
\ee
%
ou encore par linéarité de $a$ et $l$ :
\be
\hbox{Trouver } \mu_1,\ldots,\mu_{N_h} \hbox{ tels que } \sum_{i=1}^{N_h} \mu_i \; a(\varphi_i,\varphi_j)=l(\varphi_j), \quad\forall j=1,\ldots,N_h
\ee
%
c'est \`a dire résoudre le système linéaire
\be
\left(
\begin{array}{ccc}
a(\varphi_1,\varphi_1) & \cdots & a(\varphi_{N_h},\varphi_1)\\
\vdots & & \vdots\\
a(\varphi_1,\varphi_{N_h}) & \cdots & a(\varphi_{N_h},\varphi_{N_h})\\
\end{array}\right)
\left(
\begin{array}{c}
\mu_1\\
\vdots\\
\mu_{N_h}\\
\end{array}\right)
=
\left(
\begin{array}{c}
l(\varphi_1)\\
\vdots\\
l(\varphi_{N_h})\\
\end{array}\right)
\ee
%
soit
\be
A\mu = b
\label{eq:lin}
\ee
%
La matrice $A$ est a priori pleine. Toutefois, pour limiter le volume de
calculs, on va définir des fonctions de base $\varphi_i$ dont le support
sera petit, c'est \`a dire que chaque fonction $\varphi_i$ sera nulle partout
sauf sur quelques mailles. Ainsi les termes $a(\varphi_i,\varphi_j)$ seront le
plus souvent nuls, car correspondant \`a des fonctions $\varphi_i$ et
$\varphi_j$ de supports disjoints. La matrice $A$ sera donc une matrice
creuse, et on ordonnera les $\varphi_i$ de telle sorte que $A$ soit \`a
structure bande, avec une largeur de bande la plus faible possible.
%
A ce niveau, les difficultés majeures en pratique sont de trouver les
$\varphi_i$ et de les manipuler pour les calculs d'intégrales nécessaires
\`a la construction de $A$. Sans rentrer pour le moment dans les détails, on
peut toutefois indiquer que la plupart de ces difficultés seront levées
gr\^ace \`a trois idées principales :
\begin{itemize}
\item {\bf Le principe d'unisolvance} --- On s'attachera \`a trouver des
  degrés de liberté (ou ddl) tels que la donnée de ces ddl détermine
  de fa\c{c}on univoque toute fonction de $V_h$. Il pourra s'agir par exemple
  des valeurs de la fonction en quelques points. Déterminer une fonction
  reviendra alors \`a déterminer ses valeurs sur ces ddl.
%
\item {\bf Définition des $\varphi_i$} --- On définira les fonctions de
  base par $\varphi_i=1$ sur le $i^{\hbox{\tiny ème}}$ ddl, et $\varphi_i=0$
  sur les autres ddl. La manipulation des $\varphi_i$ sera alors très
  simplifiée, et les $\varphi_i$ auront par ailleurs un support réduit \`a
  quelques mailles.
%
\item {\bf La notion de ``famille affine d'éléments''} --- Le maillage
  sera tel que toutes les mailles soient identiques \`a une transformation
  affine près. De ce fait, tous les calculs d'intégrales pourront se
  ramener \`a des calculs sur une seule maille ``de référence'', par un
  simple changement de variable.
%
\end{itemize}
%
\section{Retour \`a l'exemple 1-D}
\label{sec:retour-a-lexemple}

%
\noindent
%
On reprend le problème 1-D (\ref{eq:modele-1D}).  On a écrit sa formulation
variationnelle (\cf \S\ref{sec:modele-1D}) et montré (\cf \S
\ref{sec:modele-1D2}) qu'elle admet une solution unique. On s'intéresse à
présent à la construction de l'espace d'approximation $V_h$.

\subsection{Construction du maillage}
\label{sec:constr-du-maill}


La première étape consiste à construire un maillage de $\Omega = ]a,b[$ en définissant une
subdivision (pas nécessairement régulière) $a=x_0 < x_1 < \ldots < x_N <
x_{N+1}=b$. Le maillage est donc une collection indexée de \Nma ($=N$) intervalles
$$\{I_i=[x_{i,1},x_{i,2}]\}_{i=1,...\Nma}$$ et on a
\begin{equation}
  \label{eq:1}
  [a,b]=\cup_{i=1}^\Nma [x_{1,i},x_{2,i}] \quad \mbox{et} \quad
  ]x_{1,i},x_{2,i}[ \cap ]x_{1,j},x_{2,j}[ = \emptyset \quad \mbox{ pour } i
  \neq j
\end{equation}

\begin{definition}
  Les intervalles $I_i$ sont appelées de \emph{mailles} ou \emph{éléments} ou
  \emph{cellules} du maillage, on a noté $\Nma$ le nombre de maillage
\end{definition}
\begin{definition}
  \label{def:23}
  Les points $x_i$ sont appelés les \emph{sommets} du maillage, on
  note $\Nso=N+1$ le nombre de sommets
\end{definition}

On note $h_i = x_{i+1}-x_i$ et $h = \max_{1\leq i \leq \Nma} h_i$. le maillage
est dit \emph{uniforme} si $h_i=h$ pour tout $i=\{1,...,\Nma\}$. Enfin on note
$\calTh=\{I_i\}_{i=\{1,...,\Nma\}}$, $h$ représentant la finesse globale du
maillage.

\begin{remark}
  \label{rem:1}
  En 1D on a $\Nso = \Nma+1$, en dimension supérieure des relations existent
  entre le nombre de sommets et de mailles en fonction du type de maille, ce
  sont les \emph{relations d'Euler}.
\end{remark}


\subsection{Construction de l'espace d'approximation}
\label{sec:constr-de-lesp}

L'étape suivante est de choisir les \emph{fonctions de forme} ou
\emph{fonctions de base} sur chaque maillage. On choisit les fonctions de
$V_h$ telle que leur restriction sur chaque maillage soit un \emph{espace
polynomial}.

\begin{definition}{\textbf{Espaces \Pk}}
  \label{def:24}
  Soit un entier $k \leq 1$. En dimension 1, on appelle \Pk l'espace vectoriel
  des polynômes à coefficients réels de degré inférieur ou égal à $k$
\end{definition}

On pose alors
\begin{equation}
  \label{eq:2}
  W_h = \{w_h \in L^2(\Omega); \forall i \in \{ 1,...,\Nma\}, {w_h}_{|I_i} \in \Pk\}
\end{equation}
$W_h$ est un espace de dimension finie égale à $(k+1)*\Nma$ mais il n'est pas
inclus dans $H^1_0(\Omega)$ et ne peut donc pas être utilisé pour
l'approximation du problème~(\ref{eq:FV}). En effet les fonctions de $w_h \in W_h$
peuvent être discontinues aux interfaces entre les maillages et un résultat
d'analyse fonctionnelle montre que dans ces conditions $w_h \ni
H^1(\Omega)$. Par ailleurs les fonctions de $W_h$ ne sont pas nécessairement
nulles au bord de $\Omega$. On pose donc
\begin{equation}
  \label{eq:3}
  V_h = W_h \cap H^1_0(\Omega).
\end{equation}
en d'autres termes, en dimension, on a
\begin{equation}
  \label{eq:5}
  V_h = \left\{ v_h \in {\cal C}^0 (a,b) \; ; \; {v_h}_{|I_i} \in \Pk \hbox{ et } v_h(a)=v_h(b)=0 \right\}
\end{equation}
%
Le problème approché sur $V_h$ est :
\begin{equation}
  \label{eq:11}
  ({\cal Q}_h)\qquad \hbox{Trouver } u_h\in V_h \hbox{ tel que } a(u_h,v_h)=l(v_h), \quad\forall v_h\in V_h
\end{equation}

On s'intéresse à présent à des exemples concrets d'espaces d'approximations
dans les deux sections suivantes \ref{sec:element-fini-de} et
\ref{sec:element-fini-de-1}.

\subsection{Element fini de Lagrange \Pk[1]}
\label{sec:element-fini-de}

On introduit les espaces vectoriels suivants:
\begin{equation}
  \label{eq:7}
  \Pch[1] = \{ v_h \in C^0(\Omega);\; \forall i \in \{ 1,...,\Nma\} {v_h}_{|I_i} \in \Pk[1]  \}
\end{equation}
et
\begin{equation}
  \label{eq:8}
  \Pcho[1] = \{ v_h \in \Pch[1];\; v_h(a)=v_h(b)=0 \}
\end{equation}

Les éléments de ces espaces sont des fonctions \emph{continues} et affines par
morceaux. Ils sont dérivables par morceaux sur chaque maille et ils sont
continus aux interfaces entre les mailles.
On a le résultat d'analyse fonctionnelle suivant:
\begin{theorem}
  \label{thr:3}
  $\Pch[1] \subset H^1(\Omega)$ et $\Pcho[1] \subset H^1_0(\Omega)$.
\end{theorem}

On introduit la famille de fonctions $\{\varphi_1,...,\varphi_\Nso\}$ que l'on
définit sur chaque maille de la manière suivante, pour tout $i  \in
\{2,...,\Nso-1\}$,
\begin{equation}
  \label{eq:18}
  \varphi_i(x) = \left\{
    \begin{split}
      \ds{\frac{1}{h_{i-1}} (x-x_{i-1})} & \mbox{ si } x \in I_{i-1}\\
      \ds{\frac{1}{h_{i}} (x_{i+1}-x)} & \mbox{ si } x \in I_{i}\\
      0 & \mbox{ sinon},
    \end{split}
  \right.
\end{equation}
et
\begin{equation}
  \label{eq:19}
  \begin{split}
  \varphi_1(x) &= \left\{
    \begin{split}
      \ds{\frac{1}{h_{1}} (x_2-x)} & \mbox{ si } x \in I_{1}\\
      0 & \mbox{ sinon},
    \end{split}
  \right.\\
  \varphi_\Nso(x) &= \left\{
    \begin{split}
      \ds{\frac{1}{h_{\Nso-1}} (x-x_{\Nso-1})} & \mbox{ si } x \in I_{\Nso-1}\\
      0 & \mbox{ sinon},
    \end{split}
  \right.
  \end{split}
\end{equation}

\begin{remark}
  \label{rem:6}
  Les fonctions $(\varphi_i)_{i=1,...,\Nso}$ sont dans $\Pch[1]$ et
  $(\varphi_i)_{i=2,...,\Nso-1}$ sont dans $\Pcho[1]$.
\end{remark}
\begin{remark}
  \label{rem:7}
  Les fonctions $(\varphi_i)_{i=1,...,\Nso}$ satisfont les relations
  \begin{equation}
    \label{eq:20}
    \varphi_i(x_j) = \delta_{ij},\quad i,j \in \{1,...,\Nso\},
  \end{equation}
  où $\delta_{ij}$ désigne le symbole de Kronecker tel que $\delta_{ij} = 1$
  si $i=j$ et $\delta_{ij}=0$ si $i \neq j$.
\end{remark}

Les fonctions $\varphi_i$ sont appelées \emph{fonctions chapeau} du fait de
leur graphe, voir figure~\ref{fig:chapeau}.
%
\begin{figure}[h]
\begin{center}
\includegraphics[width=0.48\linewidth]{FIG/chapeau.jpg}
%\vspace*{4 cm}
\caption{Fonction de base $\varphi_i$}
\label{fig:chapeau}
\end{center}
\end{figure}


\begin{proposition}
  \label{prop:4}
  \begin{enumerate}
  \item   La famille $\{\varphi_1,...,\varphi_\Nso\}$ est une base de $\Pch[1]$.
  \item   La famille $\{\varphi_2,...,\varphi_{\Nso-1}\}$ est une base de $\Pcho[1]$.
  \end{enumerate}
\end{proposition}
\begin{corollary}
  $\dim \Pch[1] = \Nso = \Nma+1$ et $\dim \Pcho[1] = \Nso-2 = \Nma-1$.
\end{corollary}

On introduit l'\emph{opérateur d'interpolation} suivant:
\begin{equation}
  \label{eq:21}
  \Ich[1] : \Ck[0](\bar{\Omega}) \ni v \mapsto \sum_{i=1}^\Nso v(x_i)
  \varphi_i \in \Pch[1].
\end{equation}
Pour toute fonction $v \in \Ck[0](\bar{\Omega})$, $\Ich[1] v$ est la seule
fonction continue affine par morceaux prenant les mêmes valeurs que $v$ aux
sommets $x_i, i=1,...,\Nso$.  $\Ich[1] v$ est appelée l'\emph{interpolé de
Lagrange} de $v$ de degré $1$.

\begin{remark}
  \label{rem:8}
  En dimension 1, les fonctions de $H^1(\Omega)$ sont \emph{continues}, on
  peut donc voir \Ich[1] comme un opérateur de $H^1(\Omega)$ dans
  $H^1(\Omega)$. On montre qu'il est continu et que sa norme
  $\|\Ich[1]\|_{\mathcal{L}(H^1(\Omega),H^1(\Omega))}$ est uniformément bornée en
  $h$, c'est à dire qu'il existe une constante $c$, indépendante de $h$, telle
  que pour tout $v \in H^1(\Omega)$
  \begin{equation}
    \label{eq:22}
    \|\Ich[1] v \|_{1,\Omega} \leq c \|v\|_{1,\Omega}
  \end{equation}
\end{remark}

\subsubsection{Estimation de l'erreur d'interpolation}
\label{sec:estim-de-lerr}

\begin{proposition}
  \label{prop:5}
  Pour tout $h$, et tout $v \in H^2(\Omega)$, on a
  \begin{equation}
    \label{eq:23}
    \|v - \Ich[1] v\|_{0,\Omega} \leq h^2 |v|_{2,\Omega}\quad \mbox{ et }\quad |v - \Ich[1] v|_{1,\Omega} \leq h |v|_{2,\Omega}
  \end{equation}
\end{proposition}

On dit que l'erreur d'interpolation est d'ordre 2 en norme $L^2$ et d'ordre 1
en semi-norme $H^1$ et donc en norme $H^1$.


\subsection{Element fini de Lagrange \Pk}
\label{sec:element-fini-de-1}

On introduit les espaces vectoriels suivants:
\begin{equation}
  \label{eq:9}
  \Pch = \{ v_h \in C^0(\Omega);\; \forall i \in \{ 1,...,\Nma\}, {v_h}_{|I_i} \in \Pk\}
\end{equation}
et
\begin{equation}
  \label{eq:10}
  \Pcho = \{ v_h \in \Pch;\; v_h(a)=v_h(b)=0 \}
\end{equation}

\subsubsection{Operateur d'interpolation}
\label{sec:oper-dint}

On introduit l'\emph{opérateur d'interpolation} suivant:
\begin{equation}
  \label{eq:24}
    \Ich : \Ck[0](\bar{\Omega}) \ni v \mapsto \sum_{i=1}^\Nno v(x_i)  \varphi_i \in \Pch.
\end{equation}
Pour toute fonction $v \in \Ck[0](\bar{\Omega})$, $\Ich v$ est la seule
fonction continue polynomial de degré $k$ par morceaux prenant les mêmes
valeurs que $v$ aux sommets $x_i, i=1,...,\Nso$. $\Ich v$ est appelée
l'\emph{interpolé de Lagrange} de $v$ de degré $k$.

\begin{remark}
  \label{rem:9}
  En dimension 1, les fonctions de $H^1(\Omega)$ sont \emph{continues}, on
  peut donc voir \Ich comme un opérateur de $H^1(\Omega)$ dans
  $H^1(\Omega)$. On montre qu'il est continu et que sa norme
  $\|\Ich\|_{\mathcal{L}(H^1(\Omega),H^1(\Omega))}$ est uniformément bornée en
  $h$, c'est à dire qu'il existe une constante $c$, indépendante de $h$ mais
  dépendante de $k$, telle
  que pour tout $v \in H^1(\Omega)$
  \begin{equation}
    \label{eq:25}
    \|\Ich v \|_{1,\Omega} \leq c \|v\|_{1,\Omega}
  \end{equation}
\end{remark}


Le résultat suivant permet d'estimer la précision de l'opérateur
d'interpolation \Ich,
\begin{proposition}
  \label{prop:2}
  Il existe une constante $c$, indépendante de $h$ mais dépendante de $k$,
  telle que pour tout $h$ et pour tout $v \in H^{k+1}(\Omega)$, on a
  \begin{equation}
    \label{eq:14}
    \|v - \Ich{v}\|_{0,\Omega} + h |v - \Ich{v}|_{1,\Omega}  \leq c\; h^{k+1}\; |v|_{k+1,\Omega}
  \end{equation}
  et
  \begin{equation}
    \label{eq:15}
    \sum_{m=2}^{k+1} h^m \left( \sum_{i=0}^N |v - \Ich{v}|^2_{m,I_i}\right)^{1/2}  \leq c\; h^{k+1}\; |v|_{k+1,\Omega}
  \end{equation}
\end{proposition}
\begin{remark}
  \label{rem:4}
  L'estimation~(\ref{eq:14}) montre que l'erreur d'interpolation est d'ordre
  $k+1$ en norme $\|\cdot\|_{0,\Omega}$ et qu'elle est d'ordre $k$ en norme
  $|\cdot|_{1,\Omega}$. Elle est donc d'ordre $k$ en norme
  $\|\cdot\|_{1,\Omega}$.
\end{remark}

\subsection{Analyse de convergence}
\label{sec:analyse-de-conv}

Nous nous intéressons à présent à l'analyse de la convergence de $u_h$ du
problème approché de~(\ref{eq:11}) vers la solution $u$ du problème
exact~(\ref{eq:FV}) lorsque $V_h=\Pcho[1]$ ou plus généralement $V_h=\Pcho,\;
k\geq 1$.

\subsubsection{Estimation en norme $H^1$}
\label{sec:estimation-en-norm}

Il s'agit dans un premier temps d'estimer l'erreur $u-u_h$ en norme
$H^1$. Pour cela on part de l'estimation~(\ref{eq:cea}), on a
\begin{equation}
  \label{eq:12}
  \begin{split}
    \|u-u_h\|_{1,\Omega} &\leq c\; \inf_{v_h \in \Pcho} \|u-v_h\|_{1,\Omega}\\
    & \leq c\;  \|u-\Ich{u}\|_{1,\Omega}\\
    & \leq c\; h^k |u|_{k+1,\Omega}
  \end{split}
\end{equation}
pourvu que la solution exacte soit suffisamment régulière, c'est à dire $u \in
H^{k+1}(\Omega)$.

\begin{remark}
  \label{rem:2}
  On notera que $\Ich{u} \in \Pcho$ puisque $u \in H^1_0(\Omega)$ et donc que
  $\Ich{u}(a)=\Ich{u}(b)=0$.
\end{remark}

On a donc le résultat suivant
\begin{proposition}
  \label{prop:1}
  Soit un entier $k\geq 1$. On suppose que la solution du
  problème~(\ref{eq:FV}) est dans $H^{k+1}(\Omega)$. On note $u_h$ la solution
  du problème approché~(\ref{eq:11}) avec l'espace d'approximation $V_h =
  \Pcho$. \textbf{Alors}, il existe une constante $c$, indépendante de $h$,
  telle que
  \begin{equation}
    \label{eq:13}
        \|u-u_h\|_{1,\Omega} \leq c\; h^k |u|_{k+1,\Omega}
  \end{equation}
\end{proposition}

\begin{remark}
  \label{rem:3}
  On dit que l'estimation d'erreur~(\ref{eq:13}) est \emph{optimale} car elle
  est du même ordre que l'erreur d'interpolation en norme $H^1$, voir la
  proposition~\ref{prop:2}.
\end{remark}


\subsubsection{Estimation en norme $L^2$}
\label{sec:estimation-en-norme}

\begin{proposition}
  \label{prop:3}
  Avec les hypothèses de la proposition~\ref{prop:1} et en supposant que
  $\alpha \in \Ck[1](\bar{\Omega})$, \textbf{Alors}, il existe une constante
  $c$, indépendante de $h$, telle que
  \begin{equation}
    \label{eq:16}
    \|u-u_h\|_{0,\Omega} \leq c\; h^{k+1} |u|_{k+1,\Omega}
  \end{equation}
\end{proposition}

\begin{remark}
  \label{rem:5}
  On dit que l'estimation d'erreur~(\ref{eq:16}) est \emph{optimale} car elle
  est du même ordre que l'erreur d'interpolation en norme $L^2$, voir la
  proposition~\ref{prop:2}.
\end{remark}

\subsection{Formulation algébrique $V_h=\Pch[1]$}
\label{sec:form-algebr}

En décomposant la solution approchée $u_h$ sur cette base sous la forme
$\ds{u_h = \sum_{i=1}^N \mu_i \; \varphi_i}$, on obtient, comme au paragraphe
\ref{sec:general}, le système linéaire $A\mu=b$, avec : \be
\begin{array}{rcl}
A_{ji}=a(\varphi_i,\varphi_j) & = & \ds{\int_a^b \left[ \varphi_i'(x) \varphi_j'(x)  + c(x) \varphi_i(x) \varphi_j(x)\right]\; dx }\\
 & = & \ds{ \sum_{k=0}^N \int_{x_k}^{x_{k+1}} \left[\varphi_i'(x) \varphi_j'(x)  + c(x) \varphi_i(x) \varphi_j(x)\right]\; dx }
\end{array}
\ee
%
Le support de $\varphi_i$ étant réduit \`a $[x_{i-1},x_{i+1}]$, on en
déduit que \be \left\{
\begin{array}{lll}
a(\varphi_i,\varphi_j) & = & 0 \qquad \hbox{si }|i-j|\ge 2\\
& & \\
a(\varphi_i,\varphi_{i+1}) & = & \ds{ \int_{x_i}^{x_{i+1}} \left[ \varphi_i'(x) \varphi_{i+1}'(x)  + c(x) \varphi_i(x) \varphi_{i+1}(x)\right] \; dx}\\
& & \\
a(\varphi_i,\varphi_{i-1}) & = & \ds{ \int_{x_{i-1}}^{x_i} \left[ \varphi_i'(x) \varphi_{i-1}'(x)  + c(x) \varphi_i(x) \varphi_{i-1}(x)\right] \; dx}\\
& & \\
a(\varphi_i,\varphi_{i}) & = & \ds{ \int_{x_{i-1}}^{x_{i+1}} \left[ \varphi_i'^2(x) + c(x) \varphi_i^2(x)\right] \; dx}\\
\end{array}\right.
\ee
%
$A$ est donc tridiagonale.

\section{Exercices}
\label{sec:exercices}
%
\begin{enumerate}
\item Dans le \S\ref{sec:lax-milgram}, montrer que, dans le cas où $a$ est
  symétrique, si $u$ est solution du problème variationnel, alors elle est
  solution du problème de minimisation.
\item Montrer que $\nabla J[u](v) = a(u,v) - l(v)$.
\item Montrer que, si $a$ est coercive, la matrice $A$ de (\ref{eq:lin}) est
  inversible. (C'est donc la démonstration du théorème de Lax-Milgram en
  dimension finie.)
\item Pour l'exemple 1-D traité dans ce chapitre, démontrer qu'on est bien
  dans les hypothèses du théorème de Lax-milgram
\item Calculer explicitement la matrice $A$ pour cet exemple.
\item Pour le problème 2-D du \S \ref{sec:modele-2D}, montrer que la
  formulation variationelle (\ref{eq:FV2}) admet une solution unique, qui est
  aussi solution classique si $f \in H^2(\Omega)$.
\item Démontrer les résultats du \S\ref{sec:elliptique}
\end{enumerate}

\section{TP1}
\label{sec:tp1}
\input{exercices/exo1}

%%% Local Variables:
%%% coding: utf-8
%%% mode: latex
%%% TeX-PDF-mode: t
%%% TeX-parse-self: t
%%% TeX-auto-save: t
%%% x-symbol-8bits: nil
%%% TeX-auto-regexp-list: TeX-auto-full-regexp-list
%%% TeX-master: "mef-intro"
%%% ispell-local-dictionary: "american"
%%% End:

\chapter{Eléments finis de Lagrange}
%
%
\noindent
On va présenter dans ce chapitre le type le plus simple et le plus classique
d'éléments finis.
%
%
\section{Unisolvance}
%
\noindent
%
\begin{definition}
  \label{def:22}
  Soit $\Sigma=\{ a_1,\ldots,a_N\}$ un ensemble de $N$ points distincts de
  $\RR^n$. Soit $P$ un espace vectoriel de dimension finie de fonctions de
  $\RR^n$ à valeurs dans \RR. On dit que $\Sigma$ est $P$-unisolvant ssi
  pour tous réels $\alpha_1,\ldots,\alpha_N$, il existe un unique
  élément $p$ de $P$ tel que $p(a_i)=\alpha_i,\; i=1,\ldots,N$.
\end{definition}

Ceci revient à dire que la fonction :
\begin{equation}
\label{eq:26}
\begin{array}{rcl}
{\cal L} : P & \longrightarrow & \RR^N\\
p & \longrightarrow & (p(a_1),\ldots,p(a_N))
\end{array}
\end{equation}
est bijective.\saut
%
%
En pratique, on montrera que $\Sigma$ est $P$-unisolvant en vérifiant que
dim $P=$ Card $\Sigma$, puis en montrant l'injectivité ou la surjectivité
de ${\cal L}$.
%
L'injectivité de ${\cal L}$ se démontre en établissant que la seule
fonction de $P$ s'annulant sur tous les points de $\Sigma$ est la fonction
nulle.
%
La surjectivité de ${\cal L}$ se démontre en exhibant une famille
$p_1,\ldots,p_N$ d'éléments de $P$ tels que $p_i(a_j)=\delta_{ij}$, c'est
à dire un antécédent pour ${\cal L}$ de la base canonique de $\RR^N$. En
effet, étant donnés des réels $\alpha_1,\ldots,\alpha_N$, la fonction
$\ds{p=\sum_{i=1}^N \alpha_i p_i}$ vérifie alors $p(a_j)=\alpha_j,\;
j=1,\ldots,N$.
%
%
\section{Elément fini de Lagrange}
\label{sec:lagrange}
\noindent
%
%
\begin{definition}
  Un {\bf élément fini de Lagrange} est un triplet $(K,\Sigma,P)$ tel que
  \begin{itemize}
  \item $K$ est un élément géométrique de $\RR^n$ ($n=$ 1, 2 ou 3),
    compact, connexe, et d'intérieur non vide.
%
  \item $\Sigma=\{a_1,\ldots,a_N\}$ est un ensemble fini de $N$ points
    distincts de $K$.
%
  \item $P$ est un espace vectoriel de dimension finie de fonctions réelles
    définies sur $K$, et tel que $\Sigma$ soit $P$-unisolvant (donc dim $P =
    N$).
  \end{itemize}\label{def:25}
\end{definition}

%
\begin{definition}
  Soit $(K,\Sigma,P)$ un élément fini de Lagrange. On appelle {\bf
  fonctions de base locales} de l'élément les $N$ fonctions $p_i$
  ($i=1,\ldots,N$) de $P$ telles que
  \begin{equation}
    \label{eq:27}
    p_i(a_j)=\delta_{ij}\qquad 1\le i,j \le N.
  \end{equation}
  \label{def:26}
\end{definition}
%
On vérifie aisément que $(p_1,\ldots,p_N)$ ainsi définie forme bien une
base de $P$.
%
%
\begin{definition}
  On appelle {\bf opérateur de interpolation} (ou encore $P-$interpolation)
  sur $\Sigma$ l'opérateur $\pi_K$ qui, à toute fonction $v$ définie sur
  $K$, associe la fonction $\pi_K v$ de $P$ définie par $\ds{\pi_K v =
  \sum_{i=1}^N v(a_i)\, p_i}$. $\pi_K v$ est donc l'unique élément de $P$
  qui prend les mêmes valeurs que $v$ sur les points de
  $\Sigma$.\label{def:27}
\end{definition}

%
%
\section{Exemples d'éléments finis de Lagrange}
%
\subsection{Espaces de polyn\^omes}
\noindent
%
On notera $\Pk$ l'espace vectoriel des polyn\^omes de degré total inférieur ou égal à $k$.
%
\begin{itemize}
\item Sur \RR, $\Pk=\hbox{Vect} \{ 1,X,\ldots, X^k\}\quad$ et dim $\Pk=k+1$.
\item Sur \RR$^2$, $\Pk=\hbox{Vect} \{ X^i Y^j ; 0\le i+j \le k\}\quad$ et dim $\ds{ \Pk=\frac{(k+1)(k+2)}{2}}$.
\item Sur \RR$^3$, $\Pk=\hbox{Vect} \{ X^i Y^j Z^l ; 0\le i+j+l \le k\}\quad$ et dim $\ds{ \Pk=\frac{(k+1)(k+2)(k+3)}{6}}$.
\end{itemize}
%
On notera $\Qk$ l'espace vectoriel des polyn\^omes de degré inférieur ou égal à $k$ par rapport à chaque variable.
%
\begin{itemize}
\item Sur \RR, $\Qk=\Pk$.
\item Sur \RR$^2$, $\Qk=\hbox{Vect} \{ X^i Y^j ; 0\le i,j \le k\}\quad$ et dim $\ds{ \Qk=(k+1)^2}$.
\item Sur \RR$^3$, $\Qk=\hbox{Vect} \{ X^i Y^j Z^l ; 0\le i,j,l \le k\}\quad$ et dim $\ds{ \Qk=(k+1)^3}$.
\end{itemize}
%
%
\subsection{Exemples 1-D}
\label{par:ex1d}
%
\subsubsection{Elément $P_1$}
\begin{itemize}
\item $K=[a,b]$
\item $\Sigma=\{a,b\}$
\item $P=\Pk[1]$
\end{itemize}
%
\subsubsection{Elément $P_2$}
\begin{itemize}
\item $K=[a,b]$
\item $\ds{\Sigma=\{a,\frac{a+b}{2},b\}}$
\item $P=\Pk[2]$
\end{itemize}
%
%
\subsubsection{Elément $\Pk$}
\begin{itemize}
\item $K=[a,b]$
\item $\ds{\Sigma=\{a+i\,\frac{b-a}{m},\quad i=0,\ldots,k\}}$
\item $P=\Pk$
\end{itemize}
%
%
\subsection{Exemples 2-D triangulaires}
%
%
\subsubsection{Elément $\Pk[1]$}
\begin{itemize}
\item $K$=triangle de sommets $a_1, a_2, a_3$
\item $\Sigma=\{a_1,a_2,a_3\}$
\item $P=\Pk[1]$
\end{itemize}
%
Les fonctions de base sont définies par $p_i(a_j)=\delta_{ij}$. Ce sont donc
les coordonnées barycentriques : $p_i=\lambda_i$ (cf annexe \ref{ch:bary}).
%
%
\subsubsection{Elément $\Pk[2]$}
\begin{itemize}
\item $K$=triangle de sommets $a_1, a_2, a_3$
\item $\Sigma=\{a_1,a_2,a_3, a_{12}, a_{13}, a_{23}\}$,\/ o\`u $\ds{a_{ij}=\frac{a_i+a_j}{2}}$.
\item $P=\Pk[2]$
\end{itemize}
%
Les fonctions de base sont $p_i=\lambda_i (2\lambda_i -1)$ et
$p_{ij}=4\lambda_i\lambda_j$. Un exemple de calcul de ces fonctions de base
est donné en annexe \ref{ch:bary}.
%
\begin{figure}[h]
\begin{center}
\includegraphics[width=0.95\linewidth]{FIG/elements-2D.jpg}
%\vspace*{6.5 cm}
\caption{Eléments finis triangulaire $P_1$, triangulaire $P_2$ et rectangulaire $Q_1$}
\end{center}
\end{figure}
%
%
\subsection{Exemples 2-D rectangulaires}
%
\subsubsection{Elément $\Qk[1]$}
\begin{itemize}
\item $K$=rectangle de sommets $a_1, a_2, a_3, a_4$, de c\^otés parallèles aux axes
\item $\Sigma=\{a_1,a_2,a_3,a_4\}$
\item $P=\Qk[1]$
\end{itemize}
%
Les fonctions de base sont
$\ds{p_i(X,Y)=\frac{(X-x_j)(Y-y_j)}{(x_i-x_j)(y_i-y_j)}}$, o\`u $(x_i,y_i)$
sont les coordonnées de $a_i$, et o\`u $a_j$, de coordonnées $(x_j,y_j)$
est le coin opposé à $a_i$.
%
%%
\subsection{Exemples 3-D}
%
%
%
\subsubsection{Elément tétraèdrique $\Pk[1]$}
\begin{itemize}
\item $K$=tétraèdre de sommets $a_1, a_2, a_3, a_4$
\item $\Sigma=\{a_1,a_2,a_3, a_4\}$
\item $P=\Pk[1]$
\end{itemize}
%
%
\subsubsection{Elément tétraèdrique $\Pk[2]$}
\begin{itemize}
\item $K$=tétraèdre de sommets $a_1, a_2, a_3, a_4$
\item $\ds{ \Sigma=\{a_i\}_{1\le i\le 4} \cup \{a_{ij}\}_{1\le i < j \le 4} }$
\item $P=\Pk[2]$
\end{itemize}
%
Les fonctions de base sont $p_i=\lambda_i (2\lambda_i -1)$ et $p_{ij}=4\lambda_i\lambda_j$.
%
%
\subsubsection{Elément parallélépipèdique $Q_1$}
\begin{itemize}
\item $K$=parallélépipède de sommets $a_1, \ldots , a_8$ de c\^otés parallèles aux axes
\item $\Sigma=\{a_i\}_{1\le i\le 8}$
\item $P=\Qk[1]$
\end{itemize}
%
%
\subsubsection{Elément prismatique}
\begin{itemize}
\item $K$=prisme droit de sommets $a_1, \ldots , a_6$
\item $\Sigma=\{a_i\}_{1\le i\le 6}$
\item $P=\{p(X,Y,Z)=(a+bX+cY)+Z(d+eX+fY), \;\; a,b,c,d,e,f \in \RR\}$
\end{itemize}
%
%
\begin{figure}[h]
\begin{center}
\includegraphics[width=0.95\linewidth]{FIG/elements-3D.jpg}
%\vspace*{6.5 cm}
\caption{Eléments finis tétraèdriques $P_1$ et $P_2$, parallélépipèdique $Q_1$, et prismatique}
\end{center}
\end{figure}
%
%
\section{Famille affine d'éléments finis}
\noindent
%
%
\begin{definition}
  Deux éléments finis $(\hat{K},\hat{\Sigma},\hat{P})$ et $(K,\Sigma,P)$
  sont {\bf affine-équivalents} ssi il existe une fonction affine $F$
  inversible ($F: \hat{x} \longrightarrow B\hat{x}+b$) telle que
  \begin{itemize}
  \item $K=F(\hat{K})$
  \item $a_i=F(\hat{a}_i) \qquad i=1,\ldots,N$
  \item $P=\{ \hat{p}\circ F^{-1} , \quad \hat{p}\in \hat{P} \}$
  \end{itemize}\label{def:28}
\end{definition}
%
\begin{remark}
  Si l'on est dans $\RR^n$, $B$ est donc une matrice $n\times n$ inversible,
  et $b$ est un vecteur de $\RR^n$.\label{rem:10}
\end{remark}

%
{\bf Propriété :} Soient $(\hat{K},\hat{\Sigma},\hat{P})$ et
$(K,\Sigma,P)$ deux éléments finis affine-équivalents, via une
transformation $F$. On note $\hat{p}_i \; (i=1,\ldots,N)$ les fonctions de
base locales de $\hat{K}$. Alors les fonctions de base locales de $K$ sont les
$p_i=\hat{p}_i\circ F^{-1}$.
%
\begin{definition}
  On appelle {\bf famille affine d'éléments finis} une famille
  d'éléments finis tous affine-équivalents à un même élément
  fini $(\hat{K},\hat{\Sigma},\hat{P})$, appelé {\bf élément de
  référence}.\label{def:29}
\end{definition}

%
D'un point de vue pratique, le fait de travailler avec une famille affine
d'éléments finis permet de ramener tous les calculs d'intégrales à des
calculs sur l'élément de référence.
%
Les éléments de référence sont :
\begin{itemize}
\item En 1-D : le segment $[0,1]$
\item En 2-D triangulaire : le triangle unité, de sommets $(0,0)$, $(0,1)$ et $(1,0)$.
\item En 2-D rectangulaire : le carré unité $[0,1]\times[0,1]$.
\item En 3-D tétraèdrique : le tétraèdre unité, de sommets $(0,0,0)$, $(1,0,0)$, $(0,1,0)$ et $(0,0,1)$.
\item En 3-D parallélépipèdique : le cube unité $[0,1]\times[0,1]\times[0,1]$.
\item En 3-D prismatique : le prisme unité de sommets $(0,0,0)$, $(0,1,0)$, $(1,0,0)$, et $(0,0,1)$, $(0,1,1)$, $(1,0,1)$.
\end{itemize}

%
%

\subsection{Maillages}
\label{sec:maillages}

Nous étendons ici aux dimension 2 et 3 les notions élémentaires de maillage
vues en 1D, voir la figure~\ref{fig:2}.

\begin{figure}[htbp]
  \centering
  \subfigure[2D]{\includegraphics[width=.45\linewidth]{contact2d-crop}}
  \subfigure[3D]{\includegraphics[width=.45\linewidth]{contact3d-crop}}
  \caption{Un maillage en 2D et en 3D}
  \label{fig:2}
\end{figure}

\begin{definition}
  \label{def:31}
  Un maillage est constituée d'une famille d'éléments(ou mailles ou cellules)
  $\{K_e\}_{e=1,...,N_e}$ où $N_e$ est le nombre d'éléments, nous noterons
  \begin{equation}
    \label{eq:32}
    \calTh = \{K_m\}_{m=1,...,N_e}
  \end{equation}
  avec
  \begin{equation}
    \label{eq:33}
    h=\max_{1\le e\le N_e} h_{K_e}
  \end{equation}
  et
  \begin{equation}
    \label{eq:34}
    h_{K_e}     = \diam(K_e) = \max_{x_1,x_2 \in K_e} \|x_1-x_2\|,\, e \in \{1,...,\Ne\}
  \end{equation}

\end{definition}

On travaille par la suite avec des familles de maillage et on les note
$\set{\mathcal{T}_h}_{h > 0}$.

\begin{definition}[Maillage Quasi-uniforme]
  On dira qu'une famille de maillage $\set{\mathcal{T}_h}_{h > 0}$ est
  \textrm{quasi-uniforme} s'il existe une constante $c$ telle que
  \begin{equation}
    \label{eq:35}
    \forall h,\ \forall K \in \calTh,\ h_K \geq c h
  \end{equation}
\end{definition}

\begin{remark}
  Cela veut dire que les élements sont tous de la même taille pour $h$
  donné.\label{rem:12}
\end{remark}

\subsection{Transformation géométrique}
\label{sec:transf-geom}

Un maillage est généré par
\begin{enumerate}
\item un \emph{élément de reference} noté $\hat{K}$
\item une famille de  \emph{transformations géométriques} mappant $\hat{K}$
  vers les éléments $K_e, e=1,\ldots,\Ne$ dans le maillage
\end{enumerate}

Nous supposerons que les transformations sont des $\mathcal{C}^1-$
diffeomorphismes \footnote{la transformation et son inverse sont
$\mathcal{C}^1$ et bijectives}

\begin{definition}
  \label{def:32}
  Pour une cellule $K \in \mathcal{T}_h$, on note $T_K$ la  transformation géométrique
  \begin{equation}
    \label{eq:36}
    T_K: \hat{K} \mapsto K
  \end{equation}
\end{definition}

Afin de spécifier la transformation géométrique, on  considère l'élément fini
de Lagrange, noté $(\hat{K},\hat{P}_{\mathrm{geo}},
\hat{\Sigma}_{\mathrm{geo}})$, tel que
  \begin{itemize}
  \item $\ngeo = \card{\hat{\Sigma}_{\mathrm{geo}}}$
  \item $\set{\hat{g}_1,\ldots,\hat{g}_{\ngeo}}$ les noeuds de $\hat{K}$
  \item $\set{\hat{\psi}_1,\ldots,\hat{\psi}_{\ngeo}}$ les fonctions de forme
  \end{itemize}

\begin{definition}[Élément fini géométrique]
  On dit que
  \begin{itemize}
  \item $(\hat{K},\hat{P}_{\mathrm{geo}}, \hat{\Sigma}_{\geo})$ est
    l'\emph{élément fini géométrique},
  \item $\set{\hat{g}_1,\ldots,\hat{g}_{\ngeo}}$ sont les  \emph{noeuds géométriques} et
  \item $\set{\hat{\psi}_1,\ldots,\hat{\psi}_{\ngeo}}$ sont les
    \emph{fonctions de formes géométriques}
  \end{itemize}
\end{definition}
\begin{figure}[htbp]
  \centering
  \centerline{\includegraphics[width=.5\linewidth]{pdfs/pena/py_figures/geomap_triangle}}
  \caption{Transformation géométrique associée à un triangle}
  \label{fig:3}
\end{figure}

Pour chaque $K \in \mathcal{T}_h$, on a un $\ngeo$-uplet
$\set{g^K_1,\ldots,g^K_\ngeo}$. La transformation géométrique est définie
comme suit
\begin{equation}
  \label{eq:37}
  T_K: \hat{x} \in \hat{K} \mapsto \sum_{i=1}^\ngeo\ g^K_i \hat{\psi}_i(\hat{x})
\end{equation}
et en particulier on a
\begin{equation}
  \label{eq:38}
  T_K(\hat{g}_i) = g^K_i, \quad \forall i \in \set{1,\ldots,\ngeo}
\end{equation}
\begin{remark}
  \label{rem:13}
  On a $T_K \in [\hat{P}_\geo(\hat{K})]^d$ et que  $\set{g^K_1,\ldots,g^K_\ngeo}$ sont les \emph{noeuds géométriques} de $K$.
\end{remark}

$T_K$ est un $\mathcal{C}^1$-diffeomorphism donc la \emph{numérotation} des noeuds
$\set{g^K_1,\ldots,g^K_\ngeo}$ doit être  \emph{compatible} avec les noeuds de
l'élément finit géométrique.

\begin{minipage}[l]{.3\linewidth}
\centerline{\begin{tikzpicture}
  \filldraw [gray] (2,0) circle (2pt) node [right] (a) {3}
  (1,2) circle (2pt) node[right] (b) {2}
  (0,0) circle (2pt) node[above] (c) {1};
  \node[] at (1,1) (x1) {$x$};
  \draw (2,0) -- (1,2) -- (0,0) -- (2,0);
  \node[] at (1,0.6666) {$K'$};
\end{tikzpicture}}
\centerline{Numérotation pas OK}
\end{minipage}
\begin{minipage}[c]{.3\linewidth}
\centerline{\begin{tikzpicture}
  \filldraw [gray] (-1,-1) circle (2pt) node [left] {1}
  (1,-1) circle (2pt) node[right] {2}
  (-1,1) circle (2pt) node[above] {3};
  \node[] at (-0.4,0) (hatx) {$\hat{x}$};
  \draw (-1,-1) -- ( 1,-1 ) -- (-1,1) --(-1,-1);
  \node[] at (-0.3333,-0.333) {$\hat{K}$};
\end{tikzpicture}}
\end{minipage}
\begin{minipage}[r]{.3\linewidth}
\centerline{\begin{tikzpicture}
  \filldraw [gray] (2,0) circle (2pt) node [right] (a) {1}
  (1,2) circle (2pt) node[right] (b) {2}
  (0,0) circle (2pt) node[above] (c) {3};
  \node[] at (1,1) (x2) {$x$};
  \draw (2,0) -- (1,2) -- (0,0) -- (2,0);
  \node[] at (1,0.6666) {$K$};
\end{tikzpicture}}
\centerline{Numérotation OK}
\end{minipage}


\begin{tikzpicture}[overlay]
  \path[->] (hatx) edge [bend left] node[auto] {$T_K$}(x2);
  \path[->] (hatx) edge [bend right] node[auto] {$T_{K'}$}(x1);
\end{tikzpicture}

\begin{remark}
  \label{rem:15}
  La numérotation locale des entités géométriques dans \Feel \textbf{doit}
  être consistente avec la numérotation locale des générateurs de
  maillafe. voir
  \href{http://www.geuz.org/gmsh/doc/texinfo/gmsh.html#Node-ordering}{\color{blue}$\triangleright$~format
  de fichier Gmsh} pour une numérotation locale.
\end{remark}


Un cas particulier est la transformation géométrique affine

\begin{definition}[Maillage affine]
  \label{def:33}
  Quand toutes les  \emph{transformations géométriques} $\set{T_K}_{K \in
  \mathcal{T}_h}$ sont \emph{affines}, cela veut dire que pour tout $K \in
  \mathcal{T}_h$, il existe un  vecteur $b_K \in \R[d]$ et une matrice $J_K \in
  \R[{d,d}]$ tels que
  \begin{equation}
    \label{eq:39}
    T_K : \hat{x} \in \hat{K} \mapsto b_K + J_K \hat{x}  \in K
  \end{equation}
  On dit que le maillage est \emph{affine}.
\end{definition}
\begin{remark}[Exemple]
  \label{rem:16}
  Si l'élément fini géométrique est $(\hat{K},\poly{P}_1,\Sigma_\ngeo)$ alors
  les éléments $K$ sont soit des triangles soit des tétrahèdres.
\end{remark}

\begin{remark}[\Feel]
  \label{rem:17}
  \lstinline!Mesh<Simplex<d,1> >! ou \lstinline!Mesh<Simplex<d> >! est le type
  pour les maillages affines formés de simplexes dans $\R[d]$.
  \lstinline!1! indique l'ordre de l'\emph{élément fini géométrique} (here 1).
\end{remark}

\subsection{Quelques calculs avec la transformation géométrique}
\label{sec:quelq-calc-avec}

\subsubsection{Gradient, Inverse et Jacobien}
\label{sec:gradient-inverse-et}

On note $\xi$ un ensemble de  $n$ points dans $\hat{K}$ et on note $\nabla
T_K(\xi)$ le gradient de $T_K$ aux points $\xi$
\begin{equation}
  \label{eq:40}
  \nabla T_K( \xi )\ =\ \sum_{i=0}^{\ngeo}\ g^K_i\ \nabla \psi_i (\xi)
\end{equation}
et $B_K(\xi) = \nabla T_K^{-1}(\xi)$ l'inverse $\xi$
et finalement $J_K(\xi)$ le jacobien de $T_K$ en $\xi$
\begin{equation}
\label{eq:41}
J_K(\xi)\ =\ |\det( \nabla T_K(\xi) )|
\end{equation}

\subsubsection{Dérivation dans l'élément de référence}
\label{sec:deriv-dans-lelem}

Afin de dériver un polynome dans l'élément réel $K$, grâce à la transformation
géométrique et la règle de différentiation des fonctions composées, nous
dérivons seulement dans l'élément de référence $\hat{K}$.

Soit $f: K \mapsto \R[]$ et $\hat{f}: \hat{K} \mapsto \R[]$ telle
que $\hat{f} = f \circ T_K$
\begin{equation}
  \label{eq:42}
  \nabla f\ =\  \hat{\nabla} \underbrace{\hat{f}(\xi)}_{f \circ T_K(\xi)} B_K(\xi)
\end{equation}

en 2D, on a
\begin{equation}
  \label{eq:43}
  \nabla f(x) =
  \begin{pmatrix}
    \frac{\hat{\partial} \hat{f} (\xi)}{\partial \xi_1} & \frac{\hat{\partial} \hat{f} (\xi)}{\partial \xi_2}
  \end{pmatrix}
  \begin{pmatrix}
    B_{K_{11}}(\xi) & B_{K_{12}}(\xi)\\
    B_{K_{21}}(\xi) & B_{K_{22}}(\xi)\\
  \end{pmatrix}
\end{equation}
avec $x=T_K(\xi)$
% \end{block}

\subsubsection{Intégration dans l'élément de référence}
\label{sec:integr-dans-lelem}

De manière similaire, au lieu de calculer les intégrales sur l'élément réel
$K$, nous appliquons un changement de variables et calculons les intégrales
sur l'élément de réference $\hat{K}$.

Soit $f: K \mapsto \R[]$ et $\hat{f}: \hat{K} \mapsto \R[]$ telle que
 $\hat{f} = f \circ T_K$, et ${\bm F}: K \mapsto
\R[d]$ et ${\hat{\bm F}}: \hat{K} \mapsto \R[d]$ telle que
$\hat{{\bm F}} = {\bm F} \circ T_K$, on a alors les relations suivantes

\begin{equation}
  \label{eq:44}
  \int_{K} \ f\ dx\ =\ \int_{\hat{K}} f(T_K(\xi) ) J_K( \xi )\ d \xi \ =\ \int_{\hat{K}} \hat{f}(\xi) J_K( \xi )\ d \xi
\end{equation}
\begin{equation}
  \label{eq:45}
  \int_{K}\ \nabla f\ dx\ =\ \int_{\hat{K}} \Big(\hat{\nabla} \underbrace{\hat{f}(\xi)}_{f \circ T_K(\xi)} B_K(\xi)\Big) J_K( \xi )\ d \xi
\end{equation}
\begin{equation}
  \label{eq:46}
  \int_{\partial K}\ f( x )\ dx = \int_{\partial \hat{K}} \hat{f}(\xi)\  \| B_K(\xi) \ {\hat{\bm n}}(\xi) \|\ J_K( \xi )\ d \xi
\end{equation}
\begin{equation}
  \label{eq:47}
  \int_{\partial K}\ {\bm F}( x )\ \cdot\ {\bm n}(x) dx = \int_{\partial \hat{K}} {\hat{\bm F}}( \xi )\  \cdot \Big(B_K(\xi) \ {\hat{\bm n}}(\xi) \Big) \ J_K( \xi )\ d \xi
\end{equation}
où ${\bm n}(x)$ est la \emph{normale extérieure unitaire} à $\partial K$
évaluée en $x \in \partial K$, et ${\hat{\bm n}}(\xi)$ la normale unitaire extérieure
à  $\hat{K}$ évaluée en $\xi \in \partial \hat{K}$.

\begin{remark}
  \label{rem:18}
  \Feel effectue automatiquement pour vous les changements de variables dans
  les intégrales.
\end{remark}


\begin{definition}[Maillages conformes]
  \label{def:34}
  Un maillage est dit \emph{conforme} si l'intersection de deux éléments est
  soit vide, un sommet, une arête ou une face.
\end{definition}

\begin{minipage}[l]{.3\linewidth}
  \centerline{
  \begin{tikzpicture}
    \filldraw [gray] (0,0) circle (2pt)
    (0.5,1) circle (2pt)
    (1,0) circle (2pt)
    (1.5,0.7) circle (2pt);
    \draw (0,0) -- (0.5,1) -- (1,0) -- (1.5,0.7) -- (0.5,1);
    \draw (0,0) -- (1,0);
  \end{tikzpicture}}
  \centerline{Conforme}
\end{minipage}
\begin{minipage}[c]{.3\linewidth}
  \centerline{\begin{tikzpicture}
    \filldraw [gray] (0,0) circle (2pt)
    (0.5,1) circle (2pt)
    (1,0) circle (2pt)
    (1.5,0.7) circle (2pt)
    (0.75,0.5) circle (2pt)
    ;
    \draw (0,0) -- (0.5,1) -- (1,0) -- (1.5,0.7) -- (0.5,1);
    \draw (0,0) -- (1,0);
    \draw (1.5,0.7) -- (0.75,0.5);
  \end{tikzpicture}}
  \centerline{Pas conforme}
\end{minipage}

\begin{remark}[\Feel]
  \label{rem:19}
  On ne manipule que des maillages conformes dans le cours mais \Feel peut
  traiter des maillages non-conformes par exemple dans le contexte de méthode
  de décomposition de domaines.
\end{remark}


\subsection{Génération des éléments finis de Lagrange}


\begin{itemize}
\item Soit $\mathcal{T}_h$ un maillage généré par $(\hat{K},
  \hat{P}_{\mathrm{geo}}, \hat{\Sigma}_{\mathrm{geo}})$
\item une cellule $K \in \mathcal{T}_h$ est alors l'image de  $\hat{K}$ par la
  transformation géométrique $T_K$ défini par \eqref{eq:23}
\end{itemize}

L'objectif à présent est de générer la famillage l'éléments finis de Lagrange
grâce à l'élément fini de référence $(\hat{K},\hat{P}, \hat{\Sigma})$
\begin{equation}
  \label{eq:28}
  \{(K,P_K,\Sigma_K)\}_{K \in \mathcal{T}_h}
\end{equation}
\begin{itemize}
\item on note $\{\hat{x}_1,...,\hat{x}_{\nf}\}$ les noeuds de l'élément fini
\item on note $\{\hat{\Psi}_1,...,\hat{\Psi}_{\nf}\}$ les fonctions de forme
  élément fini
\end{itemize}

\begin{proposition}
  \label{prop:6}
  \begin{itemize}
  \item Soit $K \in \mathcal{T}_h$ et $P_K=\{\hat{p} \circ T_K^{-1}; \hat{p} \in \hat{P}\}$
  \item Pour tout $i \in \{1,...,\nf\}$, on pose $x_{K,i} = T_K(\hat{x}_i)$
  \item $\Sigma_K$ est l'ensemble des \emph{degrés de liberté} associé à $\{x_{K,1},...,x_{K,\nf}\}$
  \end{itemize}
  Alors $(K,P_K,\Sigma_K)$ est un \emph{élément fini de Lagrange}.
  Les  \emph{fonctions de forme} sont définies de la fa\c {c}on suivante
  \begin{equation}
    \label{eq:49}
    \Psi_{K,i} = \hat{\Psi}_i \circ T_K^{-1}, \quad i=1,...,\nf
  \end{equation}
  et \Ilag{K}  l'\emph{opérateur d'interpolation local}  comme
    \begin{equation}
      \label{eq:50}
      \Ilag{K}: v \in \mathcal{C}^0(K) \mapsto \sum_{i=1}^\nf\ v(x_{K,i})\ \Psi_{K,i}\
      \in P_K
    \end{equation}
    Une propriété importante de \Ilag{K} est que
    \begin{equation}
      \label{eq:51}
      \forall v \in \mathcal{C}^0(K),\quad \Ilag{K}( v \circ T_K ) =
      \Ilag{K}( v ) \circ T_K.
    \end{equation}
  \end{proposition}

  \begin{theorem}[Propriété de l'interpolateur local]
    \label{thr:14}
    \begin{itemize}
    \item Soit $T_K$ une transformation affine
    \item Soit $\mathbb{P}_k \subset \hat{P}$ et $k+1 > \frac{d}{2}$
    \item Soit $h_K$ le diamètre de $K$ et $\rho_K$ le diamètre de la plus
      grande boule      inscrite  dans $K$ et  $\omega_K = \frac{h_K}{\rho_K}$
    \end{itemize}
    Alors il existe une constante $c$ independente de $K$ telle que $\forall v \in
    H^{k+1}(K)$ et pour tout $m \in \{0,...,k+1\}$,
    \begin{equation}
      \label{eq:32}
      |v - \Ilag{K}(v)|_{m,K} \leq  c h^{k+1-m} \omega_K^m |v|_{k+1,K}
    \end{equation}
  \end{theorem}
  \begin{remark}
    \label{rem:20}
    \begin{itemize}
    \item $\omega_K$ devrait être aussi proche de $1$ que possible
    \item La deuxième hypothèse technique permettant d'assurer que $H^{k+1}(K) \subset
      C^0(K)$.
    \item on obtient des résultats similaires si $v$ n'est pas suffisamment régulière
    \end{itemize}

  \end{remark}

  \begin{definition}[Espaces $H^1$ conformes]
    \label{def:35}
    Un espace vectoriel $V_h$ de fonctions définies sur un domaine $\Omega_h$
    est dit être $H^1$-conforme si $V_h \subset H^1(\Omega_h)$
  \end{definition}
  Afin de construire un tel espace on introduit tout d'abord
    \begin{equation}
      \label{eq:52}
      W_h = \{v_h \in L_2(\Omega_h); \forall K \in \mathcal{T}_h, v_h|_K \in P_K\}
    \end{equation}
    mais ce n'est pas suffisant: \emph{les fonctions $W_h$ peuvent avoir des
    sauts entre les éléments du maillage}. Nous avons donc besoin d'assurer la
    \emph{continuité} de ces fonctions
    \begin{equation}
      \label{eq:53}
      V_h = W_h \cap C^0(\Omega_h) = \{ v_h \in W_h; \forall F \in
      \mathcal{F}^i_h. \jump{v_h}_F = 0\}
    \end{equation}
    Concernant l'implémentation, nous avons besoin de d'indentifier les
    \emph{degrés de liberté communs entre les éléments} quand nous construisons
    la tables des degrés de liberté.

    Voici deux exemples d'espace $H^1$-conforme
    \begin{equation}
      \label{eq:54}
      P^k_{c,h} = \{ v_h \in C^0(\Omega_h); \forall K \in \mathcal{T}_h, v_h
      \circ T_K \in \mathbb{P}_k\}
    \end{equation}
    \begin{equation}
      \label{eq:55}
      Q^k_{c,h} = \{ v_h \in C^0(\Omega_h); \forall K \in \mathcal{T}_h, v_h
      \circ T_K \in \mathbb{Q}_k\}
    \end{equation}

    \subsection{Projections orthogonales}
    \label{sec:proj-orth}

    \begin{definition}{Projection Orthogonale}
      \label{def:36}
      On note
      \begin{eqnarray}
        \label{eq:56}
        \Pi^{0,k}_{c,h} : L_2(\Omega) \rightarrow P^k_{c,h}\\
        \Pi^{1,k}_{c,h} : H^1(\Omega) \rightarrow P^k_{c,h}
      \end{eqnarray}
      associés respectivement aux produits scalaires
      $(u,v)_{0,\Omega} = \int_\Omega u v$ and $(u,v)_{1,\Omega} = \int_\Omega u
      v + \int_\Omega \nabla u \cdot \nabla v$
      On a pour $l=1,...,k$ et si $v \in H^{l+1}(\Omega)$
      \begin{eqnarray}
        \label{eq:57}
        \|v - \Pi^{0,k}_{c,h}(v)\|_{0,\Omega} &\leq c h^{l+1} |v|_{k+1,\Omega} \\
        \|v - \Pi^{1,k}_{c,h}(v)\|_{1,\Omega} &\leq c h^{l} |v|_{k+1,\Omega}
      \end{eqnarray}
    \end{definition}

    Pour calculer  $\Pi^{0,k}_{c,h}$ et $\Pi^{1,k}_{c,h}$ on a besoin de
    résoudre les     problèmes
    \begin{problem}[Projection $L_2$]
      \label{prob:3}
      Soit $v$ une fonction de $L_2$, calculer $\Pi^{0,k}_{c,h}(v) \in
      P^k_{c,h}$ tel que $\forall v_h \in P^k_{c,h}$ on a
      \begin{equation}
        \label{eq:58}
        (\Pi^{0,k}_{c,h}(v), v_h)_{0,\Omega} = (v, v_h)_{0,\Omega}
      \end{equation}
\end{problem}

\begin{problem}[Projection $H^1$]
  \label{prob:4}
  Soit $v$ une fonction de $H^1$, calculer $\Pi^{1,k}_{c,h}(v) \in
  P^k_{c,h}$ tel que $\forall v_h \in P^k_{c,h}$ on a
  \begin{equation}
    \label{eq:59}
    (\Pi^{1,k}_{c,h}(v), v_h)_{1,\Omega} = (v, v_h)_{1,\Omega}
  \end{equation}
\end{problem}

\subsubsection{Interpolant de Lagrange sur un maillage}
\label{sec:lagr-interp-mesh}

Notons $V_h$ un espace $H^1$ conforme,
$\{\Psi_i\}_{1,...,N}$ une base nodale de $V_h$ et  $\{x_1,...,x_N\}$ les noeuds associés
alors
\begin{definition}
  \label{def:37}
  L'interpolant de Lagrange est défini par
  \begin{equation}
    \label{eq:60}
    \Ilag{h}: v \in C^0(\Omega_h) \mapsto \sum_{i=1}^N v(x_{i}) \Psi_i \in V_h
  \end{equation}
\end{definition}
\begin{remark}
  \label{rem:21}
  Noter que
  \begin{equation}
    \label{eq:61}
    \Ilag{h}(v)|_K = \Ilag{h}(v|_K)
  \end{equation}
  La restriction de l'interpolant de Lagrange à une cellule $K$ coincide avec
  l'interpolant de Lagrange appliqué à la fonction dans la cellule $K$.
\end{remark}



\begin{theorem}[Propriété de l'interpolant de Lagrange]
  \label{thr:15}
  \begin{itemize}
  \item $\{\mathcal{T}_h\}_{h>0}$ une famille de maillage quasi-uniforme et conformes
  \item $\mathbb{P}_k \subset \hat{P}$ and $k+1 > \frac{d}{2}$
  \end{itemize}
  Alors il existe une constante $c$ telle que pour tout $h$ et $v \in
  H^{k+1}(\Omega_h)$
  \begin{equation}
    \label{eq:62}
    \|v - \Ilag{h}(v)\|_{0,\Omega_h} + h |v - \Ilag{h}(v)|_{1,\Omega_h} \leq
    c h^{k+1} |v|_{k+1,\Omega_h}
  \end{equation}
\end{theorem}

\begin{example}
  Nous allons vérifier sur un exemple ce théorème. Nous considérons pour cela
  $\alpha$ un réel et $\mathbf{x}=(x_1,...,x_d) \in \mathbb{R}^d$ un point de $\Omega
  = [0,1]^d, d=1,2,3$ et $v$ la fonction définie par
  \begin{equation}
    \label{eq:97}
    \begin{split}
    v : \Omega &\rightarrow \mathbb{R}\\
    \mathbf{x} &\rightarrow ( \mathbf{x} \cdot \mathbf{x} )^{\alpha/2}\ \Pi_{i=1}^d(
    1-x_i^2)
    \end{split}
  \end{equation}
  Nous construisons l'interpolant de Lagrange de $v$ dans $P^k_{c,h}$ avec
  $k=1,...,5$ et $d=1,2,3$ et étudions l'erreur d'interpolation $L_2$ et $H^1$
  du  théorème~(\ref{eq:62})  en échelle log-log. Nous devons obtenir des
  droites de pentes $k$ (resp. $k+1$) pour la norme $L_2$ (resp. $H^1$.)

  \lstinputlisting[linerange=13-25,caption={Propriété de l'interpolant de Lagrange}]{../Codes/prudhomme/lagrange/error.cpp}
\end{example}

\subsection{Interpolation Iso-parametrique sur des domaines courbes}
\label{sec:interp-iso-param}

Quand le domaine est courbe, si nous désirons obtenir des propriétés de convergence optimale
nous avons besoin de discretiser le bord du domaine avec suffisamment  de précision

Notons $(\hat{K},\hat{P}_{\mathrm{geo}}, \hat{\Sigma}_{\geo})$ l'élément fini géométrique
et $(\hat{K},\hat{P}, \hat{\Sigma})$ l'élément fini de référence
pour $V_h$
\begin{definition}
  \label{def:38}
  \begin{itemize}
  \item si $\hat{P}_{\mathrm{geo}} = \hat{P}$ l'approximation est dite
    \emph{iso-parametrique}
  \item si $\hat{P}_{\mathrm{geo}} \subset \hat{P}$ l'approximation est dite
    \emph{sub-parametrique}
  \item si $\hat{P} \subset\hat{P}_{\mathrm{geo}}$ l'approximation est dite \emph{sur-parametrique}
  \end{itemize}
\end{definition}
\begin{remark}
  \label{rem:22}
  Gmsh un mailleur libre permet de générer des maillage d'ordre élevé jusqu'à
  l'ordre  5 en  2D et 4 en 3D
\end{remark}





\begin{theorem}[Propriétés de l'interpolation  iso-paramétrique]
  \label{thr:16}
  Supposons que
  \begin{itemize}
  \item $\{\mathcal{T}_h\}_{h>0}$ une famille de maillage quasi-uniformes et conformes
  \item $\mathbb{P}_k \subset \hat{P}$ et $k+1 > \frac{d}{2}$
  \item $k_{\mathrm{geo}} = k$
  \end{itemize}
  Alors il existe une constante $c$ telle que  pour tout $h$ et $v \in
  H^{k+1}(\Omega_h)$
  \begin{equation}
    \label{eq:63}
    \|v - \Ilag{h}(v)\|_{0,\Omega_h} + h |v - \Ilag{h}(v)|_{1,\Omega_h} \leq
    c h^{k+1} |v|_{k+1,\Omega_h}
  \end{equation}
\end{theorem}
\begin{remark}
  \label{rem:23}
  Les résultats sont identiques à ceux du theorème~\ref{thr:15}.
\end{remark}

\begin{example}
  Nous allons vérifier sur un exemple à l'erreur d'approximation de l'élément
  géométrique. Considérons les cercles unité généré par une transformation
  affine, noté $\Omega^1_h$, et d'ordre 2, noté $\Omega^2_h$, et calculons leur
  aire respective.  Construisons une famille de maillage $\{\calTh\}_{h >0}$,
  par exemple $h\in \{0.4, 0.2, 0.1, 0.05\}$ et calculons l'erreur entre le
  calcul exact de l'aire $\pi$ et le calcul numérique $\int_{\Omega^1_h} 1$ et
  $\int_{\Omega^2_h} 1$ respectivement. Le listing suivant présente le code C++
  pour effectuer cela

  \lstinputlisting[linerange=13-25,caption={Calcul de l'aire d'un cercle}]{../Codes/prudhomme/isoparam/circle.cpp}

  La table~\ref{tab:1} présente les erreurs d'approximation et la
  figure~\ref{fig:circle} présente les courbes de convergence en échelle log-log
  ainsi que les pentes associées à ces courbes. On s'attend d'après le
  théorème~(\ref{eq:62}) appliqué à des pentes à l'élément fini géométrique. On
  observe un phénomène de super-convergence pour le cas $\Omega^2_h$, on obtient
  un ordre de convergence $4$ et nous devrions obtenir $3$.
  \begin{table}[h]
    \centering
    \pgfplotstableread{../Codes/prudhomme/isoparam/circle.dat}\loadedtable
    \pgfplotstabletypeset[columns={h,error1,error2},
    columns/{h}/.style={
    column type=r,fixed, fixed zerofill,precision=3
    },
    columns/{error1}/.style={
    column name=$|\pi-\int_{\Omega^1_h} 1|$,
    sci,sci zerofill,
    precision=2},
    columns/{error2}/.style={
    column name=$|\pi-\int_{\Omega^2_h} 1|$,
    sci,sci zerofill,
    precision=2},
    every head row/.style={before row=\toprule,after row=\midrule},
    every last row/.style={after row=\bottomrule}
    ]\loadedtable
    \caption{Erreur de convergence}
    \label{tab:1}
  \end{table}
  \begin{figure}[h]
    \centering
    \begin{tikzpicture}[scale=0.70]
      \begin{loglogaxis}[%x=3cm,
        xlabel=h,ylabel=$|\pi-\int_{\Omega_h} 1|$,
        % title={ error curves },
        legend style={at={(0,1)}, anchor=north west}]
        \addplot table[x=h,y={create col/linear regression={y=error1}}]{../Codes/prudhomme/isoparam/circle.dat};
        \xdef\slopea{\pgfplotstableregressiona}
        \addlegendentry{$\mathbb{P}_1$ pente = $\pgfmathprintnumber{2.17}$}
        \addplot table[x=h,y={create col/linear regression={y=error2}}]{../Codes/prudhomme/isoparam/circle.dat};
        \xdef\slopeb{\pgfplotstableregressiona}
        \addlegendentry{$\mathbb{P}_2$ pente = $\pgfmathprintnumber{4.36}$}
      \end{loglogaxis}
    \end{tikzpicture}
    \caption{Convergence des approximations $\int_{\Omega^1_h} 1$ et $\int_{\Omega^2_h} 1$ vers $\pi$}
    \label{fig:circle}
  \end{figure}

\end{example}


\subsection{\Feel}
\label{sec:feel}

Dans \Feel l'ordre polynomial de la transformation géométrique est donné par
le second argument template
\begin{lstlisting}
Mesh<Simplex<$d$, $k_{\mathrm{geo}}$> >
Mesh<Hypercube<$d$, $k_{\mathrm{geo}}$> >
\end{lstlisting}
\begin{itemize}
\item $d$ est la dimension
\item $k_{\mathrm{geo}}$ est l'ordre polynomial de la transformation géométrique.
\end{itemize}



Un maillage est décomposé en un ensemble
\begin{itemize}
\item d'éléments  décomposés  en sous entités  (volume,face,arête,point),
\item faces(decomposés en sous entités),
\item arêtes(decomposés en sous entités) et
\item points.
\end{itemize}
et à chaque élément $K$ est associé une transformation géométrique $T_K$.

À fin de parcourir les éléments et faces du maillage, \Feel fournit des
fonctions renvoyant des \emph{itérateurs} (début et fin) sur ces ensembles
\begin{itemize}
\item \lstinline!elements(mesh)! retourne 2 itérateurs sur l'ensemble des
  éléments du maillage
\item \lstinline!markedelements(mesh,<int>)!  et
  \lstinline!markedelements(mesh,<string>)! retourne 2 itérateurs sur les
  éléments marqués par  l'entier \lstinline!<int>! et la chaîne des caractères
  \lstinline!<string>! respectivement, ca correspondra typiquement à des
  propriétés de matériau
\item \lstinline!boundaryfaces(mesh)! retourne 2 itérateurs sur les faces au
  bord du maillage
\item \lstinline!markedfaces(mesh,<int>)! et
  \lstinline!markedelements(mesh,<string>)! retourne 2 itérateurs sur les
  faces marquées par  l'entier \lstinline!<int>! et la chaîne des caractères
  \lstinline!<string>! respectivement, ca correspondra typiquement aux
  conditions aux limites
\end{itemize}


L'espace d'approximation $V_h$  $H^1$ conforme (espaces de functions continues
sur $\Omega$ polynomiales par morceaux de degré $\leq k$) est défini comme suit
\begin{lstlisting}
FunctionSpace<Mesh<Simplex<$d$, $k_{\mathrm{geo}}$> >,
              bases<Lagrange<$k$> > > $V_h$;
FunctionSpace<Mesh<Hypercube<$d$, $k_{\mathrm{geo}}$> >,
              bases<Lagrange<$k$> > > $V_h$;
\end{lstlisting}




%
%
\section{Du problème global aux éléments locaux}
\label{sec:glob}
%
%
\noindent
On va maintenant faire le lien entre la résolution d'un problème par
méthode d'éléments finis et les notions qui viennent d'être
introduites.
%
Soit une EDP à résoudre sur un domaine $\Omega$, et $V$ l'espace de
Hilbert dans lequel on cherche une solution de la formulation variationnelle
du problème. On réalise un maillage de $\Omega$ par une famille affine de
$N_e$ éléments finis $(K_i,\Sigma_i,P_i)_{i=1,\ldots,N_e}$.
%
Par unisolvance, la solution approchée $u_h$ sera entièrement définie
sur chaque élément $(K_i,\Sigma_i,P_i)$ par ses valeurs sur les points de
$\Sigma_i$, qu'on appellera les {\bf noeuds du maillage}. Il est à noter
qu'un noeud sera en général commun à plusieurs éléments
adjacents. Le nombre total de noeuds $N_h$ est donc inférieur à
$N_e\times\hbox{Card} \Sigma_i$, et on a dim $V_h = N_h$. Notons
$a_1,\ldots,a_{N_h}$ les noeuds du maillage. Le problème approché se
ramène donc à la détermination des valeurs de $u_h$ aux points $a_i$: ce
sont les degrés de liberté du problème approché.
%
On va construire une base de $V_h$ en associant à chaque ddl $a_i$ un vecteur de la base. On définit ainsi les {\bf fonctions de base globales} $\varphi_i$ ($i=1,\ldots,N_h$) par

\begin{equation}
\varphi_i \,_{|K_j} \in P_j, \quad j=1,\ldots,N_e \quad\hbox{ et }\quad \varphi_i(a_j)=\delta_{ij}, \; 1\le i,j \le N_h\label{eq:28}
\end{equation}


%
L'espace d'approximation interne est donc alors :

\begin{equation}
V_h = \hbox{Vect }\left\{\varphi_1,\ldots,\varphi_{N_h}\right\}\label{eq:29}
\end{equation}


%
%
Il est facile de remarquer qu'une telle fonction $\varphi_i$ est nulle
partout, sauf sur les éléments dont $a_i$ est un noeud. En effet, si $a_i$
n'appartient pas à un élément $K$, $\varphi_i$ est nulle sur tous les
noeuds de $K$, et donc sur $K$ tout entier par unisolvance.

De plus, sur un élément $K$ dont $a_i$ est un noeud, $\varphi_i$ vaut 1
sur $a_i$ et 0 sur les autres noeuds de $K$. Donc $\varphi_i\, _{|K}$ est une
fonction de base locale de $K$. On voit donc que {\bf la fonction de base
globale $\varphi_i$ est construite comme réunion des fonctions de base
locales sur les éléments du maillage dont $a_i$ est un noeud.}
%
\begin{figure}[h]
\begin{center}
\includegraphics[width=0.75\linewidth]{FIG/fonction-globale.jpg}
%\vspace*{8 cm}
\caption{Exemple de fonction de base globale (élément triangulaire $P_1$)}
\label{fig:fnglob}
\end{center}
\end{figure}\\
%
C'est à ce niveau que se situe le lien entre les définitions locales
introduites au \S \ref{sec:lagrange} et le problème global approché à
résoudre. Par ailleurs, ceci implique que tous les calculs à effectuer sur
les fonctions de base globales peuvent se ramener à des calculs sur les
fonctions de base locales, et donc simplement à des calculs sur
l'élément de référence (car on a maillé le domaine avec une famille
d'éléments finis affine-équivalents).
%
%
\newpage
%
\noindent
\begin{remark}
  Ce type de définition des fonctions de base n'est possible que si le
  maillage est {\bf conforme}, c'est à dire si l'intersection entre deux
  éléments est soit vide, soit réduite à un sommet ou une arête en
  dimension 2 (ou à un sommet, une arête ou une face en dimension 3). On
  interdit ainsi les situations du type de celle de la figure
  \ref{fig:nonconf}.\label{rem:11}
\end{remark}

%
%
\begin{figure}[h]
\begin{center}
\includegraphics[width=0.75\linewidth]{FIG/conforme.jpg}
%\vspace*{6 cm}
\caption{Exemples de maillage non conforme}
\label{fig:nonconf}
\end{center}
\end{figure}
%
%
\section{Exercices}
%
\begin{enumerate}
\item Calculer les fonctions de base locales des éléments finis de Lagrange introduits dans ce chapitre.
\item Donner l'allure des fonctions de base globales correspondantes. Sont-elles continues ? dérivables ?
\item Pour les éléments finis de Lagrange introduits dans ce chapitre, écrire le changement de variable affine entre élément quelconque et élément de référence.
\end{enumerate}


%%% Local Variables:
%%% coding: utf-8
%%% mode: latex
%%% TeX-PDF-mode: t
%%% TeX-parse-self: t
%%% TeX-auto-save: t
%%% x-symbol-8bits: nil
%%% TeX-auto-regexp-list: TeX-auto-full-regexp-list
%%% TeX-master: "mef-intro"
%%% ispell-local-dictionary: "american"
%%% End:

\chapter{Approximation de problèmes coercifs}
\label{cha:appr-de-probl}

Dans ce chapitre, on s'intéresse à l'approximation de problèmes coercifs:
\begin{itemize}
\item le Laplacien et des variantes sur les conditions aux limites et le
  l'équation elle-même;
\item l'élasticité linéaire qui permet de modéliser de petites déformations
  d'un milieu continu déformable.
\end{itemize}

\section{Le Laplacien}
\label{sec:le-laplacien}

on s'instéresse dans cette section à l'\emph{approximation élément fini
conforme} du problème suivant:
\begin{problem}
  \label{prob:1}
  On cherche $u$ telle que
  \begin{equation}
    \label{eq:64}
    \begin{split}
      -\Delta u &= f \mbox{ dans } \Omega\\
      u &= 0 \mbox{ sur } \partial \Omega
    \end{split}
  \end{equation}
\end{problem}


\subsection{Cadre Mathematique}
\label{sec:cadre-mathematique}

On suppose que $f \in L^2(\Omega)$. La formulation faible du
problème~\ref{prob:1} est la suivante:
\begin{problem}
  \label{prob:2}
  On cherche $u \in H^1_0(\Omega)$ telle que
  \begin{equation}
    \label{eq:65}
    \ds\int_\Omega \nabla u \cdot \nabla v = \ds \int_\Omega f\, v,\quad \forall v \in H^1_0(\Omega)
  \end{equation}
\end{problem}

\subsubsection{Problème bien posé}
\label{sec:probleme-bien-pose}

Introduisons
\begin{itemize}
\item $V = H^1_0(\Omega)$ doté de la norme $\|\cdot\|_{1,\Omega}$ telle que
  $\|v\|_{1,\Omega} = (\|v\|^2_{0,\Omega} + \|\nabla v\|^2_{0,\Omega})^{1/2}$
\item la forme bilinéaire $a \in \mathcal{L}(V \times V, \R)$ telle que $a(u,v) = \int_\Omega
  \nabla u \cdot \nabla v $
\item la forme linéaire $\ell \in \mathcal{L}(V, \R)$ telle que $l(v) = \int_\Omega
  f \nabla v $
\end{itemize}

Le problème~\ref{prob:2} s'écrit sous forme abstraite
\begin{problem}
  \label{prob:5}
  On cherche $u \in V$ telle que
  \begin{equation}
    \label{eq:65}
    a(u, v) = \ell(v), \quad \forall v \in V
  \end{equation}
\end{problem}

L'espace $V$ est un espace de Hilbert et les formes $a$ et $\ell$ sont
continues sur $V\times V$ et $V$ respectivement. Il ne reste plus qu'à
vérifier si le problème est bien posé (existence d'une solution unique). Pour
cela on utilise démontre la \emph{coercivité} de la forme bilinéaire $a$ sur
$V \equiv H^1_0(\Omega)$. Ceci se fait grâce au lemme suivant:
\begin{lemma}[Inégalité de Poincaré]
  \label{lem:1}
  Soit $\Omega$ un ouvert borné de $\R$. Il existe une constante $c_\Omega$
  (dépendente de $\Omega$ telle que
  \begin{equation}
    \label{eq:66}
    \forall v \in H^1_0(\Omega),\quad \|v\|_{0,\Omega} \le c_\Omega \|\nabla v\|_{0,\Omega}
  \end{equation}
\end{lemma}
\begin{remark}
  \label{rem:24}
  $c_\Omega$ est homogène à une longeur et peut être interprétée comme une
  mesure caractéristique de $\Omega$
\end{remark}

Grâce à l'inégalité de Poincaré, on a le résultat suivant
\begin{proposition}
  \label{prop:7}
  La forme bilinéaire $a$ du problème~\ref{prob:2} est \emph{coercive}

  \begin{proof}
    On note tout d'abord que par l'inégalité de Poincaré et la définition de
    $\|\cdot\|_{1,\Omega}$
    \begin{equation}
      \label{eq:68}
      \|v\|^2_{1,\Omega} \le (1 + c^2_\Omega) \|\nabla v\|^2_{0,\Omega}
    \end{equation}
    On en déduit que
    \begin{equation}
      \label{eq:67}
      \forall v \in H^1_0(\Omega),\quad a(v,v) = \|\nabla v\|^2_{0,\Omega} \ge
      \ds \frac{1}{1+c^2_\Omega} \|v\|^2_{1,\Omega}
    \end{equation}
    Le Lemme de Lax-Milgram~\ref{thr:12} permet alors de conclure sur l'existence d'une
    solution unique pour le problème~\ref{prob:2}.
  \end{proof}
\end{proposition}

\subsubsection{Approximation conforme}
\label{sec:appr-conf}

On utilise une approximation conforme par éléments finis de Lagrange. On
considère $\Omega$ un polygone ou polyhèdre régulier de $\R[2]$ ou $\R[3]$
respectivement et un maillage $\calTh = \{K_e\}_{e=1...\Ne}$ de $\Omega$. On considère un élément
fini de référence $(\hat{K},\hat{P},\hat{\Sigma})$ tel que $\Pk \subset \hat{P}$
et $k+1 > \frac{d}{2}$, voir le théorème~\ref{thr:16}. On note
\begin{equation}
  \label{eq:70}
  L^k_{c,h} = \{ v_h \in C^0(\bar{\Omega}); \forall K \in \mathcal{T}_h,\ v_h
  \circ T_K \in \hat{P}\}
\end{equation}
où $T_K$ est la tranformation géométrique de $\hat{K}$ dans $K$.
\begin{itemize}
\item Si on utilise $\hat{P}=\Pk$ on a $L^k_{c,h} = P^k_{c,h}$.
\item Si on utilise $\hat{P}=\Qk$ on a $L^k_{c,h} = Q^k_{c,h}$.
\end{itemize}
Afin de construire un espace d'approximation conforme \ie $V_h \subset V
=H^1_0(\Omega)$ on prend
\begin{equation}
  \label{eq:71}
  V_h = L^k_{c,h} \cap H^1_0(\Omega)
\end{equation}
c'est à dire que les fonctions de $V_h$ satisfont les conditions aux limites
outre le fait d'être dans $L^k_{c,h}$. Le problème discret s'écrit alors
\begin{problem}
  \label{prob:6}
  Trouver $u_h \in V_h$ telle que
  \begin{equation}
    \label{eq:72}
    a(u_h,v_h) = \ell(v_h),\, \forall v_h \in V_h
  \end{equation}
\end{problem}
qui est bien posé (existence et unicité de $u_h$) car $a$ est coercive sur $V$
et que $V_h \subset V$. On a le résultat suivant:
\begin{theorem}[Convergence de $u_h$]
  \label{thr:17}
  On suppose que $u$ solution de \ref{prob:2} est dans $H^{k+1}(\Omega) \cap
  H^1_0(\Omega)$, alors il existe une constante $c_1$ telle que pour tout $h$
  \begin{equation}
    \label{eq:73}
    \|u-u_v\|_{1,\Omega} \le c_1 h^k |u|_{k+1,\Omega}
  \end{equation}
  et il existe une constante $c_2$ telle que pour tout $h$
  \begin{equation}
    \label{eq:74}
    \|u-u_v\|_{0,\Omega} \le c_2 h^{k+1} |u|_{k+1,\Omega}
  \end{equation}
\begin{proof}
  La preuve de (\ref{eq:73}) est obtenue grâce au lemme de Cea~(\ref{eq:cea})
  et du théorème d'interpolation~(\ref{eq:62}).
  La preuve de (\ref{eq:74}) est obtenue grâce au lemme d'Aubin-Nitsche qui
  permet d'affirmer qu'il existe une constante $c$ telle que
  \begin{equation}
    \label{eq:75}
    \|u-u_h\|_{0,\Omega} \leq c h |u-u_h|_{1,\Omega}
  \end{equation}
  et donc que (\ref{eq:74}) se déduit de (\ref{eq:74}).
\end{proof}
\end{theorem}

\subsubsection{Implémentation avec \Feel}
\label{sec:impl-en-feel++}

Avec \Feel, on ne construit pas explicitement l'espace $V_h$ mais
$L^k_{c,h}$. Le traitement des conditions aux limites de Dirichlet du problème
(\ref{eq:64}) peut être effectué de diverses fa\c {c}ons, nous en verrons une.

Commencons par le maillage, dans un premier temps nous définissons le type du
maillage contenant soit des éléments de type simplexe~(segment, triangle,
tetrahèdre) ou de type hypercube~(segment, quadrangle, hexahèdre).

\begin{lstlisting}[caption={Maillage $\calTh$}]
  // un maillage de simplexe dans $\R$ telle que la transformation
  // géométrique $T_K,\ K \in \calTh$  soit affine
  typedef Mesh<Simplex<d,1> > mesh_type;

  // un maillage d'hypercube dans $\R$ telle que la transformation
  // géométrique $T_K,\ K \in \calTh$  soit affine en chacune des variables
  // typedef Mesh<Hypercube<d,1> > mesh\_type;

  // generate the mesh associated to the unit square $[0,1]^2$ using triangles
  auto mesh = unitSquare();
\end{lstlisting}
\begin{remark}
  \label{rem:25}
  Le mot clé \texttt{auto} permet de faire de l'inférence de type, pour plus
  de détails consultez
  \href{http://fr.wikipedia.org/wiki/C%2B%2B11#Inf.C3.A9rence_de_types}{la page C++11 de Wikipedia}.
\end{remark}
Ensuite nous pouvons définir l'espace $L^k_{c,h}$,
\begin{lstlisting}[caption={$L^k_{c,h}$}]
  // Vh est une structure de donnée allouée dynamiquement
  auto Vh = Pch<1>( mesh );
  // u est un élément de Vh
  auto u = Vh->element();
  // u est un autre élément de Vh
  auto u = Vh->element();
\end{lstlisting}

À présent, nous définissons les formes bilinéaires $a$ et $\ell$ qui sont
respectivement des formes bilinéaires et linéaires.

\begin{lstlisting}[caption={Définitions de $a$ et $\ell$}]
  // $a \in \mathcal{L}(V_h \times V_h,\ \R[\phantom{1}])$
  auto a = form2( _test=Vh, _trial=Vh );
  // $a = \sum_{e=1...\Ne} \int_{K_e} \nabla \varphi_j \cdot \nabla \varphi_i,\quad  i,j=1...,\dim{V_h}$
  a = integrate( _range=elements(mesh),
                 _expr=gradt(u)*trans(grad(v)) );

  // $\ell \in \mathcal{L}(V_h,\ \R[\phantom{1}])$
  auto l = form1( _test=Vh );
  // $\ell = \sum_{e=1...\Ne} \int_{K_e} f  \varphi_i,\quad  i=1...,\dim{V_h}$
  l = integrate( _range=elements(mesh), _expr=f*id(v) );
\end{lstlisting}

Afin de traiter les conditions aux limites de Dirichlet homogènes, on peut
utiliser le mot-clé \texttt{on} qui permet de les imposer de manière forte.
\begin{lstlisting}[caption={Traitement des conditions de Dirichlet avec \texttt{on}}]
  a += on(_range=boundaryfaces(mesh), _element=u, _rhs=l, _expr=constant(0.) );
\end{lstlisting}
\begin{remark}
  \label{rem:26}
  Le mot-clé \texttt{constant} permet de transformer une type numérique (\eg
  entier, flottant) en une expression utilisable par le langage de
  \Feel. Notez également l'opération \texttt{+=} qui permet de rajouter le
  traitement des conditions de Dirichlet tout en gardant les contributions
  précédentes. L'opération \texttt{=} aurait d'abord remis à $0$ les entrées
  de la matrice associée à $a$.
\end{remark}

Enfin nous pouvons résoudre le problème~\ref{prob:6}

\begin{lstlisting}[caption={Résolution du problème discret~\ref{prob:6}}]
  a.solve( _rhs=l, _solution=u );
\end{lstlisting}

Le listing complet
\lstinputlisting[linerange=12-25,caption={Exemple Laplacien avec conditions de
Dirichlet homogènes}]{../Codes/prudhomme/laplacian/dirichlet_homogene.cpp}


\subsection{Conditions aux limites}
\label{sec:cond-aux-limit}

\subsubsection{Conditions aux limites de Dirichlet non homogène}
\label{sec:cond-aux-limit-1}

On suppose toujours $f \in L^2(\Omega)$ et on se donne une fonction $g \in
C^{0,1}(\partial \Omega)$\footnote{$g$ est Lipschitzienne sur $\partial
\Omega$}. On s'intéresse au problème suivant:
\begin{problem}
  \label{prob:7}
  On cherche $u : \Omega \rightarrow \R[]$ telle que
  \begin{equation}
    \label{eq:76}
    \begin{split}
    -\Delta u &= f \mbox{ dans } \Omega\\
    u &= g \mbox{ sur } \partial \Omega
    \end{split}
  \end{equation}
\end{problem}

\begin{remark}
  \label{rem:27}
  L'hypothèse $g \in C^{0,1}(\partial \Omega)$ permet d'affirmer qu'il existe
  $u_g \in H^1(\Omega)$ telle que $u_{g_{|\partial \Omega}} = g$.
\end{remark}

On se ramène au problème avec conditions de Dirichlet homogène en faisant le
change d'inconnue $u_0=u-u_g$ et on s'intéresse au problème suivant:
\begin{problem}
  \label{prob:8}
  On cherche $u_0 \in H^1_0(\Omega)$ telle que
  \begin{equation}
    \label{eq:77}
    a(u_0,v) = \ell(v) - a(u_g,v),\quad \forall v \in H^1_0(\Omega)
  \end{equation}
\end{problem}
Ce problème est bien posé d'après Lax-Milgram, voir section précédente.

\begin{theorem}
  \label{thr:18}
  On suppose que $u$ solution de \ref{prob:8} est dans $H^{k+1}(\Omega) \cap
  H^1_0(\Omega)$, alors il existe une constante $c_1$ telle que pour tout $h$
  \begin{equation}
    \label{eq:73}
    \|u-u_v\|_{1,\Omega} \le c_1 h^k |u|_{k+1,\Omega}
  \end{equation}
  et il existe une constante $c_2$ telle que pour tout $h$
  \begin{equation}
    \label{eq:74}
    \|u-u_v\|_{0,\Omega} \le c_2 h^{k+1} |u|_{k+1,\Omega}
  \end{equation}
\end{theorem}

Avec \Feel, les conditions Dirichlet non-homogènes sont traitées par exemple
avec le mot-clé \texttt{on}.
\begin{lstlisting}
  // définition de la fonction, p.ex $g=sin(2 \pi x)$
  auto g = sin(2*pi*Px() );
  // définition de $a$
  ...
  // ajout des conditions de Dirichlet non-homogène
  a += on( _range=boundaryfaces(mesh), _expr=g );
\end{lstlisting}
\begin{remark}
  \label{rem:28}
  Il n'y a pas besoin de rajouter le terme $a(u_g,v)$ au second membre $\ell(v)$, il est
  pris en compte automatiquement par \texttt{on}.
\end{remark}
Voici le listing complet de l'exemple du laplacien avec conditions de Dirichlet non-homogène
\lstinputlisting[linerange=12-26,caption={Exemple Laplacien avec conditions de
Dirichlet non-homogènes}]{../Codes/prudhomme/laplacian/dirichlet_nonhomogene.cpp}

\subsubsection{Condition aux limites de Neumann}
\label{sec:cond-aux-limit-2}

Étant donnés un réel $\mu$ strictement positif, $f \in L^2(\Omega)$ et $g \in
L^2(\partial \Omega)$, on s'intéresse au problème suivant:
\begin{problem}
  \label{prob:9}
  On cherche $u : \Omega \rightarrow \R[]$ telle que
  \begin{equation}
    \label{eq:78}
    \begin{split}
      -\Delta u + \mu u &= f, \mbox{ dans } \Omega\\
      \partial_\Next u &= g, \mbox{ sur } \partial\Omega
    \end{split}
  \end{equation}
\end{problem}
où $\partial_\Next u = \nabla u \cdot \Next = \sum_{i=1}^d n_i \partial_i u$
dénote la dérivée normale de $u$ avec $\Next=(n_1,...,n_d) \in \R[d]$ la
normale extérieure unitaire en un point du bord de $\Omega$.
La formulation faible s'écrit
\begin{problem}
  \label{prob:13}
  On cherche $u \in H^1(\Omega)$ telle que
  \begin{equation}
    \label{eq:79}
    a( u, v ) = \ell(v),\ \forall v \in H^1(\Omega)
  \end{equation}
  avec
  \begin{equation}
    \label{eq:80}
    a( u, v ) = \ds \int_\Omega \nabla u \cdot \nabla v + \mu u v
  \end{equation}
  et
  \begin{equation}
    \label{eq:81}
    \ell( v ) = \ds \int_\Omega f v + \int_{\partial\Omega} g v
  \end{equation}
\end{problem}

On a
\begin{equation}
  \label{eq:82}
  a(v, v) = \ds \int_\Omega \nabla v \cdot \nabla v + \mu v v \ge \min(1,\mu)
  \int_\Omega \nabla v \cdot \nabla v +  v v  = \min(1,\mu) \|v\|_{1,\Omega}, \quad
  \forall v \in H^1(\Omega)
\end{equation}
ce qui nous permet d'affirmer que $a$ est coercive sur $H^1(\Omega)$ et que le
problème~\ref{prob:13} est bien posé grâce à Lax-Milgram.

\lstinputlisting[linerange=12-25,caption={Exemple Laplacien avec conditions de
Neumann}]{../Codes/prudhomme/laplacian/neumann.cpp}

\subsubsection{Conditions aux limites de Robin}
\label{sec:cond-aux-limit-3}

Étant donnés un réel $\mu$ strictement positif, $f \in L^2(\Omega)$ et $g \in
L^2(\partial \Omega)$, on s'intéresse au problème suivant:
\begin{problem}
  \label{prob:14}
  On cherche $u : \Omega \rightarrow \R[]$ telle que
  \begin{equation}
    \label{eq:83}
    \begin{split}
      -\Delta u  &= f, \mbox{ dans } \Omega\\
      \mu u + \partial_\Next u &= g, \mbox{ sur } \partial\Omega.
    \end{split}
  \end{equation}
\end{problem}
La formulation faible s'écrit
\begin{problem}
  \label{prob:15}
  On cherche $u \in H^1(\Omega)$ telle que
  \begin{equation}
    \label{eq:84}
    a( u, v ) = \ell(v),\ \forall v \in H^1(\Omega)
  \end{equation}
  avec
  \begin{equation}
    \label{eq:85}
    a( u, v ) = \ds \int_\Omega \nabla u \cdot \nabla v + \int_{\partial
    \Omega} \mu u v
  \end{equation}
  et
  \begin{equation}
    \label{eq:86}
    \ell( v ) = \ds \int_\Omega f v + \int_{\partial\Omega} g v
  \end{equation}
\end{problem}

On a
\begin{equation}
  \label{eq:69}
  \begin{split}
    a(v, v) & = \ds \int_\Omega \nabla v \cdot \nabla v + \int_{\partial\Omega} \mu v v \\
    & \geq \min(1,\mu)\left( \int_\Omega \nabla v \cdot \nabla v +
      \int_{\partial\Omega} v v\right)  \\
    &\geq \min(1,\mu) \|v\|_{1,\Omega}, \quad \forall v \in H^1(\Omega)
  \end{split}
\end{equation}
La forme bilinéaire $a$ est donc coercive et le problème~\ref{prob:15} est
bien posé grâce à Lax-Milgram.

On considère le problème discret suivant
\begin{problem}
  \label{prob:11}
  On cherche $u_h \in L^k_{c,h}(\Omega)$ telle que
  \begin{equation}
    \label{eq:87}
    a(u_h,v_h) = \ell(v_h)\quad \forall v_h \in L^k_{c,h}
  \end{equation}
\end{problem}
Le problème est bien posé  ($L^k_{c,h} \subset H^1(\Omega)$). La convergence de
$u_h$ est donnée par le théorème~\ref{thr:17}.

Considérons $\Omega=[0,1]^2$ et les données $\mu=0.01$, $f=1$ et $g=0$. Le
code \Feel permettant de résoudre le problème discret~\ref{prob:11}, voir
\texttt{../Codes/prudhomme/laplacian/robin.cpp} pour le listing complet.
\lstinputlisting[linerange=12-26,caption={Exemple Laplacien avec conditions de
Robin}]{../Codes/prudhomme/laplacian/robin.cpp}


\subsection{Advection-diffusion-réaction avec diffusion dominante}
\label{sec:advection-diffusion}

On s'intéresse au problème suivant:
\begin{problem}
  \label{prob:10}
  On cherche $u : \Omega \rightarrow \R[]$ telle que
  \begin{equation}
    \label{eq:30}
    \begin{split}
      -\nabla \cdot ( {\bm \alpha} \nabla u  ) + {\bm \beta} \cdot \nabla u + \mu u &= f
      \mbox{ dans } \Omega\\
      u &= 0 \mbox{ sur } \partial \Omega\\
    \end{split}
  \end{equation}
\end{problem}

La variations sur les conditions aux limites de la
section~\ref{sec:cond-aux-limit} s'appliquent.
\begin{itemize}
\item $-\nabla \cdot ( {\bm \alpha} \nabla u  )$ est un terme de diffusion,
\item ${\bm \beta} \cdot \nabla u$ est un terme de convection,
\item $\mu u$ est un terme de réaction
\end{itemize}
Ce type d'équation est très fréquente en ingéniérie, biologie ou encore finance.

On suppose que ${\bm \alpha} \in [L^{\infty}(\Omega)]^{d,d}$, ${\bm \beta} \in
[L^{\infty}(\Omega)]^{d}$ et $\mu \in L^\infty(\Omega)$. On suppose que
l'opérateur $\mathcal{L}$ tel que $\mathcal{L} u = -\nabla \cdot ( \alpha \nabla u
) + \beta \cdot \nabla u + \mu u$ est elliptique au sens suivant:

\begin{definition}[Opérateurs Elliptiques]
  \label{def:39}
  L'opérateur $\mathcal{L}$ est elliptique si il existe une constante $\alpha_0$
  telle que presque pour tout $x\in\Omega$
  \begin{equation}
    \label{eq:88}
    \forall \xi=(\xi_1, \ldots , \xi_n)\in\RR^n,\quad {\ds \sum_{i,j=1}^n  \alpha_{ij}(x) \, \xi_i \, \xi_j  \ge \alpha_0 \, \| \xi \|^2 }
  \end{equation}
\end{definition}

\begin{remark}
  \label{rem:29}
  Le Laplacien est dans la catégorie est opérateurs elliptiques, il correspond à
  ${\bm \beta} = 0$, $\mu = 0$ et ${\bm \alpha}=\mathcal{I}_d$ avec $\mathcal{I}_d$ la
  matrice identité de $\R[d,d]$
\end{remark}


La formulation faible s'écrit
\begin{problem}
  \label{prob:12}
  On cherche $u \in H^1_0(\Omega)$ telle que
  \begin{equation}
    \label{eq:89}
    a(u,v) = \ell(v) \quad \forall v \in H^1_0(\Omega)\\
  \end{equation}
  avec
  \begin{equation}
    \label{eq:90}
    a(u,v)=\int_\Omega ({\bm \alpha} \nabla u ) \cdot \nabla v + ({\bm \beta} \cdot
    \nabla u) v + \mu u v
  \end{equation}
  et
  \begin{equation}
    \label{eq:91}
    \ell(v) = \int_\Omega f v
  \end{equation}
\end{problem}

On note $\gamma = \essinf_{x \in \Omega} (\mu -\frac{1}{2} \nabla \cdot {\bm
\beta})$, on peut alors montrer que sous la condition
\begin{equation}
  \label{eq:92}
  \alpha_0 > - \min( 0, \gamma c_\Omega  )
\end{equation}
où $c_\Omega$ est la constante de l'inégalité de Poincaré, la forme bilinéaire
$a$ est coercive et donc que, grâce au théorème de Lax-Milgram, le problème
\ref{prob:12} est bien posé. La coercivité est garantie si $\alpha_0$ est
suffisamment grand c'est à dire que si la diffusion est dominante.

L'approximation élément fini est similaire à celle du Laplacian, de plus les
variantes sur les conditions aux limites s'appliquent également: condition de
Dirichlet non homogène, de Neumann ou de Robin.



\section{Élasticité Linéaire}
\label{sec:elasticite-lineaire}

On s'intéresse dans cette section à l'approximation par éléments finis de
problèmes de mécanique des milieux continus en 3D.

Si ${\bm f}: \Omega \rightarrow \mathbb{R}^3$ est la charge extérieur
s'appliquant au domaine $\Omega$ et qu'on note $\disp: \Omega \rightarrow
\mathbb{R}^3$ le déplacement de la structure induit par cette charge ${\bm f}$
alors en supposant que les déformations soient petites pour être modélisées
dans le cadre de l'elasticité linéaire on a la relation suivante à
l'équilibre:
\begin{equation}
  \label{eq:31}
  \nabla \cdot \stresst + {\bm f} = {\bm 0} \mbox{ dans } \Omega
\end{equation}
où $\stresst: \Omega \rightarrow \R[d,d]$ est le \emph{tenseur des
contraintes} défini par la relation
\begin{equation}
  \label{eq:93}
  \stresst = \lambda \tr(\deformt) \Id[3] + 2 \mu \deformt
\end{equation}
avec $\deformt : \Omega \rightarrow \R[d,d]$ le \emph{tenseur des déformations} défini par
\begin{equation}
  \label{eq:94}
  \deformt = \frac{1}{2} \left( \nabla \disp + \nabla \disp^T \right),
\end{equation}
$\lambda$ et $\mu$ les coefficients de Lamé et $\Id[3]$ la matrice identité de
$\R[3,3]$. On a alors
\begin{equation}
  \label{eq:95}
  \stresst = \lambda( \nabla \cdot \disp ) \Id[3] + \mu( \nabla \disp + \nabla \disp^T)
\end{equation}

\paragraph{Coefficients de Lamé}
\label{sec:coefficients-de-lame}
Ils sont des coefficients phénoménologiques contraints par les relations
suivants:
\begin{itemize}
\item $\mu >0$
\item $\lambda + \frac{2}{3} \mu \ge 0$
\end{itemize}
Dans ce qui suit, on supposera que $\lambda \ge 0$ et ces coefficients
constants. D'un point de vue pratique, ces coefficients sont obtenus par les
\emph{module d'Young} $E$ et  \emph{coefficient de Poisson} $\nu$ tels que
\begin{equation}
  \label{eq:96}
  \lambda = \ds \frac{E \nu}{( 1+\nu )*( 1-2 \nu )} , \quad \mu =\frac{E}{2 ( 1+\nu )}
\end{equation}


\lstinputlisting[linerange=12-33,caption={Exemple en élasticité linéaire}]{../Codes/prudhomme/elasticite_lineaire/trous.cpp}


%%% Local Variables:
%%% coding: utf-8
%%% mode: latex
%%% TeX-PDF-mode: t
%%% TeX-parse-self: t
%%% TeX-auto-save: t
%%% x-symbol-8bits: nil
%%% TeX-auto-regexp-list: TeX-auto-full-regexp-list
%%% TeX-master: "mef-intro"
%%% ispell-local-dictionary: "american"
%%% End:

\usepackage{beamerarticle}

\chapter{Approximation de problèmes mixtes}
\label{cha:appr-de-probl-1}

\section{Model Problems}
\label{sec:model-problems}

  We consider now model problems as systems of PDEs where several functions are
  unknowns and which don't play the same roles \emph{mathematically} and
  \emph{physically}.
  \begin{problem}{Stokes}
    \begin{equation}
      \label{eq:chmixte:98}
      \left\{\begin{array}[c]{rl}
          -\Delta u + \nabla p & = f\ \mbox{ in } \Omega\\
          \nabla \cdot u & = 0\ \mbox{ in } \Omega
        \end{array}\right.
    \end{equation}
    where $u: \Omega \mapsto \R{d}$ is a velocity and $p: \Omega \mapsto \R{}$
    is a pressure.
  \end{problem}
  \begin{problem}{Darcy}
    \begin{equation}
      \label{eq:chmixte:99}
      \left\{\begin{array}[c]{rl}
          \sigma + \nabla u & = f\ \mbox{ in } \Omega\\
          \nabla \cdot \sigma & = g\ \mbox{ in } \Omega
        \end{array}\right.
    \end{equation}
    where $\sigma: \Omega \mapsto \R{d}$ is a velocity and $u: \Omega \mapsto \R{}$
    is a hydraulic charge(pressure).

  \end{problem}


\section{Applications}

  We shall focus on Stokes, but the abstract setting of the next section is
  the same for Stokes and Darcy.

\begin{block}{Stokes and incompressible Navier-Stokes for Newtonian fluids}
  The Stokes model is the basis for fluid mechanics models and is a
  simplication of the Navier-Stokes equations where the viscous effects/terms
  are much bigger than the convective ones
  \begin{equation}
    \label{eq:chmixte:3}
    \left\{\begin{array}[c]{rl}
           \rho( \frac{\partial u}{\partial t} + u \cdot \nabla u) - \nu \Delta u + \nabla p & = f\ \mbox{ in } \Omega\\
           \nabla \cdot u & = 0\ \mbox{ in } \Omega
         \end{array}\right.
     \end{equation}
     The first equation results from the conservation of momentum and the second
     from the conservation of mass.
   \end{block}

  \begin{remark}{}
    The well-posedness of these problems results from a so-called \alert{inf-suf
    condition} which is not automatically transfered at the \emph{discrete level}.
  \end{remark}

  \begin{remark}{In practice}
    In order to ensure that the finite element approximation is well-posed, we
    will need to choose approximation spaces that satisfy a \emph{compatibility
    condition} that ensures that a \emph{discrete inf-sup condition} is satisfied.
  \end{remark}

\section{Saddle point problems}
\label{sec:saddle-point-probl}

\subsection{Abstract Continuous Setting}
  Denote
  \begin{itemize}
  \item $X$ and $M$ two Hilbert spaces\footnote{An euclidian space which is
  complete for the norm induced by the scalar product}
  \item two linear forms $f \in X'=\mathcal{L}(X, \R{})$ and $g \in
    M'=\mathcal{L}(M, \R{})$
  \item $a \in \mathcal{L}(X\times X, \R{})$ and $b \in \mathcal{L}(X\times M,
    \R{})$ two bilinear forms
  \end{itemize}
  We are interested in the following abstract problem:
  \begin{problem}[Abstract formulation]
    \label{prob:chmixte:1}
    Look for $(u,p) \in X \times M$ such that
    \begin{equation}
      \label{eq:chmixte:4}
      \left\{
        \begin{array}[c]{rl}
          a(u,v) + b(v,p) & = f(v), \quad \forall v \in X\\
          b(u,q) & = g(q), \quad \forall q \in M
        \end{array}
      \right.
    \end{equation}
  \end{problem}



\subsection{Definition of a saddle point problem}
  \begin{definition}[Saddle point problem]
    \label{def:chmixte:1}
    If the bilinear form $a$ is \emph{symmetric} and positive on $X\times X$, we
    say that problem~\ref{prob:chmixte:1} is a \emph{saddle point problem}.
  \end{definition}

  \begin{remark}{Structure of the problem}
    \begin{itemize}
    \item the space of solution is the same of the test space
    \item the unknown $p$ does not appear in the second equation
    \item the unknown functions $u$ and $p$ are coupled via the same bilinear
      form $b$ is the first and second equation.
    \end{itemize}
    The next question is : \alert{is the problem~\ref{prob:chmixte:1} well-posed?}
  \end{remark}

\section{Well posed problem}
\label{sec:well-posed-problemframe}


\subsection{Reformulation}
  Let's rewrite Problem~\ref{prob:chmixte:1}. Denote $V=X\times M$ and introduce $c
  \in \mathcal{L}(V\times V, \R{})$ such that
  \begin{equation}
    \label{eq:chmixte:5}
    c((u,p),(v,q)) = a(u,v)+b(v,p)+b(u,q)
  \end{equation}
  and $h\in \mathcal{L}(V,\R{})$ such that
  \begin{equation}
    \label{eq:chmixte:6}
    h(v,q) = f(v)+g(q)
  \end{equation}
  then problem~\ref{prob:chmixte:1} reads
  \begin{problem}
    \label{prob:chmixte:2}
    Look for $(u,p) \in V$ such that
    \begin{equation}
      \label{eq:chmixte:7}
        \begin{array}[c]{rl}
          c((u,p), (v,q)) & = h(v,q), \quad \forall (v,q) \in V
        \end{array}
      \end{equation}
  \end{problem}


  \begin{theorem}[Well posedness of the Saddle-Point problem]
    \label{thr:chmixte:1}
    We suppose that $a$ is coercive over $X$, the problem~\ref{prob:chmixte:2} is
    \emph{well-posed} if and only if the bilinear form $b$ satisfies the
    following inf-sup condition: there exists $\beta > 0$ such that
    \begin{equation}
      \label{eq:chmixte:8}
      \inf_{q \in M} \sup_{v \in X} \frac{b(v,q)}{||v||_X ||q||_M} \geq \beta
    \end{equation}
  \end{theorem}
  \begin{remark}
    \begin{itemize}
    \item Lax-Milgram provides only a sufficient condition for well-posedness
    \item the form $c$ in Problem~\ref{prob:chmixte:2} does not satisfy Lax-Milgram.
    \end{itemize}
  \end{remark}


  Let's introduce the so-called \emph{Lagrangian} $l \in \mathcal{L}(X\times M,
  \R{})$ defined by
  \begin{equation}
    \label{eq:chmixte:9}
    l(v,q) =  \frac{1}{2} a(v,v) + b(v,q) - f(v) - g(q)
  \end{equation}
  \begin{definition}[Saddle point]
    \label{def:chmixte:2}
    We say that the point $(u,p)\in X\times M$ is a saddle point of $l$ if
    \begin{equation}
      \label{eq:chmixte:10}
      \forall (v,q) \in X\times M, \quad l(u,q) \leq l(u,p) \leq l(v,p)
    \end{equation}
  \end{definition}
  \begin{proposition}[Unicity of the saddle point]
    \label{prop:chmixte:1}
    Under the hypothesys of theorem~\ref{thr:chmixte:1}, the Lagrangian $l$ defined
    by~\eqref{eq:chmixte:9} has a \emph{unique} saddle point. Moreover this saddle point
    is the the unique solution of problem~\ref{prob:chmixte:1}.
  \end{proposition}

\section{Finite element approximation}
\label{sec:finite-elem-appr}

\subsection{Abstract Discrete Problem}
  We now turn to the approximation of the problem~\ref{prob:chmixte:1} by a \emph{standard
  Galerkin method} in a \emph{conforming} way.

  Denote the two spaces $X_h \subset X$ and $M_h \subset M$, we consider the
  following problem:
  \begin{problem}
    \label{prob:chmixte:3}
    Look for $(u_h,p_h) \in X_h \times M_h$ such that
    \begin{equation}
      \label{eq:chmixte:11}
      \left\{
        \begin{array}[c]{rl}
          a(u_h,v_h) + b(v_h,p_h) & = f(v_h), \quad \forall v_h \in X_h\\
          b(u_h,q_h) & = g(q_h), \quad \forall q_h \in M_h
        \end{array}
      \right.
    \end{equation}
  \end{problem}


  \begin{theorem}[Well-posedness of the discrete problem]
    \label{thr:chmixte:2}
    We suppose that $a$ is coercive over $X$ and that $X_h \subset X$ and $M_h
    \subset M$. Then the problem~\ref{prob:chmixte:3} is \emph{well-posed} if and only
    if the following \emph{discrete inf-sup condition}  is satisfied: there
    exists $\beta_h  > 0$ such that
    \begin{equation}
      \label{eq:chmixte:12}
      \inf_{q_h \in M_h} \sup_{v_h \in X_h} \frac{b(v_h,q_h)}{||v_h||_{X_h} ||q_h||_{M_h}} \geq \beta_h
    \end{equation}
  \end{theorem}

  \begin{remark}{Compatibility condition}
    Problem~\ref{prob:chmixte:3} to be well posed, requires that the spaces $X_h$ and
    $M_h$ satisfy the condition~\eqref{eq:chmixte:12}. This is known as the
    Babuska-Brezzi (BB) or Ladyhenskaya-Babuska-Brezzi (LBB).
  \end{remark}


  Regarding error analysis, we have the following lemma
  \begin{lemma}
    \label{lem:1}
      Thanks to the Lemma of Céa applied to Saddle-Point Problems, the unique
      solution $(u,p)$ of problem~\ref{prob:chmixte:3}  satisfy
      \begin{equation}
        \label{eq:chmixte:13}
        \begin{array}[c]{rl}
          ||u-u_h||_X & \leq c_{1h} \inf_{v_h \in X_h}  ||u-v_h||_X + c_{2}
          \inf_{q_h \in M_h}  ||q-q_h||_M\\
          ||p-p_h||_X & \leq c_{3h} \inf_{v_h \in X_h}  ||u-v_h||_X + c_{4h} \inf_{q_h \in M_h}  ||q-q_h||_M
        \end{array}
      \end{equation}
      where
      \begin{itemize}
      \item $c_{1h} =
        (1+\frac{||a||_{X,X}}{\alpha})(1+\frac{||b||_{X,M}}{\beta_h})$ where
        $\alpha$ is the coercivity constant of $a$ over X.
      \item $c_{2} = \frac{||b||_{X,M}}{\alpha}$
      \item $c_{3h} = c_{1h} \frac{||a||_{X,X}}{\beta_h}$, $c_{4h} = 1+ \frac{||b||_{X,M}}{\beta_h}+\frac{||a||_{X,X}}{\beta_h}$
      \end{itemize}
  \end{lemma}
  \begin{remark}
    The constants $c_{1h}, c_{3h}, c_{4h}$ are as large as $\beta_h$ is small.
  \end{remark}



\subsection{Linear system associated}
\label{sec:linear-system}

The discretisation process leads to a linear system. We denote
  \begin{itemize}
  \item $N_u = \dim {X_h}$
  \item $N_p = \dim {M_h}$
  \item $\{\phi_i\}_{i=1,...,N_u}$ a basis of $X_h$
  \item $\{\psi_k\}_{k=1,...,N_p}$ a basis of $M_h$
  \item for all $u_h = \sum_{i=1}^{N_u} u_i \phi_i$, we associate $U \in
    \R{N_u}$, $U=(u_1,\ldots,u_{N_u})^T$, the component vector of $u_h$ is
    $\{\phi_i\}_{i=1,\ldots,N_u}$
  \item for all $p_h = \sum_{k=1}^{N_p} u_k \psi_k$, we associate $P \in
    \R{N_p}$, $P=(p_1,\ldots,p_{N_p})^T$, the component vector of $p_h$ is $\{\psi_k\}_{k=1,\ldots,N_p}$
  \end{itemize}

  \begin{block}{}
    The matricial form of problem ~\ref{prob:chmixte:3} reads
    \begin{equation}
      \label{eq:chmixte:15}
      \begin{bmatrix}
        \mathcal{A} & \mathcal{B}^T\\
        \mathcal{B} & 0
      \end{bmatrix}
      \begin{bmatrix}
        U \\
        P
      \end{bmatrix}
      =
      \begin{bmatrix}
        F\\
        G
      \end{bmatrix}
    \end{equation}
    where the matrix $\mathcal{A} \in \R{N_u,N_u}$ and $\mathcal{B} \in
    \R{N_p,N_u}$ have the coefficients
    \begin{equation}
      \label{eq:chmixte:16}
      \mathcal{A}_{ij} = a(\phi_j,\phi_i), \quad \mathcal{B}_{ki} = b(\phi_i,\psi_k)
    \end{equation}
    and the vectors $\mathcal{F} \in \R{N_u}$ and $\mathcal{G} \in \R{N_p}$
    have the coefficients
    \begin{itemize}
    \item $F_i=f(\phi_i)$
    \item $G_k=g(\psi_k)$
    \end{itemize}
  \end{block}


  \begin{remark}[Remarks on the linear system]
    \label{rem:2}
    \begin{enumerate}
    \item Since $a$ is symmetric and coercive, $\mathcal{A}$ is \emph{symmetric
       positive definite}
     \item The matrix of the system \eqref{eq:chmixte:15} is symmetric but not positive
     \item The inf-sup condition~\eqref{eq:chmixte:12} is equivalent to the fact that
       $\mathcal{B}$ is of \emph{maximum rank}, \emph{i.e.} $\ker(\mathcal{B}^T)
       = \{0 \}$.
     \item From theorem~\ref{thr:chmixte:2}, the matrix of the system~\eqref{eq:chmixte:15} is
       invertible
    \end{enumerate}
  \end{remark}


  \begin{block}{When the Inf-sup condition is not satisfied}
    The counter examples when the inf-sup condition~\eqref{eq:chmixte:12} is not
    satisfied(e.g. $\mathcal{B}$ is not maximum rank ) occur usually in two
    cases:
    \begin{enumerate}
    \item $\dim {M_h} > \dim {X_h}$: the space of pressure is too large for the
      matrix $\mathcal{B}$ to be maximum rank. In that case $\mathcal{B}$ is
      injective ($\ker(\mathcal{B}) = \{0\})$. We call this \emph{locking}.
    \item there exists a vector $Q^* \neq 0$ in $\ker(\mathcal{B}^T)$. The
      discrete field $q^*_h$ in $M_h$, $q^*_h=\sum_{k=1}^{N_p} Q^*_k \psi_k$,
      associated is called a \emph{spurious mode}. $q^*_H$ is such that
      \begin{equation}
        \label{eq:chmixte:14}
        b(v_h,q^*_h)=0.
      \end{equation}

    \end{enumerate}
  \end{block}

We now introduce the Uzawa matrix.
\begin{block}{Uzawa}
  The matrix
    \begin{equation}
      \label{eq:chmixte:17}
      \mathcal{U} = \mathcal{B} \mathcal{A}^{-1} \mathcal{B}^T
    \end{equation}
    is called the \emph{Uzawa matrix}. It is \emph{symmetric positive definite}
    from the properties of $\mathcal{A}$, $\mathcal{B}$
  \end{block}
  \begin{block}{Applications}
    The Uzawa matrix occurs when eliminating the velocity in
    system~\eqref{eq:chmixte:15} and get a linear system on $P$:
    \begin{equation}
      \label{eq:chmixte:18}
      \mathcal{U} P = \mathcal{B} \mathcal{A}^{-1} F - G
    \end{equation}
    then one application is to solve \eqref{eq:chmixte:15} by solving \eqref{eq:chmixte:18}
    iteratively and compute the velocity afterwards.
  \end{block}

\section{Mixed finite element for Stokes}
\label{sec:mixed-finite-element}

\subsection{Variational formulation }
\label{sec:vari-form-}

We start with the Well-posedness at the continuous level
  \begin{itemize}
  \item   We consider the model problem~\eqref{eq:chmixte:1} with homogeneous Dirichlet
  condition on velocity $u = 0$ on $\partial \Omega$
  \item   We suppose the $f \in [L^2(\Omega)]^d$ and $g \in L^2(\Omega)$ with
  \begin{equation}
    \label{eq:chmixte:20}
    \int_\Omega g = 0
  \end{equation}
    Introduce
  \begin{equation}
    \label{eq:chmixte:19}
    L^2_0(\Omega) = \Big\{ q \in L^2(\Omega): \int_\Omega q = 0 \Big\}
  \end{equation}

  \end{itemize}

  \begin{remark}{The condition~\eqref{eq:chmixte:20}}
    The condition~\eqref{eq:chmixte:20} comes from the divergence theorem applied to the
    divergence equation and the fact that $u=0$ on the boundary
    \begin{equation}
      \label{eq:chmixte:21}
      \int_\Omega g = \int_\Omega \nabla \cdot u = \int_{\partial \Omega} u
      \cdot n = 0
    \end{equation}
    This is a \emph{necessary} condition for the existence of a solution $(u,p)$
    for the Stokes equations with these boundary conditions.
  \end{remark}


We turn now to the variational formulation.  The Stokes problem reads
    \begin{problem}[Stokes Variational Formulation]
      \label{prob:chmixte:4}
      Look for $(u,p) \in [H^1_0(\Omega)]^d \times L^2_0(\Omega)$ such that
      \begin{equation}
        \label{eq:chmixte:25}
      \left\{
        \begin{array}[c]{rl}
          \int_\Omega \nabla u : \nabla v -\int_\Omega p \nabla \cdot v  & =
          \int_\Omega f \cdot v, \quad \forall v \in [H^1_0(\Omega)]^d\\
          \int_\Omega q \nabla \cdot u & = - \int_\Omega g q, \quad \forall q \in L^2_0(\Omega)
        \end{array}
      \right.
    \end{equation}
    \end{problem}
    We retrieve the problem~\ref{prob:chmixte:1} with $X=[H^1_0(\Omega)]^d$ and
    $M=L^2_0(\Omega)$ and
    \begin{equation}
      \label{eq:chmixte:22}
      \begin{array}[c]{rlrl}
      a(u,v) &= \int_\Omega \nabla u : \nabla v,& \quad b(v,p) &= -\int_\Omega p
      \nabla \cdot v,\\
      \quad f(v) &=  \int_\Omega f \cdot v,& \quad g(q) &= - \int_\Omega g q
      \end{array}
    \end{equation}
    \begin{remark}{Pressure up to a constant}
      The pressure is known up to a constant, that's why we look for them in
      $L^2_0(\Omega)$ to ensure uniqueness.
    \end{remark}



% \begin{frame}
%   \frametitle{Well-posedness}

%   \begin{lemma}
%     \label{lem:2}
%     Denote $\Omega$ a domain of $\R{d}, d \geq 2$, Then the operator
%     \begin{equation}
%       \label{eq:chmixte:23}
%       \nabla \cdot : [H^1_0(\Omega)]^d \mapsto L^2_0(\Omega)
%     \end{equation}
%     is surjective.
%   \end{lemma}

%   This lemma and the theorem
%

\subsection{Finite element approximation}
\label{sec:finite-elem-appr-1}

  Denote $X_h \subset [H^1_0(\Omega)]^d$ and $M_h \subset L^2_0(\Omega)$

  \begin{problem}[Discrete Stokes formulation]
    \label{prob:chmixte:5}
    Look for $(u_h,p_h) \in X_h \times M_h$ such that
    \begin{equation}
      \label{eq:chmixte:24}
      \left\{
        \begin{array}[c]{rl}
          \int_\Omega \nabla u_h : \nabla v_h + \int_\Omega p_h \nabla \cdot v_h
          & = \int_\Omega f \cdot v_h, \quad \forall v_h \in X_h\\
          \int_\Omega q_h \nabla \cdot u_h & = -\int_\Omega g q_h, \quad \forall q_h \in M_h
        \end{array}
      \right.
    \end{equation}
  \end{problem}
  \begin{remark}[Well-posedness]
    \label{rem:1}
    This problem, thanks to theorem~\ref{thr:chmixte:2} is well-posed if and only if
    $X_h$ and $M_h$ are such that there exists $\beta_h > 0$
    \begin{equation}
      \label{eq:chmixte:26}
      \inf_{q_h \in M_h} \sup_{v_h \in X_h} \frac{\int_\Omega q_h \nabla \cdot v_h}{||v_h||_{X_h} ||q_h||_{M_h}} \geq \beta_h
    \end{equation}
  \end{remark}

% \lstinputlisting{mystokes.cpp}

\subsection{Some counter examples: bad finite element for Stokes}
\label{sec:counter-examples}


In this section we present two classical bad finite element approximations.

\subsubsection{Finite element $\poly{P}_1/\poly{P}_0$: locking}
    thanks to the Euler relations


    \begin{equation}
      \label{eq:chmixte:28}
      \begin{array}[c]{rl}
        N_{\mathrm{cells}} - N_{\mathrm{edges}} + N_{vertices}  &= 1-I\\
      N^\partial_{\mathrm{vertices}} - N^\partial_{\mathrm{edges}} &= 0
      \end{array}
    \end{equation}
    where $I$ is the number of holes in $\Omega$.  We have that $\dim {M_h} =
    N_{\mathrm{cells}}$, $\dim {X_h} = 2 N^i_{\mathrm{vertices}}$ and so
    \begin{equation}
      \label{eq:chmixte:29}
      \dim {M_h} - \dim {X_h} = N_{\mathrm{cells}} - 2 N^i_{\mathrm{vertices}} =
      N^\partial_{\mathrm{edges}} - 2 > 0
    \end{equation}
    so $M_h$ is too rich for the condition \eqref{eq:chmixte:26} and we have
    $\ker(\mathcal{B}) = \{0\}$ such that the \emph{only} discrete $u_h^*$,
    with components $U^*$, satisfying $\mathcal{B} U^*$ is the nul field, $U^*=0$.


\subsubsection{Finite element $\poly{Q}_1/\poly{P}_0$: spurious mode}

    We can construct in that case a function $q_h^*$ on a uniform grid which is
    equal alternatively -1, +1 (chessboard) in the cells of  the mesh, then
    \begin{equation}
      \label{eq:chmixte:27}
      \forall v_h \in [Q^1_{c,h}]^d, \quad \int_\Omega q^*_h \nabla \cdot v_h = 0
    \end{equation}
    and thus the associated $X_h$, $M_h$ do not satisfy  the condition~\eqref{eq:chmixte:26}.

\subsubsection{Finite element $\poly{P}_1/\poly{P}_1$: spurious mode}
    We can construct in that case a function $q_h^*$ on a uniform grid which is
    equal alternatively -1, 0, +1 at the vertices of the mesh, then
    \begin{equation}
      \label{eq:chmixte:27}
      \forall v_h \in [P^1_{c,h}]^d, \quad \int_\Omega q^*_h \nabla \cdot v_h = 0
    \end{equation}
    and thus the associated $X_h$, $M_h$ do not satisfy  the condition~\eqref{eq:chmixte:26}.


\subsection{Mini-Element}
\label{sec:finite-elem-stok}

  The problem with the $\poly{P}_1/\poly{P}_1$ mixed finite element is that the
  velocity is not \emph{rich} enough. A cure is to add a function $v_h^*$ in the
  velocity approximation space to ensure that
  \begin{equation}
    \label{eq:chmixte:30}
    \int_\Omega q^*_h \nabla \cdot v_h^* \neq 0
  \end{equation}
  where $q_h^*$ is the spurious mode. To do that we add the bubble function to
  the $\poly{P}_1$ velocity space.
  \begin{definition}[Bubble function]
    \label{def:chmixte:3}
    Recall the construction of finite elements on a reference convex $\hat{K}$.
    We say that $\hat{b}: \hat{K} \mapsto \R{}$ is a bubble function if:
    \begin{itemize}
    \item $\hat{b} \in H^1_0(\hat{K})
$
    \item $0 \leq \hat{b}(\hat{x}) \leq 1, \quad \forall \hat{x} \in \hat{K}$
    \item $\hat{b}(\hat{C}) = 1, \quad \mbox{where} \hat{C}$ is the barycenter of $\hat{K}$
    \end{itemize}
  \end{definition}


  \begin{exampleblock}{Example}
    The function
    \begin{equation}
      \label{eq:chmixte:31}
      \hat{b} = (d+1)^{d+1} \Pi_{i=0}^d\ \hat{\lambda}_i
    \end{equation}
    where $(\hat{\lambda}_0, \ldots, \hat{\lambda}_d)$ denote the barycentric
    coordinates on $\hat{K}$
  \end{exampleblock}


  Denote $\hat{b}$ a bubble fonction on $\hat{K}$, we set
  \begin{equation}
    \label{eq:chmixte:32}
    \hat{P} = [\poly{P}_1(\hat{K}) \oplus \vect{} (\hat{b})]^d,
  \end{equation}
  and introduce
  \begin{eqnarray}
    \label{eq:chmixte:33}
    X_h &=& \Big\{ v_h \in [C^0(\bar{\Omega})]^d : \forall K \in \mathcal{T}_h, v_h
    \circ T_K \in \hat{P}; v_{h_|{\partial \Omega}} = 0 \Big\}\\
    M_h &=& P^1_{c,h}
  \end{eqnarray}

  \begin{lemma}[Compatibility condition]
    \label{lem:3}
    The spaces $X_h$ and $M_h \cap L^2_0(\Omega)$ satisfy the compatibility
    condition~\eqref{eq:chmixte:26} uniformly in $h$.
  \end{lemma}

  \begin{theorem}[Error estimation for the mini-element]
    \label{thr:chmixte:3}
    Suppose that $(u,p)$, solution of problem~\ref{prob:chmixte:1}, is smooth enough,
    ie. $u \in [H^2(\Omega)]^d \cap [H^1_0(\Omega)]^d$ and $p\in H^1(\Omega)
    \cap L^2_0(\Omega)$. Then there exists a constant $c$ such that for all $h
    >0$
    \begin{equation}
      \label{eq:chmixte:34}
      \| u- u_h \|_{1,\Omega} + \|p-p_h\|_{0,\Omega} \leq c h (\|u\|_{2,\Omega}
      +\|p\|_{1,\Omega})
    \end{equation}
    and if the Stokes problem is stabilizing then
    \begin{equation}
      \label{eq:chmixte:35}
      \|u-u_h\|_{0,\Omega} \leq c h^2 ( \|u\|_{2,\Omega} +\|p\|_{1,\Omega}).
    \end{equation}
  \end{theorem}
  \begin{definition}[Stabilizing Stokes problem]
    \label{def:chmixte:4}
    We say that the Stokes problem is stabilizing if there exists a constant
    $c_S$ such that for all $f \in [L^2(\Omega)]^d$, the unique solution $(u,p)$
    of \eqref{eq:chmixte:25} with $g=0$ is such that:
    \begin{equation}
      \label{eq:chmixte:36}
      \|u\|_{2,\Omega} + \|p\|_{1,\Omega} \leq c_S \|f\|_{0,\Omega}
    \end{equation}
    A sufficient condition for stabilizing Stokes problem is that the $\Omega$
    is a polygonal convex in 2D or of class $C^1$  in $\R{d}, d=2,3$.
  \end{definition}

\subsection{Taylor-Hood Element}
\label{sec:taylor-hood-element}


  The mini-element solved the compatibility condition problem, but the error
  estimation in equation~\eqref{eq:chmixte:35} is not optimal in the sense that
  \begin{enumerate}
  \item the pressure space is sufficiently rich to enable a $h^2$ convergence in
    the pressure error,
  \item but the velocity space is not rich enough to ensure a $h^2$ convergence
    in the velocity error.
  \end{enumerate}

  \begin{block}{Taylor-Hood element}
    The idea of the Taylor-Hood element is to enrich even more the velocity
    space to ensure optimal convergence in $h$. Here we will take $[\poly{P}_2]^d$ for
    the velocity and $\poly{P}_1$ for the pressure.
  \end{block}


  Introduce
  \begin{eqnarray}
    \label{eq:chmixte:39}
    X_h &=&  [P^2_{c,h}]^d\\
    M_h &=& P^1_{c,h}
  \end{eqnarray}

  \begin{lemma}[Compatibility condition]
    \label{lem:3}
    The spaces $X_h$ and $M_h \cap L^2_0(\Omega)$ satisfy the compatibility
    condition~\eqref{eq:chmixte:26} uniformly in $h$.
  \end{lemma}

  \begin{theorem}[Error estimation for the Taylor-Hood element]
    \label{thr:chmixte:3}
    Suppose that $(u,p)$, solution of problem~\ref{prob:chmixte:1}, is smooth enough,
    ie. $u \in [H^3(\Omega)]^d \cap [H^1_0(\Omega)]^d$ and $p\in H^2(\Omega)
    \cap L^2_0(\Omega)$. Then there exists a constant $c$ such that for all $h
    >0$
    \begin{equation}
      \label{eq:chmixte:40}
      \| u- u_h \|_{1,\Omega} + \|p-p_h\|_{0,\Omega} \leq c h^2 (\|u\|_{3,\Omega}
      +\|p\|_{2,\Omega})
    \end{equation}
    and if the Stokes problem is stabilizing then
    \begin{equation}
      \label{eq:chmixte:41}
      \|u-u_h\|_{0,\Omega} \leq c h^3 ( \|u\|_{3,\Omega} +\|p\|_{2,\Omega}).
    \end{equation}
  \end{theorem}


  \begin{block}{Generalized Taylor-Hood element}
    We consider the mixed finite elements $\poly{P}_k/\poly{P}_{k-1}$ and
    $\poly{Q}_k/\poly{Q}_{k-1}$ which allows to approximate the velocity and
    pressure respectively with, on Simplices
    \begin{eqnarray}
        \label{eq:chmixte:42}
        X_h &=&  [P^{k}_{c,h}]^d\\
        M_h &=& P^{k-1}_{c,h}
      \end{eqnarray}
      On Hypercubes
      \begin{eqnarray}
        \label{eq:chmixte:43}
        X_h &=&  [Q^{k}_{c,h}]^d\\
        M_h &=& Q^{k-1}_{c,h}
      \end{eqnarray}

      We then have
      \begin{equation}
      \label{eq:chmixte:40}
      \|u-u_h\|_{0,\Omega} + h ( \| u- u_h \|_{1,\Omega} + \|p-p_h\|_{0,\Omega} ) \leq c h^{k+1} (\|u\|_{k+1,\Omega}
      +\|p\|_{k,\Omega})
    \end{equation}
  \end{block}


\begin{block}{Other Discretization spaces}
  \begin{itemize}
   \item Discrete inf-sup condition: dictates the choice of spaces
   \item Inf-sup stables spaces:
	\begin{itemize}
	  \item $\mathbb Q_k$-$\mathbb Q_{k-2}$, $\mathbb Q_k$-$\mathbb Q^{disc}_{k-2}$
	  \item $\mathbb P_k$-$\mathbb P_{k-1}$, $\mathbb P_k$-$\mathbb P_{k-2}$, $\mathbb P_k$-$\mathbb P^{disc}_{k-2}$
	  \item Discrete inf-sup constant independent of $h$, but dependent on $k$
	\end{itemize}
  \end{itemize}
 \end{block}


\subsection{Numerical validation: Test case}

  We consider the Kovasznay solution of the steady Stokes equations, see Kovasznay \cite{kovasznay_test}. The exact solution is
\begin{equation}\label{kovasznay_problem}
\begin{array}{r c l}
 \displaystyle \mathbf{u}(x,y) & = & \displaystyle \prect{1 - e^{\lambda x } \cos (2 \pi y), \frac{\lambda}{2 \pi} e^{\lambda x } \sin (2 \pi y)}^T \\
 \displaystyle p(x,y) & = & \displaystyle -\frac{e^{2 \lambda x}}{2} \\
 \displaystyle \lambda & = & \displaystyle \frac{1}{2 \nu} - \sqrt{\frac{1}{4\nu^2} + 4\pi^2}.
\end{array}
\end{equation}
The domain is defined as  $\domain = (-0.5,1) \times (-0.5,1.5)$ and $\nu = 0.035$. The forcing term for the momentum equation is obtained from the solution and is
\begin{equation}
 \mathbf{f} = \left( e^{\lambda x}  \left( \left( \lambda^2 - 4\pi^2 \right) \nu \cos (2\pi y) - \lambda e^{\lambda x} \right), e^{\lambda x} \nu \sin (2 \pi y) (-\lambda^2 + 4 \pi^2)           \right)^T
\end{equation}
Dirichlet boundary conditions are derived from the exact solution.






%%% Local Variables:
%%% coding: utf-8
%%% mode: latex
%%% TeX-PDF-mode: t
%%% TeX-parse-self: t
%%% TeX-auto-save: t
%%% x-symbol-8bits: nil
%%% TeX-auto-regexp-list: TeX-auto-full-regexp-list
%%% TeX-master: "mef-intro"
%%% ispell-local-dictionary: "american"
%%% End:

%\include{ch4}
%\chapter{Convergence de la m\'ethode des \'el\'ements finis}
%
\section{Introduction}
\noindent
%
On suppose ici que l'on r\'esout un probl\`eme sur un domaine $\Omega\in\RR^n$ de fa\c{c}on approch\'ee par m\'ethode d'\'el\'ements finis.\\
Le but de ce chapitre est de fournir une estimation de l'erreur $\|u-u_h\|_m$ o\`u $\|.\|_m$ d\'esigne la norme $H^m$. La r\'egularit\'e de $u$ et de $u_h$ (et donc les valeurs possibles pour $m$) d\'ependant \'evidemment du probl\`eme continu et du type d'\'el\'ements finis choisis pour sa r\'esolution, on exposera ici la d\'emarche de fa\c{c}on g\'en\'erale, en supposant les fonctions suffisamment r\'eguli\`eres par rapport \`a la valeur de $m$. En pratique, on aura le plus souvent $m=$ 0, 1 ou 2.
%
%
\saut
On notera ${\cal T}_h$ le maillage de $\Omega$ consid\'er\'e. On supposera ici le domaine $\Omega$ polygonal, ce qui permet de recouvrir exactement $\Omega$ par le maillage. Si ce n'est pas le cas, les calculs qui suivent doivent \^etre modifi\'es pour tenir compte de l'\'ecart entre le domaine couvert par le maillage et le domaine r\'eel.
\saut
%
%
Les diff\'erentes \'etapes du calcul seront, de fa\c{c}on assez sch\'ematique, les suivantes :
\begin{center}
\begin{tabular}{lp{8 cm}}
$\|u-u_h\|_m \le C \|u-\pi_h u\|_m$ & L'erreur d'approximation est born\'ee par l'erreur d'interpolation\\
%
$\ds{\|u-\pi_h u\|_m^2 = \sum_{K\in{\cal T}_h}  \|u-\pi_h u\|_{m,K}^2 }$ & On se ram\`ene \`a des majorations locales sur chaque \'el\'ement\\
%
$\ds{\|u-\pi_h u\|_{m,K} \le C(K) \|\hat{u}-\hat{\pi}\hat{u}\|_{m,\hat{K}} }$ & On se ram\`ene \`a l'\'el\'ement de r\'ef\'erence\\
 & \\
%
$\ds{\|\hat{u}-\hat{\pi} \hat{u}\|_{m,\hat{K}} \le \hat{C} |\hat{u}|_{k+1,\hat{K}} }$ & Majoration sur l'\'el\'ement de r\'ef\'erence\\
 & \\
%
$\ds{\|u-\pi_h u\|_m \le C' h^{k+1-m} |u|_{k+1} }$ & Assemblage des majorations locales
%
\end{tabular}
\end{center}
%
\section{Calcul de majoration d'erreur}
%
\subsection{Etape 1: majoration par l'erreur d'interpolation}
%
\noindent
%
L'\'equation (\ref{eq:cea}) \'etablie au \S \/ \ref{sec:estim} indique que
\be
\|u-u_h\|_m \le \frac{M}{\alpha}\; \|u-v_h\|_m \quad \forall v_h\in V_h
\ee
%
On peut l'appliquer dans le cas particulier o\`u $v_h=\pi_h u$, ce qui donne
\be
\|u-u_h\|_m \le \frac{M}{\alpha}\; \|u-\pi_h u\|_m
\label{eq:cea2}
\ee
%
%
\subsection{Etape 2: D\'ecomposition sur les \'el\'ements}
%
\noindent
%
On a, avec des notations \'evidentes :
$$
\begin{array}{lll}
\ds{\|u-\pi_h u\|_m^2 }& = &\ds{\sum_{K\in{\cal T}_h} \|u-\pi_h u\|_{m,K}^2 }\\
& = & \ds{\sum_{K\in{\cal T}_h} \sum_{l=0}^m |u-\pi_h u|_{l,K}^2}
\end{array}
$$
%
Le calcul est donc ramen\'e \`a un calcul sur chaque \'el\'ement, pour toutes les semi-normes $|.|_{l,K},\; l=0,\ldots,m$.
%
%
\subsection{Etape 3: Passage \`a l'\'el\'ement de r\'ef\'erence}
%
\noindent
%
{\bf Th\'eor\`eme :} Soit $K$ un \'el\'ement quelconque de ${\cal T}_h$, et $\hat{K}$ l'\'el\'ement de r\'ef\'erence. Soit $F$ la transformation affine de $\hat{K}$ vers $K$ : $F(\hat{x})=B\hat{x}+b$, avec $B$ inversible. On a :
\be
\forall v\in H^l(K), \qquad |\hat{v}|_{l,\hat{K}} \le C(l,n)\; \|B\|^l_2 \; |\hbox{det} B|^{-1/2} \, |v|_{l,K}
\label{eq:majref}
\ee
%
{\it D\'emonstration} : Il s'agit l\`a en fait d'un simple r\'esultat de changement de variable dans une int\'egrale.\\
Soit $v$ une fonction $l$ fois diff\'erentiable au point $x$. On note $D^l v(x)$ sa d\'eriv\'ee $l^{\hbox{\tiny \`eme}}$ au sens de Fr\'echet au point $x$. Il s'agit donc d'une forme $l$-lin\'eaire sym\'etrique sur $\RR^n$. On notera $D^l v(x).(\xi_1,\ldots,\xi_l)$ sa valeur pour $l$ vecteurs $\xi_i\in\RR^n$ ($1\le i \le l$).\saut
%
Reprenons les notations du \S\ref{sec:sobolev}. $\alpha=(\alpha_1,\ldots,\alpha_n)$ d\'esigne un multi-entier, et on note $|\alpha|=\alpha_1+\cdots+\alpha_n$. On a alors :
\be
|v|_{l,K}^2 = \int_{x\in K} \sum_{|\alpha|=l}\left\|\partial^{|\alpha|}v (x)\right\|^2 dx
\ee
%
et :
\be
\partial^{|\alpha|}v (x) = D^{|\alpha|}v(x).(\underbrace{e_1,\ldots,e_1}_{\alpha_1 \hbox{{\tiny fois}}}, \ldots , \underbrace{e_n,\ldots,e_n}_{\alpha_n \hbox{{\tiny fois}}})
\ee
o\`u $(e_1,\ldots,e_n)$ d\'esigne la base canonique de $\RR^n$.
%
Alors, en posant :
\be
\|D^l v(x)\| = \sup_{\xi_1,\ldots,\xi_l \in \left(\RR^{\ast}\right)^n} \; \frac{D^l v(x).(\xi_1,\ldots,\xi_l)}{|\xi_1| \ldots |\xi_l|}\qquad ,
\ee
%
on d\'eduit qu'il existe des constantes $\gamma_1$ et $\gamma_2$ d\'ependant uniquement de $n$ et $l$ (donc en particulier ind\'ependantes de $v$) telles que \be
\gamma_1 \; |v|_{l,K} \le \left( \int_{x\in K} \| D^l v(x)\|^2 \, dx\right)^{1/2} \le \gamma_2 \; |v|_{l,K}
\label{eq:maj1}
\ee
%
Par ailleurs, si l'on utilise le changement de variable $x=F(\hat{x})=B\hat{x}+b$ dans $D^l v(x)$, il vient :
\be
\forall \xi_1,\ldots,\xi_l \in \RR^n,\qquad
D^l \hat{v}(\hat{x}).(\xi_1,\ldots,\xi_l) = D^l v(x).(B\xi_1,\ldots,B\xi_l)
\ee
d'o\`u
\be
\|D^l \hat{v}(\hat{x})\| \le \|B\|^l\; \|D^l v(x)\|
\ee
%
Or, $D^l v(x) = D^l v(F(\hat{x}))$. Donc
\be
\int_{\hat{x}\in \hat{K}}\|D^l \hat{v}(\hat{x})\|^2 \; d\hat{x} \le \|B\|^{2l}\; \int_{\hat{x}\in \hat{K}} \|D^l v(F(\hat{x}))\|^2 \; d\hat{x}
= \|B\|^{2l}\; |\hbox{det }B|^{-1} \int_{x\in K} \|D^l v(x)\|^2 \; dx
\label{eq:maj2}
\ee
%
En minorant et majorant (\ref{eq:maj2}) gr\^ace \`a (\ref{eq:maj1}), on obtient :
\be
\gamma^2_1 \; |\hat{v}|^2_{l,\hat{K}} \le \|B\|^{2l}\; |\hbox{det }B|^{-1} \gamma^2_2 \; |v|^2_{l,K}
\ee
d'o\`u le r\'esultat (\ref{eq:majref}).
%
\vspace*{5 mm}\\
{\bf Corollaire :} On a de m\^eme :
\be
\forall v\in H^l(K), \qquad |v|_{l,K}  \le C(l,n)\; \|B^{-1}\|^l_2 \; |\hbox{det} B|^{1/2} \; |\hat{v}|_{l,\hat{K}}
\label{eq:majref2}
\ee
%
%
\subsubsection{Estimation de $\|B\|$}
%
\noindent
%
Soit $h_K$ le diam\`etre de $K$, c'est \`a dire le maximum des distances euclidiennes entre deux points de $K$. Soit $\rho_K$ la rondeur de $K$, c'est \`a dire le diam\`etre maximum des sph\`eres incluses dans $K$. On a :
\be
\|B\| = \sup_{x\ne 0} \frac{\|Bx\|}{\|x\|} = \sup_{\|x\|=\hat{\rho}} \frac{\|Bx\|}{\hat{\rho}}
\ee
%
Soit $x$ un vecteur de $\RR^n$ tel que $\|x\|=\hat{\rho}$. Par d\'efinition de $\hat{\rho}$, il existe deux points $\hat{y}$ et $\hat{z}$ de $\hat{K}$ tels que $x=\hat{y}-\hat{z}$. Alors $Bx=B\hat{y}-B\hat{z}=F(\hat{y})-F(\hat{z})=y-z$ avec $y$ et $z$ appartenant \`a $K$. Par d\'efinition de $h_K$, $\|y-z\| \le h_K$. Donc $\|Bx\| \le h_K$. En reportant dans la d\'efinition de $\|B\|$, on obtient donc :
\be
\|B\| \le \frac{h_K}{\hat{\rho}}
\label{eq:kk1}
\ee
%
Et on a \'evidemment de m\^eme :
\be
\|B^{-1}\| \le \frac{\hat{h}}{\rho_K}
\label{eq:kk2}
\ee
%
%
\subsection{Etape 4: Majoration sur l'\'el\'ement de r\'ef\'erence}
%
\noindent
%
Le r\'esultat principal est le suivant :\saut
{\bf Th\'eor\`eme :} Soient $l$ et $k$ deux entiers tels que $0\le l \le k+1$.
Si $\hat{\pi} \in {\cal L}(H^{k+1}(\hat{K}),H^l(\hat{K}))$ laisse $P_k(\hat{K})$ invariant (c'est \`a dire v\'erifie $\forall \hat{p}\in P_k(\hat{K}), \hat{\pi}\hat{p}=\hat{p}$), alors
\be
\exists C(\hat{K},\hat{\pi}) ,\;  \forall \hat{v} \in H^{k+1}(\hat{K}), \; |\hat{v}-\hat{\pi}\hat{v}|_{l,\hat{K}} \le C |\hat{v}|_{k+1,\hat{K}}
\label{eq:majref0}
\ee
%
{\it D\'emonstration} :\\
$\hat{\pi} \in {\cal L}(H^{k+1}(\hat{K}),H^l(\hat{K}))$, et donc  $I-\hat{\pi} \in {\cal L}(H^{k+1}(\hat{K}),H^l(\hat{K}))$ car $l\le k+1$.\\
Et donc $|\hat{v}-\hat{\pi}\hat{v}|_{l,\hat{K}} \le \|I-\hat{\pi}\|_{{\cal L}(H^{k+1}(\hat{K}),H^l(\hat{K}))}\; \|\hat{v}\|_{k+1,\hat{K}}$.\saut
%
On utilise maintenant l'invariance de $P_k(\hat{K})$:
\be
\forall \hat{p}\in P_k(\hat{K}), \; \hat{v}-\hat{\pi}\hat{v} = (I-\hat{\pi})(\hat{v}) = (I-\hat{\pi})(\hat{v}+\hat{p})
\ee
Donc
\be
|\hat{v}-\hat{\pi}\hat{v}|_{l,\hat{K}} \le \|I-\hat{\pi}\|_{{\cal L}(.,.)} \inf_{\hat{p}\in P_k(\hat{K})} \|\hat{v}+\hat{p}\|_{k+1,\hat{K}}
\ee
%
On aura donc d\'emontr\'e le th\'eor\`eme si l'on montre que
\be
\exists C,\; \forall \hat{v}\in H^{k+1}(\hat{K}) \;  \inf_{\hat{p}\in P_k(\hat{K})} \|\hat{v}+\hat{p}\|_{k+1,\hat{K}} \le C |\hat{v}|_{k+1,\hat{K}}
\ee
%
Soit $(f_i)_{i=0,\ldots,k}$ une base du dual de $P_k(\hat{K})$. D'apr\`es le th\'eor\`eme d'Hahn-Banach, il existe des formes lin\'eaires continues sur $H^{k+1}(\hat{K})$, que l'on notera encore $f_i$, et qui prolongent les $f_i$. En particulier, si $\hat{p}\in P_k(\hat{K})$ v\'erifie $f_i(\hat{p})=0,\, (i=0,\ldots,k)$, alors $\hat{p}=0$. Nous allons montrer que
\be
\exists C, \, \forall \hat{v}\in H^{k+1}(\hat{K}), \; \|\hat{v}\|_{k+1,\hat{K}} \le C \left\{ |\hat{v}|_{k+1,\hat{K}} + \sum_{i=0}^k |f_i(\hat{v})| \right\}
\label{eq:ref2}
\ee
%
On aura le r\'esultat souhait\'e en appliquant (\eq:ref2}) \`a $\hat{v}+\hat{q}$, avec $\hat{q}$ tel que $f_i(\hat{q})=f_i(-\hat{v})$.\\
%
La relation (\ref{eq:ref2}) se d\'emontre par l'absurde. Si elle n'est pas vraie, alors il existe une suite  de fonctions $\hat{v}_n$ de $H^{k+1}(\hat{K})$ telles que :
\be
 \|\hat{v}_n\|_{k+1,\hat{K}} =1, \;\;
|\hat{v}_n|_{k+1,\hat{K}} \longrightarrow 0,\; \hbox{ et } \forall i \;  f_i(\hat{v}_n)\longrightarrow 0
\ee
%
Par compl\'etude de $H^{k+1}(\hat{K})$, on extrait une sous-suite convergente vers $\hat{v} \in H^{k+1}(\hat{K})$. Mais $|\hat{v}_n|_{k+1,\hat{K}} \longrightarrow 0$. Donc $\hat{v} \in P_k(\hat{K})$ et $f_i(\hat{v})=0$. D'o\`u une contradiction.
%
%
\subsection{Etape 5: Assemblage des majorations locales}
%
%
\subsubsection{Majoration sur un \'el\'ement quelconque}
%
\noindent
%
En rassemblant les r\'esultats pr\'ec\'edents, on peut \'etablir une majoration sur un \'el\'ement quelconque $K$ du maillage. On a :
%
\begin{eqnarray*}
|v-\pi_K v|_{l,K} & \le & C(l,n)\; \|B^{-1}\|^l\; |\hbox{det }B|^{1/2} \; |\hat{v}-\hat{\pi}\hat{v}|_{l,\hat{K}}\qquad\hbox{d'apr\`es (\ref{eq:majref2})} \\
 & \le & C(l,n)\; \|B^{-1}\|^l\; |\hbox{det }B|^{1/2} \; C(\hat{K},\hat{\pi})\; |\hat{v}|_{k+1,\hat{K}} \qquad\hbox{d'apr\`es (\ref{eq:majref0})}\\
& \le & C(l,n)\; \|B^{-1}\|^l\; |\hbox{det }B|^{1/2} \; C(\hat{K},\hat{\pi})\; C(k+1,n) \; \|B\|^{k+1} |\hbox{det }B|^{-1/2}\; |v|_{k+1,K}\\
& & \qquad\qquad\qquad\qquad\hbox{d'apr\`es (\ref{eq:majref})}\\
& \le & C(l,n)\; \frac{\hat{h}^l}{\rho_K^l} \;  \; C(\hat{K},\hat{\pi})\; C(k+1,n) \; \frac{h_K^{k+1}}{\hat{\rho}^{k+1}} \; |v|_{k+1,K} \qquad\hbox{d'apr\`es (\ref{eq:kk1}) et (\ref{eq:kk2})}\\
 \end{eqnarray*}
%
D'o\`u finalement :
\be
|v-\pi_K v|_{l,K}  \le  \hat{C}(\hat{\pi},\hat{K},l,k,n)\; \frac{h_K^{k+1}}{\rho_K^l} \;   |v|_{k+1,K}
\label{eq:majloc}
\ee
%
Il est important de remarquer \`a ce niveau que $\hat{C}$ est ind\'ependant de $K$.
%
%
\subsubsection{Assemblage des r\'esultats locaux}
%
\noindent
%
On va maintenant reprendre la majoration (\ref{eq:majloc}) pour tous les \'el\'ements du maillage et toutes les valeurs de $l=0,\ldots,m$. On va d\'efinir deux quantit\'es  repr\'esentatives du maillage :
\begin{itemize}
\item $h\quad$ tel que $h_K \le h, \; \forall K\in {\cal T}_h\qquad$ (diam\`etre maximum des \'el\'ements)
\item $\sigma\quad$ tel que $\ds{\frac{h_K}{\rho_K}} \le \sigma, \; \forall K\in {\cal T}_h\qquad$ (caract\'erise l'aplatissement des \'el\'ements)
\end{itemize}
%
Alors :
%
\begin{eqnarray*}
\|v-\pi_K v\|^2_{m,K} & = & \sum_{l=0}^m |v-\pi_K v|^2_{l,K} \\
 & \le & \sum_{l=0}^m \hat{C}^2(\hat{\pi},\hat{K},l,k,n)\;
 \left(\frac{h_K^{k+1}}{\rho_K^l}\right)^2 \;   |v|^2_{k+1,K}\qquad\hbox{d'apr\`es (\ref{eq:majloc})}\\
 & \le & \sum_{l=0}^m \hat{C}^2(\hat{\pi},\hat{K},l,k,n)\; \left\{\left(\frac{h_K}{\rho_K}\right)^l\; h_K^{m-l}\; h_K^{k+1-m}\right\}^2 \;   |v|^2_{k+1,K}\\
 & \le & \left\{ \sum_{l=0}^m \hat{C}^2(\hat{\pi},\hat{K},l,k,n)\; \sigma^{2l} h^{2m-2l} \right\} \; \left[ h^{k+1-m}\; |v|_{k+1,K} \right]^2
 \end{eqnarray*}
%
Le terme entre accolades ne tend ni vers 0 ni vers l'infini quand $h$ tend vers 0. D'o\`u :
\be
\|v-\pi_K v\|_{m,K} \le \hat{C}'(\hat{\pi},\hat{K},l,k,n,\sigma,h)\; h^{k+1-m}\; |v|_{k+1,K}
\ee
%
En sommant ensuite sur tous les \'el\'ements du maillage :
\begin{eqnarray*}
\|v-\pi_h v\|^2_{m} & = & \sum_{K\in {\cal T}_h} \|v-\pi_K v\|^2_{m,K} \\
 & \le & \sum_{K\in {\cal T}_h} \left[ \hat{C}'(\hat{\pi},\hat{K},l,k,n,\sigma,h)\; h^{k+1-m} \; |v|_{k+1,K} \right]^2
 \end{eqnarray*}
%
D'o\`u finalement :
\be
\|v-\pi_h v\|_{m} \le C({\cal T}_h,m,k,n) \; h^{k+1-m}\; |v|_{k+1}
\label{eq:majfinal}
\ee
%
\subsection{R\'esultat final}
%
\noindent
%
En reportant (\ref{eq:majfinal}) dans (\ref{eq:cea2}), on obtient le r\'esultat final classique de majoration d'erreur :
\be
\|u-u_h\|_{m} \le {\cal C} \; h^{k+1-m}\; |u|_{k+1}
\label{eq:majfinal2}
\ee
%
\section{Quelques commentaires}
%
%
\begin{itemize}
\item
Une utilisation fr\'equente de (\ref{eq:majfinal2}) a lieu dans le cas $m=1$. Alors si l'espace de polyn\^omes $P_k(\hat{K})\subset H^1(\hat{K})$ (ce qui est toujours le cas) et si $\hat{\pi}$ est bien d\'efini sur $H^{k+1}(\hat{K})$, on a :
\be
\hbox{si }u\in H^{k+1}(\Omega),\quad \|u-u_h\|_1 \le {\cal C} \; h^k \; |u|_{k+1}
\ee
%
\item Si le domaine $\Omega$ n'est pas polygonal, la majoration pr\'ec\'edente n'est plus valable. On peut alors \'etablir d'autres majorations du m\^eme type -- se r\'ef\'erer par exemple \`a Raviart et Thomas (1983).
%
\item De m\^eme, si les calculs d'int\'egrales ne sont pas faits exactement mais \`a l'aide d'une int\'egration num\'erique, une erreur suppl\'ementaire doit \^etre prise en compte, qui conduit \`a une nouvelle majoration d'erreur -- voir l\`a-aussi par exemple Raviart et Thomas (1983).
\end{itemize}

%\chapter{Quelques aspects pratiques de la m\'ethode des \'el\'ements finis}
%
\noindent
%
On va donner dans ce chapitre quelques indications pratiques concernant la
programmation d'une m\'ethode d'\'el\'ements finis. On  pourra trouver beaucoup
plus de d\'etails par exemple dans les ouvrages de Joly (1990) et de Lucquin et
Pironneau (1996).\saut
%
On exposera ici quelques principes g\'en\'eraux dans le cas classique d'une m\'ethode d'\'el\'ements finis $P_1$ de Lagrange sur un maillage triangulaire d'un domaine de $\RR^2$. Des modifications devront donc \^etre apport\'ees pour d'autres types d'\'el\'ements finis.
%
%
\section{Maillage}
\noindent
%
%
On va se placer ici dans le cas fr\'equent de la r\'esolution d'un probl\`eme sur un ouvert born\'e de $\RR^2$, not\'e $\Omega$. Ce probl\`eme comporte des conditions aux limites sur $\partial \Omega$, qui peuvent \^etre de type Dirichlet ou Neumann. On notera $\Gamma_0$ la partie de $\partial \Omega$ o\`u ces conditions sont de type Dirichlet, et $\Gamma_1$ la partie o\`u elles sont de type Neumann. On aura $\Gamma_0 \cap \Gamma_1=\emptyset$ et $\partial \Omega=\Gamma_0 \cup \Gamma_1$.\saut
%
%
Comme dans les chapitres pr\'ec\'edents, on va noter $N_e$ le nombre d'\'el\'ements du maillage, et $N_h$ le nombre de noeuds (qui sont ici simplement les sommets des triangles).\saut
%
Le rep\'erage des noeuds est fait par l'interm\'ediaire d'un tableau COOR$(2,N_h)$ :
\begin{center}
\begin{tabular}{ll}
 COOR$(1,k)=$ & abscisse du noeud $k$\\
 COOR$(2,k)=$ & ordonn\'ee du noeud $k$
\end{tabular}
\end{center}
%
C'est le seul tableau travaillant avec les coordonn\'ees ``physiques'' dans le domaine. Tous les autres seront des tableaux d'adressage relatif.\saut
%
La d\'efinition des \'el\'ements du maillage est faite par un tableau CONEC$(3,N_e)$, qui fait le lien noeuds-\'el\'ements. Chaque \'el\'ement contient 3 noeuds ``locaux'' (car on travaille dans cet exemple sur des triangles). Ces noeuds correspondent chacun \`a un indice du tableau $COOR$, qu'on appellera leur ``num\'ero global''.\\
%
\hspace*{2 cm} CONEC$(i,l)=$ num\'ero global du $i^{\hbox{\tiny \`eme}}$ noeud local de l'\'el\'ement $l$ ($i=1,2,3$)\vspace*{5 mm}\\
%
Le rep\'erage des conditions de Dirichlet est r\'ealis\'e directement par un tableau DIRI$(N_0)$ balayant les $N_0$ noeuds de $\Gamma_0$ :\\
%
\hspace*{2 cm} DIRI$(i)=$ num\'ero global du $i^{\hbox{\tiny \`eme}}$ noeud de $\Gamma_0$.\saut
%
Le rep\'erage des conditions de Neumann est r\'ealis\'e  par un premier tableau NEUM$_0(N_1)$ balayant les $N_1$ \'el\'ements ayant un c\^ot\'e sur $\Gamma_1$, et par un second tableau NEUM$(3,N_1)$ indiquant quels c\^ot\'es des \'el\'ements sont sur $\Gamma_1$ :
%
\begin{center}
\begin{tabular}{lp{10 cm}}
 NEUM$_0(j)=$ & num\'ero du $j^{\hbox{\tiny \`eme}}$ \'el\'ement ayant un c\^ot\'e sur $\Gamma_1$.\\
%
NEUM$(i,j)=$ & 1 si le c\^ot\'e oppos\'e au $i^{\hbox{\tiny \`eme}}$ noeud local du $j^{\hbox{\tiny \`eme}}$ \'el\'ement de la liste NEUM$_0$ est sur $\Gamma_1$, 0 sinon.
%
\end{tabular}
\end{center}
%
%
\section{Assemblage de la matrice du syst\`eme}
\noindent
%
%
Le syst\`eme lin\'eaire auquel aboutit la d\'emarche des \'el\'ements finis est 
$A\mu=b$, avec
\begin{eqnarray*}
A_{ij} & = & a(\varphi_i,\varphi_j) = \int_\Omega \cdots \; + \int_{\Gamma_1} \cdots\\
b_{j} & = & l(\varphi_j) = \int_\Omega \cdots \; + \int_{\Gamma_1} \cdots\\
\end{eqnarray*}
%
Les \'etapes de la programmation sont alors :
\begin{enumerate}
\item Calcul de $A_{ij}=\int_\Omega \cdots$ , pour $i=1,\ldots,N_h$, $j=1,\ldots,N_h$.
%
\item Calcul de $b_j=\int_\Omega \cdots$ , pour $j=1,\ldots,N_h$.
%
\item Conditions de Neumann : \\
Pour tous les noeuds $j$ des \'el\'ements ayant un c\^ot\'e sur $\Gamma_1$ faire:
\begin{itemize}
\item $\ds{A_{ij}=A_{ij} + \int_{\Gamma_1} \cdots}\;$ pour les noeuds $i$ des \'el\'ements ayant un c\^ot\'e sur $\Gamma_1$
\item $\ds{b_j=b_j + \int_{\Gamma_1} \cdots }$
\end{itemize}
%
\item Conditions de Dirichlet : on modifie les composantes de $A$ et $b$ o\`u les noeuds de $\Gamma_0$ interviennent.\\
Pour tous les noeuds $j$ de $\Gamma_0$ faire:
\begin{itemize}
\item $b_i=b_i-u_j A_{ij}$ pour tous les noeuds $i$ de $\Omega \setminus \Gamma_0$ ($u_j$ d\'esigne la valeur impos\'ee sur le noeud $j$ par les conditions de Dirichlet)
\item $A_{ij}=A_{ji}=0$ pour tous les noeuds $i$ de $\Omega \setminus \Gamma_0$
\item $A_{ij}=0$ pour tous les noeuds $i$ de $\Gamma_0$
\item $A_{jj}=1$
\item $b_j=u_j$
\end{itemize}
%
\vspace*{10 mm}
\end{enumerate}
%
%
L'\'etape la plus co\^uteuse est la premi\`ere, c'est \`a dire le calcul de $\ds{A_{ij}=\int_\Omega H(\varphi_i,\varphi_j)}$, o\`u $H$ est un op\'erateur d\'ependant du probl\`eme que l'on traite. $A_{ij}$ peut \'evidemment \^etre d\'ecompos\'e sur les \'el\'ements du maillage :
$$
A_{ij} = \sum_{l=1}^{N_l} A_{ij}^{(l)} \qquad\hbox{ avec } A_{ij}^{(l)}= \int_{K_l} H(\varphi_i,\varphi_j)
$$
%
Une m\'ethode na\"{\i}ve de calcul serait :
\begin{center}
\begin{minipage}{10 cm}
{\tt 
Pour $i=1$ \`a $N_h$\\
Pour $j=1$ \`a $N_h$\\
\hspace*{5 mm}Pour $l=1$ \`a $N_l$\\
\hspace*{10 mm}Calcul de $A_{ij}^{(l)}$\\
\hspace*{10 mm}$A_{ij}=A_{ij}+A_{ij}^{(l)}$\\
\hspace*{5 mm}Fin pour\\
Fin pour\\
Fin pour
}\\
\end{minipage}
\end{center}
%
%
Toutefois, on calcule ainsi une grande majorit\'e de contributions $A_{ij}^{(l)}$ nulles. En effet, $A_{ij}^{(l)}\ne 0$ ssi les noeuds $i$ et $j$ appartiennent \`a l'\'el\'ement $l$. On peut donc reprendre l'assemblage de $A$
en bouclant cette fois sur les \'el\'ements, pour ne calculer que les termes utiles:
%
\begin{center}
\begin{minipage}{10 cm}
{\tt 
Pour $l=1$ \`a $N_l$\\
\hspace*{5 mm}Pour $i_0=1$ \`a $3$\\
\hspace*{5 mm}Pour $j_0=1$ \`a $3$\\
\hspace*{10 mm}$i=$ CONEC$(i_0,l)$\\
\hspace*{10 mm}$j=$ CONEC$(j_0,l)$\\
\hspace*{10 mm}Calcul de $A_{ij}^{(l)}$\\
\hspace*{10 mm}$A_{ij}=A_{ij}+A_{ij}^{(l)}$\\
\hspace*{5 mm}Fin pour\\
\hspace*{5 mm}Fin pour\\
Fin pour
}\\
\end{minipage}
\end{center}
%
%
On voit que la m\'ethode d'assemblage na\"{\i}ve conduit \`a $N_h^2\times N_l$ calculs \'el\'ementaires, contre $9 N_l$ pour la seconde m\'ethode. Dans le cas r\'eel d'un maillage \`a $N_l=10^6$ \'el\'ements, on aura environ $N_h=5\, 10^5$ noeuds, ce qui m\`ene \`a $2.5\, 10^{17}$ calculs \'el\'ementaires pour la m\'ethode na\"{\i}ve, contre $9\, 10^6$ pour la seconde m\'ethode !!
%
%
\section{Formules de quadrature}
%
%
\subsection{D\'efinitions}
\noindent
%
%
Le calcul des int\'egrales sur les \'el\'ements du maillage a souvent lieu par int\'egration num\'erique. On utilise pour cela une formule de quadrature, c'est \`a dire une formule du type 
\be
\int_K f(x) \; dx \simeq \sum_{m=1}^M \omega_m f(\xi_m)
\label{eq:quadra}
\ee
o\`u les  $\xi_m$  sont des points de $K$ (appel\'es points de quadrature) et les $\omega_m$ des coefficients de pond\'eration (ou encore des poids). On choisit en g\'en\'eral ces formules de telle sorte qu'elles soient exactes pour les polyn\^omes jusqu'\`a un certain degr\'e. On a d'ailleurs le r\'esultat suivant :\vspace*{3 mm}\\
%
{\bf Th\'eor\`eme :} Si la formule (\ref{eq:quadra}) est exacte pour les polyn\^omes de degr\'e inf\'erieur ou \'egal \`a $r$, alors il existe une constante $C>0$ telle que, pour toute fonction $f\in {\cal C}^{r+1}(K)$ :
\be
\left| \int_K f(x)\; dx - \sum_{m=1}^M \omega_m f(\xi_m) \right| \le C \; |K| \; h^{r+1}
\label{eq:quadrature}
\ee
o\`u $|K|$ est la mesure de $K$ et $h$ son diam\`etre.
%
%
\subsection{Quadrature en 1-D}
\noindent
%
%
On cherche \`a estimer $\ds{\int_a^b f(x)\; dx}$. Les formules les plus courantes sont :\vspace*{3 mm}\\
\begin{description}
\item[formule des rectangles]
\be
\int_a^b f(x)\; dx \simeq (b-a) \, f(a)
\ee
Elle est exacte pour les polyn\^omes constants.\\
%
\item[formule du point milieu]
\be
\int_a^b f(x)\; dx \simeq (b-a) \, f\left( \frac{a+b}{2} \right)
\ee
Elle est exacte pour les polyn\^omes de degr\'e inf\'erieur ou \'egal \`a 1.\\
%
\item[formule des trap\`ezes]
\be
\int_a^b f(x)\; dx \simeq (b-a) \, \frac{f(a)+f(b)}{2}
\ee
Elle est exacte pour les polyn\^omes de degr\'e inf\'erieur ou \'egal \`a 1.\\
%
\item[formule \`a deux points internes]
\be
\int_a^b f(x)\; dx \simeq (b-a) \,  \frac{f(\xi_1)+f(\xi_2)}{2}
\ee
avec $\xi_1=a+\lambda (b-a)$, $\xi_2=b-\lambda (b-a)$ et $\ds{\lambda=\frac{\sqrt{3}-1}{2\, \sqrt{3}} }$.
Elle est exacte pour les polyn\^omes de degr\'e inf\'erieur ou \'egal \`a 2.\\
%
\item[formule de Simpson]
\be
\int_a^b f(x)\; dx \simeq (b-a) \, \frac{f(a)+4f(\frac{a+b}{2})+f(b)}{6}
\ee
Elle est exacte pour les polyn\^omes de degr\'e inf\'erieur ou \'egal \`a 3.
%
\end{description}
%
%
%
\subsection{Quadrature en 2-D triangulaire}
\noindent
%
%
$K$ est cette fois un triangle, de sommets $a_1,a_2,a_3$, de milieux des ar\^etes $a_{12}, a_{13}, a_{23}$, de centre de gravit\'e $a_0$, et de surface $|K|$.
%
\begin{description}
\item[formule centr\'ee]
\be
\int_K f(x)\; dx \simeq |K| \, f(a_0)
\ee
Elle est exacte pour les polyn\^omes de degr\'e inf\'erieur ou \'egal \`a 1.\\
%
\item[formule sur les sommets]
\be
\int_K f(x)\; dx \simeq \frac{|K|}{3} \, \left( f(a_1)+f(a_2)+f(a_3) \right)
\ee
Elle est exacte pour les polyn\^omes de degr\'e inf\'erieur ou \'egal \`a 1.\\
%
\item[formule sur les milieux]
\be
\int_K f(x)\; dx \simeq \frac{|K|}{3} \, \left( f(a_{12})+f(a_{23})+f(a_{13}) \right)
\ee
Elle est exacte pour les polyn\^omes de degr\'e inf\'erieur ou \'egal \`a 2.
\end{description}
%
%
\small
~\vspace*{3cm}\\
\subsection*{Compl\'ements}
%
\begin{enumerate}
\item D\'emontrer la formule (\ref{eq:quadrature})
\item Pour chaque formule de quadrature 1-D et 2-D pr\'esent\'ee dans ce chapitre, d\'emontrer qu'elle est exacte pour les polyn\^omes jusqu'\`a un certain degr\'e.
\end{enumerate}

%
\normalsize



\appendix
\chapter{Outils d'analyse fonctionnelle}
%
%
\section{Quelques rappels}
%
\subsection{Normes et produits scalaires}
%
\noindent
Soit $E$ un espace vectoriel.\\
%
\begin{definition}
  \label{def:7}
  $\|.\|$ : $E \rightarrow \RR$ est une {\bf norme} sur $E$ ssi elle v\'erifie :
  \begin{description}
  \item[$\qquad$(N1)] $\left( \| x \| = 0 \right)  \Longrightarrow (x=0)$
  \item[$\qquad$(N2)] $\forall\, \lambda\in\RR,\; \forall x\in E, \quad \| \lambda x \|  = |\lambda| \; \| x \| $
  \item[$\qquad$(N3)] $\forall\,  x,y \in E, \quad \| x+ y \| \le \|x \| + \|y\|\qquad$ (in\'egalit\'e triangulaire)
    \\
  \end{description}
\end{definition}


%
%
\begin{example}
Pour $E=\RR^n$ et $x=(x_1,\ldots,x_n) \in\RR^n$, on d\'efinit les normes
$$
\| x \|_1 = \sum_{i=1}^n |x_i|
\qquad
\| x \|_2 = \left( \sum_{i=1}^n x_i^2 \right)^{1/2}
\qquad
\| x \|_\infty = \sup_{i} |x_i|
$$
\end{example}

%

%
%
%
\begin{definition}
  On appelle {\bf produit scalaire} sur $E$ toute forme bilin\'eaire sym\'etrique d\'efinie positive.\\
  $\quad<.,.>$ : $E\times E \rightarrow \RR$ est donc un produit scalaire sur
  $E$ ssi il v\'erifie :
  \begin{description}
  \item[$\qquad$(S1)] $\forall\; x,y \in E, \quad <x,y> = <y,x>$
  \item[$\qquad$(S2)] $\forall\; x_1,x_2,y \in E, \quad <x_1+x_2,y> = <x_1,y>
    + <x_2,y> $
  \item[$\qquad$(S3)] $\forall\; x,y \in E, \, \forall\, \lambda\in\RR,\quad
    <\lambda x,y> = \lambda <x,y> $
  \item[$\qquad$(S4)] $\forall\;  x \in E, x\ne 0, \quad  <x,x>\; > 0 $\\
  \end{description}
  \label{def:8}
\end{definition}

%
A partir d'un produit scalaire, on peut d\'efinir une {\bf norme induite} : $ \| x \| = \sqrt{<x,x>} $\\
%
On a alors, d'apr\`es (N3), l'{\bf in\'egalit\'e de Cauchy-Schwarz} : $\ds{ | <x,y> | \le \| x \| \; \| y \| }$

%
%
{\bf Exemple :} Pour $E=\RR^n$, on d\'efinit le produit scalaire $\ds{<x,y> = \sum_{i=1}^n x_i \, y_i}$. Sa norme induite est $\| . \|_2$ d\'efinie pr\'ec\'edemment.

%
%
Un espace vectoriel muni d'une norme est appel\'e {\bf espace norm\'e}. \\
Un espace vectoriel muni d'un produit scalaire est appel\'e {\bf espace pr\'ehilbertien}. En particulier, c'est donc un espace norm\'e pour la norme induite.
%
%
\subsection{Suites de Cauchy - espaces complets}
%
%
\noindent
\begin{definition}
  \label{def:1}
  Soit $E$ un espace vectoriel et $(x_n)_n$ une suite de $E$. $(x_n)_n$ est une {\bf suite de Cauchy} ssi $\forall \varepsilon > 0,\;\; \exists N / \forall p>N, \forall q>N, \quad \|x_p - x_q \| < \varepsilon$

\end{definition}

%
Toute suite convergente est de Cauchy. La r\'eciproque est fausse.
%
\begin{definition}
  \label{def:2}
  Un espace vectoriel est {\bf complet} ssi toute suite de Cauchy y est convergente.
\end{definition}


%
\begin{definition}
  \label{def:3}
  Un espace norm\'e complet est un {\bf espace de Banach}.
\end{definition}

%
\begin{definition}
  \label{def:4}
  Un espace pr\'ehilbertien complet est un {\bf espace de Hilbert}.\\
\end{definition}

%
\begin{definition}
  \label{def:5}
  Un espace de Hilbert de dimension finie est appel\'e  {\bf espace euclidien}.
\end{definition}

%
%
%
\section{Espaces fonctionnels}
%
%
\noindent
\begin{definition}
  \label{def:6}
  Un {\bf espace fonctionnel} est un espace vectoriel dont les \'el\'ements
  sont des fonctions.\end{definition}

\begin{example}
  ${\cal C}^p([a;b])$ d\'esigne l'espace des fonctions d\'efinies sur
  l'intervalle $[a,b]$, dont toutes les d\'eriv\'ees jusqu'\`a l'ordre $p$
  existent et sont continues sur $[a,b]$.
\end{example}



%
Dans la suite, les fonctions seront d\'efinies sur un sous-ensemble de $\RR^n$ (le plus souvent un ouvert not\'e $\Omega$), \`a valeurs dans $\RR$ ou $\RR^p$.
%

\begin{example}
  La temp\'erature $T(x,y,z,t)$ en tout point d'un objet $\bar{\Omega}\subset \RR^3$ est une fonction de $ \bar{\Omega} \times \RR \longrightarrow \RR$.
\end{example}


%

%
Les normes usuelles les plus simples sur les espaces fonctionnels sont les
{\bf normes} $\bf L^p$ d\'efinies par :
$$
\| u \|_{L^p} = \left ( \int_{\Omega } |u|^p \right) ^{1/p} \quad ,\; p\in [1,+ \infty[ ,
\qquad
\hbox{et}\qquad
\| u \|_{L^\infty} = {\hbox{Sup}}_{\Omega } |u|
$$
%
%
Comme on va le voir, ces formes $L^p$ ne sont pas n\'ecessairement des
normes. Et lorsqu'elles le sont, les espaces fonctionnels munis de ces normes
ne sont pas n\'ecessairement des espaces de Banach. Par exemple, les formes
$L^\infty$ et $L^1$ sont bien des normes sur l'espace ${\cal C}^0([a;b])$, et
cet espace est complet si on le munit de la norme $L^\infty$, mais ne l'est
pas si on le munit de la norme $L^1$.
%
Pour cette raison, on va d\'efinir les espaces ${\cal L}^p(\Omega)$ ($p\in [1,+ \infty[$) par
$$
{\cal L}^p(\Omega) = \left\{  u : \Omega \rightarrow \RR, \hbox{ mesurable, et telle que } \int_\Omega |u|^p<\infty  \right\}
$$
%
( on rappelle qu'une fonction $u$ est mesurable ssi $\{ x /  |u(x)|<r \}$ est mesurable $\forall r>0$. )
%
Sur ces espaces ${\cal L}^p(\Omega)$, les formes $L^p$ ne sont pas des normes. En effet, $\| u \|_{L^p} = 0$ implique que $u$ est nulle presque partout dans ${\cal L}^p(\Omega)$, et non pas $u=0$. C'est pourquoi on va d\'efinir les {\bf espaces} $\bf L^p(\Omega)$ :
%
\begin{definition}
  $L^p(\Omega)$ est la classe d'\'equivalence des fonctions de ${\cal
  L}^p(\Omega)$ pour la relation d'\'equivalence ``\'egalit\'e presque
  partout''. Autrement dit, on confondra deux fonctions d\`es lors qu'elles
  sont \'egales presque partout, c'est \`a dire qu'elles ne diff\`erent que
  sur un ensemble de mesure nulle.\label{def:9}
\end{definition}

%
\begin{theorem}
  La forme $L^p$ est une norme sur $L^p(\Omega)$, et $L^p(\Omega)$ muni de la
  norme $L^p$ est un espace de Banach (c.a.d. est complet).\label{thr:1}
\end{theorem}

%
Un cas particulier tr\`es important est $p=2$. On obtient alors \boldmath
l'{\bf espace fonctionnel $L^2(\Omega)$}, \unboldmath c'est \`a dire l'espace
des fonctions de carr\'e sommable sur $\Omega$ (\`a la relation
d'\'equivalence ``\'egalit\'e presque partout'' pr\`es). A la norme $L^2$ :
$\| u \|_{L^2} = \left( \int_\Omega u^2 \right)^{1/2} $, on peut associer la
forme bilin\'eaire $(u,v)_{L^2} = \int_\Omega u\, v$. Il s'agit d'un produit
scalaire, dont d\'erive la norme $L^2$. D'o\`u :
%
\begin{theorem}
  $L^2(\Omega)$ est un espace de Hilbert.\label{thr:2}
\end{theorem}

%
%
%
%
\section{Notion de d\'eriv\'ee g\'en\'eralis\'ee}
\label{sec:notion-de-derivee}
%
%
\noindent
Nous venons de d\'efinir des espaces fonctionnels complets, ce qui sera un bon
cadre pour d\'emontrer l'existence et l'unicit\'e de solutions d'\'equations
aux d\'eriv\'ees partielles, comme on le verra plus loin notamment avec le
th\'eor\`eme de Lax-Milgram. Toutefois, on a vu que les \'el\'ements de ces
espaces $L^p$ ne sont pas n\'ecessairement des fonctions tr\`es
r\'eguli\`eres. D\`es lors, les d\'eriv\'ees partielles de telles fonctions ne
sont pas forc\'ement d\'efinies partout. Pour s'affranchir de ce probl\`eme,
on va \'etendre la notion de d\'erivation.
%
Le v\'eritable outil \`a introduire pour cela est la notion de {\bf
distribution}, due \`a L. Schwartz (1950). Par manque de temps dans ce cours,
on se contentera ici d'en donner une id\'ee tr\`es simplifi\'ee, avec la
notion de {\bf d\'eriv\'ee g\'en\'eralis\'ee}.  Cette derni\`ere a des
propri\'et\'es beaucoup plus limit\'ees que les distributions, mais permet de
``sentir" les aspects n\'ecessaires pour mener \`a la formulation
variationnelle.
%
%
Dans la suite, $\Omega$ sera un ouvert (pas n\'ecessairement born\'e) de $\RR^n$.
%
%
\subsection{Fonctions tests}
\label{sec:fonctions-tests}
%
%
\noindent
\begin{definition}
  Soit $\varphi : \Omega \rightarrow \RR$. On appelle {\bf support de $\bf
  \varphi$} l'adh\'erence de $\{ x \in \Omega / \varphi(x) \ne 0 \}$.\label{def:10}
\end{definition}

%
\begin{example}
  Pour $\Omega = ]-1,1[$, et $\varphi$ la fonction constante \'egale \`a 1,
  $\hbox{Supp}\, \varphi = [-1,1]$.
\end{example}

%
\begin{definition}
  On note ${\cal D}(\Omega)$ l'espace des fonctions de $\Omega$ vers $\RR$, de
  classe ${\cal C}^\infty$, et \`a support compact inclus dans $\Omega$.
  ${\cal D}(\Omega)$ est parfois appel\'e {\bf espace des fonctions-tests}.\label{def:11}
\end{definition}

 %
\begin{example}
  L'exemple le plus classique dans le cas 1-D est la fonction \be \varphi(x) =
  \left\{
    \begin{array}{ll}
      \ds{ e^{- \frac{1}{1-x^2}} } & \hbox{si } |x|<1\\
      0 &  \hbox{si } |x|\ge 1\\
    \end{array}
  \right.
  \label{eq:fonction-test1}
  \ee
%
$\varphi$ est une fonction de ${\cal D}(]a,b[)$ pour tous $a < -1 < 1 < b$.
\end{example}

%
Cet exemple s'\'etend ais\'ement au cas multi-dimensionnel ($n>1$). Soit $a\in\Omega$ et $r>0$ tel que la boule ferm\'ee de centre $a$ et de rayon $r$ soit incluse dans $\Omega$. On pose alors :
\be
 \varphi(x) = \left\{
 \begin{array}{ll}
 \ds{ e^{- \frac{1}{r^2-|x-a|^2}} } & \hbox{si } |x-a|<r\\
 0 &  \hbox{sinon }\\
 \end{array}
 \right.
\label{eq:fonction-test2}
\ee
%
$\varphi$ ainsi d\'efinie est \'el\'ement de  ${\cal D}(\Omega)$.
%
\begin{theorem}
  $\overline{{\cal D}(\Omega) } = L^2(\Omega)$\label{thr:4}
\end{theorem}

%
%
%
\subsection{D\'eriv\'ee g\'en\'eralis\'ee}
\label{sec:derivee-generalisee}
%
%
\noindent
Soit $u\in {\cal C}^1(\Omega)$ et $\varphi \in {\cal D}(\Omega)$. Par int\'egration par parties (\cf annexe \ref{sec:green}), on a :
$$
\int_\Omega \partial_i u\;  \varphi = - \int_\Omega u \; \partial_i\varphi + \int_{\partial \Omega} u \; \varphi \; {\bf e}_i.{\bf n}
$$
%
%
Ce dernier terme (int\'egrale sur le bord de $\Omega$) est nul car $\varphi$
est \`a support compact (donc nul sur $\partial \Omega$). Or $\int_\Omega u
\; \partial_i\varphi$ a un sens par exemple d\`es que $u\in L^2(\Omega)$. Donc
le terme $\int_\Omega \partial_i u\; \varphi$ a aussi du sens, sans que $u$ ne
soit n\'ecessairement de classe ${\cal C}^1$. Ceci permet de d\'efinir
$\partial_i u$ m\^eme dans ce cas.
%
\begin{definition}{cas 1-D}
  $\quad$ Soit $I$ un intervalle de \RR, pas forc\'ement born\'e. On
  dit que $u\in L^2(I)$ admet une {\bf d\'eriv\'ee g\'en\'eralis\'ee} dans
  $L^2(I)$ ssi $\exists u_1\in L^2(I)$ telle que $\forall \varphi\in {\cal
  D}(I), \quad \int_I u_1\;\varphi = - \int_I u \varphi'$\label{def:12}
\end{definition}

%
\begin{example}
  Soit $I=]a,b[$ un intervalle born\'e, et $c$ un point de $I$. On consid\`ere
  une fonction $u$ form\'ee de deux branches de classe ${\cal C}^1$, l'une sur
  $]a,c[$, l'autre sur $]c,b[$, et se raccordant de fa\c{c}on continue mais
  non d\'erivable en $c$. Alors $u$ admet une d\'eriv\'ee g\'en\'eralis\'ee
  d\'efinie par $u_1(x)=u'(x)\quad \forall x\ne c$. En effet :
$$
\forall \varphi\in {\cal D}(]a,b[)\qquad \int_a^b u \varphi' = \int_a^c +
\int_c^b = - \int_a^c u' \varphi - \int_c^b u'\varphi +
\underbrace{(u(c^-)-u(c^+))}_{=0} \, \varphi(c)
$$
par int\'egration par parties. La valeur $u_1(c)$ n'a pas d'importance: on a
de toute fa\c{c}on au final la m\^eme fonction de $L^2(I)$, puisqu'elle est
d\'efinie comme classe d'\'equivalence de la relation d'\'equivalence
``\'egalit\'e presque partout".
\end{example}

%o
%
\begin{definition}
  En it\'erant, on dit que $u$ admet une {\bf d\'eriv\'ee g\'en\'eralis\'ee
  d'ordre $\bf k$} dans $L^2(I)$, not\'ee $u_k$, ssi $\ds{\forall \varphi\in
  {\cal D}(I), \quad \int_I u_k\;\varphi = (- 1)^k \; \int_I u \varphi^{(k)}
  }$\label{def:13}
\end{definition}

%
Ces d\'efinitions s'\'etendent naturellement pour la d\'efinition de d\'eriv\'ees partielles g\'en\'eralis\'ees, dans le cas $n>1$.

%
\begin{theorem}
  Quand elle existe, la d\'eriv\'ee g\'en\'eralis\'ee est unique.\label{thr:5}
\end{theorem}

\begin{theorem}
  Quand $u$ est de classe ${\cal C}^1(\bar{\Omega})$, la d\'eriv\'ee
  g\'en\'eralis\'ee est \'egale \`a la d\'eriv\'ee classique.\label{thr:6}
\end{theorem}


%
%
%
%
\section{Espaces de Sobolev}
%
%
\subsection{Les espaces $H^m$}
\label{sec:sobolev}
\noindent

\begin{definition}
  $\ds{ H^1(\Omega) = \left\{ u \in L^2(\Omega)\; / \; \partial_i u \; \in
    L^2(\Omega), \quad 1 \le i \le n \right\} }$ o\`u $\partial_i u$ est
  d\'efinie au sens de la d\'eriv\'ee g\'en\'eralis\'ee.\label{def:14}
\end{definition}
%
$H^1(\Omega)$ est appel\'e {\bf espace de Sobolev d'ordre 1}.
%
\begin{definition}
  Pour tout entier $m\ge 1$,
$$
H^m(\Omega) = \left\{ u \in L^2(\Omega) \; / \; \partial^\alpha u \; \in
  L^2(\Omega) \quad \forall \alpha =(\alpha_1,\ldots,\alpha_n) \in \NN^n\hbox{
  tel que}\; |\alpha|= \alpha_1+\cdots+\alpha_n \le m \right\} $$\label{def:15}
\end{definition}
%
$H^m(\Omega)$ est appel\'e {\bf espace de Sobolev d'ordre $\bf m$}.
%
Par extension, on voit aussi que $H^0(\Omega)=L^2(\Omega)$.
Dans le cas de la dimension 1, on \'ecrit plus simplement pour $I$ ouvert de $\RR$ :
$$ H^m(I) =  \left\{ u \in L^2(I)  \; / \;   u', \ldots, u^{(m)} \in L^2(I) \right\} $$

%
\begin{theorem}
  $H^1(\Omega)$ est un espace de Hilbert pour le produit scalaire
$$
(u,v)_1 = \int_\Omega u \, v\, + \sum_{i=1}^n \; \int_\Omega \partial_i u
\; \partial_i v = (u,v)_0 + \sum_{i=1}^n (\partial_i u, \partial_i v )_0
$$
en notant $(.,.)_0$ le produit scalaire $L^2$. On notera $\|.\|_1$ la norme
associ\'ee \`a $(.,.)_1$.\label{thr:7}
\end{theorem}
%
On d\'efinit de m\^eme un produit scalaire et une norme sur $H^m(\Omega)$ par
$$
(u,v)_m =   \sum_{|\alpha| \le m} ( \partial^\alpha u , \partial^\alpha v )_0 \qquad
\hbox{ et }\qquad
\| u \|_m = (u,u)_m^{1/2}
$$
%
\begin{theorem}
  $H^m(\Omega)$ muni du produit scalaire $(.,.)_m$ est un espace de Hilbert.\label{thr:8}
\end{theorem}


%
\begin{theorem}
  Si $\Omega$ est un ouvert de $\RR^n$ de fronti\`ere $\partial\Omega$
  ``suffisamment r\'eguli\`ere" (par exemple ${\cal C}^1$), on a l'inclusion :
  $H^m(\Omega) \subset {\cal C}^k(\bar{\Omega})$ pour $\ds{ k < m-\frac{n}{2}
  }$\label{thr:9}
\end{theorem}

%
\begin{example}
  En particulier, on voit que pour un intervalle $I$ de $\RR$, on a $H^1(I)
  \subset {\cal C}^0(\bar{I})$, c'est \`a dire que, en 1-D, toute fonction
  $H^1$ est continue.

  L'exemple de $u(x) = x\, \sin\frac{1}{x}$ pour $x\in]0,1]$ et $u(0)=0$
  montre que la r\'eciproque est fausse.

  L'exemple de $u(x,y) = | \ln (x^2+y^2) |^k$ pour $0<k<1/2$ montre qu'en
  dimension sup\'erieure \`a 1 il existe des fonctions $H^1$ discontinues.
\end{example}

%
%
\subsection{Trace d'une fonction}
%
%
Pour pouvoir faire les int\'egrations par parties qui seront utiles par exemple pour la formulation variationnelle, il faut pouvoir d\'efinir le prolongement ({\em la trace}) d'une fonction sur le bord de l'ouvert $\Omega$.
%
\underline{Si $n=1$ (cas 1-D)} : on consid\`ere un intervalle ouvert $I=]a,b[$ born\'e. On a vu que  $H^1(I) \subset {\cal C}^0(\bar{I})$. Donc, pour $u\in H^1(I)$, $u$ est continue sur $[a,b]$, et $u(a)$ et $u(b)$ sont bien d\'efinies.
%
\underline{Si $n>1$} : on n'a plus $H^1(\Omega) \subset {\cal C}^0(\bar{\Omega})$. Comment alors d\'efinir la trace ? La d\'emarche est la suivante :
\begin{itemize}
\item On d\'efinit l'espace
$$
{\cal C}^1(\bar{\Omega}) = \left\{  \varphi : \Omega \rightarrow \RR \;/\;  \exists O \hbox{ ouvert contenant } \bar{\Omega},\; \exists \psi \in {\cal C}^1(O),\; \psi_{|\Omega} = \varphi \right\}
$$
Autrement dit, ${\cal C}^1(\bar{\Omega})$ est l'espace des fonctions ${\cal C}^1$ sur $\Omega$, prolongeables par continuit\'e sur $\partial\Omega$ et dont le gradient est lui-aussi prolongeable par continuit\'e. Il n'y a donc pas de probl\`eme pour d\'efinir la trace de telles fonctions.
%
\item On montre que, si $\Omega$ est un ouvert born\'e de fronti\`ere $\partial\Omega$ ``assez r\'eguli\`ere", alors ${\cal C}^1(\bar{\Omega})$ est dense dans $H^1(\Omega)$.
%
\item L'application lin\'eaire continue, qui \`a toute fonction $u$ de  ${\cal C}^1(\bar{\Omega})$  associe sa trace sur $\partial\Omega$, se prolonge alors en une application lin\'eaire continue de $H^1(\Omega)$ dans $L^2(\partial\Omega)$, not\'ee $\gamma_0$, qu'on appelle {\bf application trace}.
\boldmath
On dit que $\gamma_0(u)$ {\bf est la trace de $u$ sur }$\partial\Omega$.
\unboldmath
%
\end{itemize}
%
%
Pour une fonction $u$ de $H^1(\Omega)$ qui soit en m\^eme temps continue sur $\bar{\Omega}$, on a \'evidemment $\gamma_0(u) = u_{|\partial\Omega}$. C'est pourquoi on note souvent par abus simplement $u_{|\partial\Omega}$ plut\^ot que $\gamma_0(u)$.

%
%
On peut de fa\c{c}on analogue d\'efinir $\gamma_1$, application trace qui permet de prolonger la d\'efinition usuelle de la d\'eriv\'ee normale sur $\partial\Omega$.  Pour $u\in H^2(\Omega)$, on a $\partial_i u \in H^1(\Omega)$, $\forall i=1,\ldots,n$, et on peut donc d\'efinir $\gamma_0(\partial_i u)$. La fronti\`ere $\partial\Omega$ \'etant ``assez r\'eguli\`ere" (par exemple, id\'ealement, de classe ${\cal C}^1$), on peut d\'efinir la normale $n=\left(   \begin{array}{l}  n_1 \\ \vdots \\ n_n \end{array} \right)$ en tout point de $\partial\Omega$. On pose alors $\ds{\gamma_1(u) = \sum_{i=1}^n \gamma_0(\partial_i u) n_i}$. Cette application continue $\gamma_1$ de $H^2(\Omega)$ dans $L^2(\partial\Omega)$ permet donc bien de prolonger la d\'efinition usuelle de la d\'eriv\'ee normale. Dans le cas o\`u $u$ est une fonction  de $H^2(\Omega)$ qui soit en m\^eme temps dans  ${\cal C}^1(\bar{\Omega})$, la d\'eriv\'ee normale au sens usuel de $u$ existe, et $\gamma_1(u)$ lui est \'evidemment  \'egal. C'est pourquoi on note souvent, par abus, $\partial_n u$ plut\^ot que $\gamma_1(u)$.
%
%
\subsection{Espace $\bf H^1_0(\Omega)$}
\label{sec:H10}
%
%
\noindent
\begin{definition}
  Soit $\Omega$ ouvert de $\RR^n$. L'espace $H^1_0(\Omega)$ est d\'efini comme
  l'adh\'erence de ${\cal D}(\Omega)$ pour la norme $\|.\|_1$ de
  $H^1(\Omega)$. (on rappelle que ${\cal D}(\Omega)$ est l'espace des
  fonctions ${\cal C}^\infty$ sur $\Omega$ \`a support compact, encore
  appel\'e espace des fonctions tests)\label{def:16}
\end{definition}

%
\begin{theorem}
  Par construction $H^1_0(\Omega)$ est un espace complet. C'est un espace de
  Hilbert pour la norme $\|.\|_1$\label{thr:10}
\end{theorem}

%
\underline{Si $n=1$ (cas 1-D)} : on consid\`ere un intervalle ouvert $I=]a,b[$ born\'e. Alors $$H^1_0(]a,b[) = \left\{ u \in H^1(]a,b[),\; u(a)=u(b)=0 \right\}$$
%
\underline{Si $n>1$} :  Si $\Omega$ est un ouvert born\'e de fronti\`ere``assez r\'eguli\`ere" (par exemple ${\cal C}^1$ par morceaux), alors  $H^1_0(\Omega) = \ker \gamma_0$.
%
$H^1_0(\Omega)$ est donc le sous-espace des fonctions de  $H^1(\Omega)$ de trace nulle sur la fronti\`ere $\partial\Omega$.
%
\begin{definition}
  Pour toute fonction $u$ de $H^1(\Omega)$, on peut d\'efinir :
$$\ds{ |u|_1 = \left( \sum_{i=1}^n \| \partial_i u \|_0^2 \right)^{1/2}
= \left( \int_\Omega \sum_{i=1}^n \left( \partial_i u \right)^2 dx
\right)^{1/2} } \vspace*{5 mm}
$$\label{def:17}
\end{definition}

%
\begin{theorem}[\textbf{In\'egalit\'e de Poincar\'e}]
  \label{thr:11}
  Si $\Omega$ est born\'e dans au moins une direction, alors il existe une
  constante $C(\Omega)$ telle que $\forall u \in H^1_0(\Omega), \; \|u\|_0 \le
  C(\Omega)\; |u|_1$.
\end{theorem}

%
On en d\'eduit que $|.|_1$ est une norme sur $H^1_0(\Omega)$, \'equivalente \`a la norme $\|.\|_1$.
%

%
%
\begin{corollary}
  Le r\'esultat pr\'ec\'edent s'\'etend au cas o\`u l'on a une condition de
  Dirichlet nulle seulement sur une partie de $\partial\Omega$, si $\Omega$
  est connexe.
\end{corollary}

On suppose que $\Omega$ est un ouvert born\'e connexe, de fronti\`ere ${\cal
C}^1$ par morceaux. Soit $V=\left\{ v\in H^1(\Omega),\, v=0 \hbox{ sur
  }\Gamma_0 \right\}$ o\`u $\Gamma_0$ est une partie de $\partial\Omega$ de
mesure non-nulle. Alors il existe une constante $C(\Omega)$ telle que $\forall
u \in V, \; \|u\|_{0,V} \le C(\Omega)\; |u|_{1,V}$, o\`u $\|.\|_{0,V}$ et
$|.|_{1,V}$ d\'esignent les norme et semi-norme induites sur $V$.
%
On en d\'eduit que $|.|_{1,V}$ est une norme sur $V$, \'equivalente \`a la norme $\|.\|_{1,V}$.
%
%
\small
~\vspace*{5mm}\\
\subsection*{Exercices}
%
\begin{enumerate}
\item Montrer que les fonctions d\'efinies par (\ref{eq:fonction-test1}) et
  (\ref{eq:fonction-test2}) sont bien ${\cal C}^\infty$ \`a support compact.
\item Montrer que ${\cal C}^0([a,b])$ est un espace complet pour la norme $L^\infty$.
\item Montrer que ce n'est pas le cas pour la norme $L^1$ (exhiber une suite
  de Cauchy non convergente dans ${\cal C}^0([a,b])$).
%\item Montrer que $L^1( est le complété de C0 pour la norme L1
%Lp complété de C0 pour norme Lp ? est-ce vrai ??
%
\item D\'emontrer que, lorsqu'elle existe, la d\'eriv\'ee g\'en\'eralis\'ee est unique.
\item D\'emontrer que, pour une fonction de classe ${\cal C}^1$, la
  d\'eriv\'ee g\'en\'eralis\'ee est \'egale \`a la d\'eriv\'ee classique.
\item Soit une fonction de $[a,b]$ vers $\RR$, form\'ee de deux branches de
  classe ${\cal C}^1$ sur $[a,c[$ et $]c,b]$, et discontinue en $c$.  Montrer
  qu'elle n'admet pas de d\'eriv\'ee g\'en\'eralis\'ee. (il faudrait alors
  avoir recours \`a la notion de distribution pour d\'eriver cette fonction).
\item Montrer que $|.|_1$ est une norme sur $H^1_0(\Omega)$, \'equivalente \`a la norme $\|.\|_1$
\end{enumerate}

%
\normalsize

%%
%



%%% Local Variables:
%%% coding: utf-8
%%% mode: latex
%%% TeX-PDF-mode: t
%%% TeX-parse-self: t
%%% TeX-auto-save: t
%%% x-symbol-8bits: nil
%%% TeX-auto-regexp-list: TeX-auto-full-regexp-list
%%% TeX-master: "mef-intro"
%%% ispell-local-dictionary: "american"
%%% End:

\chapter{Coordonn\'ees barycentriques}
\label{ch:bary}
%
\noindent
%
Soit $K$ un triangle de $\RR^2$ de sommets $a_1, a_2, a_3$. On appelle coordonn\'ees barycentriques de $K$ les fonctions affines $\lambda_1, \lambda_2, \lambda_3$ de $K$ dans \RR \/ d\'efinies par 
\be
\lambda_j(a_i) = \delta_{ij}, \qquad 1\le i,j \le 3
\label{eq:bary}
\ee
%
On voit que la somme $\lambda_1+\lambda_2+\lambda_3$ est une fonction affine qui vaut 1 sur chacun des 3 sommets. C'est donc la fonction constante \'egale \`a 1.\vspace*{5 mm}\\
%
Si l'on note $(x_i,y_i)$ les coordonn\'ees d'un sommet $a_i$ et $\lambda_j(x,y)=\alpha_j \, x + \beta_j \, y + \gamma_j$, la relation (\ref{eq:bary}) est \'equivalente au syst\`eme lin\'eaire :
\be
\left\{
\begin{array}{l}
\alpha_j x_i + \beta_j y_i + \gamma_j = 0 \\
\alpha_j x_k + \beta_j y_k + \gamma_j = 0 \\
\alpha_j x_j + \beta_j y_j + \gamma_j = 1 \\
\end{array}
\right.
\ee
o\`u $\{i,j,k\}$ est une permutation de $\{1,2,3\}$. La r\'esolution de ce syst\`eme m\`ene \`a :
\begin{eqnarray*}
\alpha_j & = & \ds{ \frac{y_k-y_i}{(x_j-x_k)(y_j-y_i) - (x_j-x_i)(y_j-y_k)} }\\
\beta_j & = & \ds{ \frac{x_i-x_k}{(x_j-x_k)(y_j-y_i) - (x_j-x_i)(y_j-y_k)} }\\
\gamma_j & = & \ds{ \frac{x_k y_i - x_i y_k}{(x_j-x_k)(y_j-y_i) - (x_j-x_i)(y_j-y_k)} }
\end{eqnarray*}
%
Si l'on note $|K|$ l'aire du triangle et $\varepsilon$ le signe de ($\stackrel{\longrightarrow}{a_ka_j},\stackrel{\longrightarrow}{a_ia_j}$), on peut aussi r\'e\'ecrire ces \'egalit\'es sous la forme :
\begin{eqnarray*}
\alpha_j & = & \ds{ \frac{y_k-y_i}{2\varepsilon \; |K|} }\\
\beta_j & = & \ds{ \frac{x_i-x_k}{2\varepsilon \; |K|} }\\
\gamma_j & = & \ds{ \frac{x_k y_i - x_i y_k}{2\varepsilon \; |K|} }
\end{eqnarray*}
%
%
{\bf Propri\'et\'e :} $(\lambda_1,\lambda_2,\lambda_3)$ est une base de $P_1$.
\vspace*{5 mm}\\
%
{\bf Propri\'et\'e :} $(\lambda_1\lambda_2,\lambda_1\lambda_3,\lambda_2\lambda_3,\lambda_1^2,\lambda_2^2,\lambda_3^2)$ est une base de $P_2$.
\vspace*{5 mm}\\
%
{\bf Propri\'et\'e :} On note $\{i,j,k\}$ une permutation de $\{1,2,3\}$ et $m,n,p$ des entiers. On a alors~:
\be
\int_K \lambda_i^m \lambda_j^n \lambda_k^p \; dx = \frac{2\, |K| \, m!\, n! \, p!}{(2+m+n+p)!}
\ee
%
\vspace*{10 mm}\\
%
La notion de coordonn\'ees barycentriques s'\'etend naturellement au cas d'un t\'etra\`edre dans $\RR^3$. On travaille dans ce cas avec 4 coordonn\'ees barycentriques.


\begin{figure}[h]
\begin{center}
\includegraphics[width=0.85\linewidth]{FIG/coord-barycentriques.jpg}
%\vspace*{6 cm}
\caption{Coordonn\'ees barycentriques sur un triangle}
\label{fig:coord-bar}
\end{center}
\end{figure}
%
%
\noindent
{\bf Propri\'et\'e :} Soit $p$ un polyn\^ome de degr\'e quelconque \`a
deux variables qui s'annule sur une droite d'\'equation $\lambda(x,y)=
0$. Alors il  existe un polyn\^ome $q$ tel que $p=\lambda q$.\\
Soit $\overrightarrow n$  un vecteur orthogonal \`a la droite 
d'\'equation $\lambda(x,y)=0$. Si de plus
$\displaystyle{\dpa{p}{n}(x,y)}$
s'annule sur cette droite, alors il existe un polyn\^ome $r$
tel que $p= \lambda^2 r$.
\vspace*{5mm}\\
%
%
%
\noindent
{\bf Exemples de calcul de fonctions de base :} 
\begin{itemize}
\item {\bf Fonctions de base $\bf{P_1}$} Soit $p_i$ la fonction de base $P_1$ associ\'ee au sommet $a_i$. Elle est d\'efinie par : $p_i(a_i) = 1$, $p_i(a_j)=0$ pour $i\ne j$, et $p_i \in P_1$. C'est donc exactement la d\'efinition des coordonn\'ees barycentriques : $p_i = \lambda_i$
%
\item {\bf Fonctions de base $\bf{P_2}$} Soit $p_1$ la fonction de base $P_2$ associ\'ee au sommet $a_1$. Elle est d\'efinie par : $p_1(a_1) = 1$, $p_1(a_2)=p_1(a_3)=p_1(a_{12})=p_1(a_{13})=p_1(a_{23})=0$ , et $p_1 \in P_2$.\\
La restriction de $p_1$ \`a la droite $[a_2, a_3]$ est un polyn\^ome \`a {\bf une} variable, de degr\'e 2, qui s'annule en trois points distincts $a_2$, $a_{23}$ et $a_3$. Elle est donc identiquement nulle sur la droite $[a_2, a_3]$, dont l'\'equation est $\lambda_1=0$. Donc il existe un polyn\^ome $q_1$ de degr\'e inf\'erieur ou \'egal \`a 1 tel que $p_1=\lambda_1 q_1$. \\
Les  relations $p_1(a_{12})=p_1(a_{13})=0$ deviennent donc $q_1(a_{12})=q_1(a_{13})=0$ (car $\lambda_1(a_{12}) \ne 0$ et $\lambda_1(a_{13}) \ne 0$). Donc la restriction de $q_1$ \`a la droite $[a_{12}, a_{13}]$ est un polyn\^ome \`a {\bf une} variable, de degr\'e 1, qui s'annule en deux points distincts $a_{12}$ et $a_{13}$. Elle est donc identiquement nulle sur la droite $[a_{12}, a_{13}]$, dont l'\'equation est $\lambda_1-1/2=0$. Donc il existe une constante $\alpha$  telle que $q_1=\alpha (\lambda_1-1/2)$, soit $p_1= \alpha \lambda_1 (\lambda_1-1/2)$.\\
La relation $p_1(a_1) = 1$ fournit finalement $\alpha = 2$. D'o\`u $p_1= \lambda_1 (2\lambda_1-1)$.
%
%
\end{itemize}










\chapter{Calcul d'int\'egrales}
%
\section{Formules de Green}
\label{sec:green}
\noindent
%
%
Soit $\Omega$ un ouvert non-vide de $\RR^n$, de fronti\`ere not\'ee $\partial\Omega$. On note $n$ la normale locale sur $\partial\Omega$.\saut
%
On a les propri\'et\'es suivantes, appel\'ees formules de Green, qui sont en fait simplement des cas particuliers d'int\'egration par parties :
\be
\int_\Omega \frac{\partial u}{\partial x_k}\; v\; dx = - \int_\Omega u \; \frac{
\partial v}{\partial x_k} \; dx + \int_{\partial \Omega} u\, v \; (e_k.n)\; ds
\ee
%
o\`u $e_k$ est le vecteur unitaire dans la direction $x_k$.
%
%
\be
\int_\Omega \Delta u \; v\; dx = - \int_\Omega \nabla u \; \nabla v \; dx + \int_{\partial \Omega} \frac{\partial u}{\partial n}\, v \; ds
\ee
%
%
\be
\int_\Omega u \; \hbox{div} E \; dx = - \int_\Omega \nabla u \; E \; dx + \int_{\partial \Omega}  u \; (E.n)\; ds
\ee
%
%
\section{Changement de variable dans une int\'egrale}
\noindent
%
%
Soient $\hat{K}$ et $K$ deux ouverts de $\RR^n$. Soit $F$ un ${\cal C}^1$- diff\'eomorphisme de $\hat{K}$ dans $K$, c'est \`a dire une bijection de classe ${\cal C}^1$ dont la r\'eciproque est \'egalement de classe ${\cal C}^1$. On note $(e_1,\ldots,e_n)$ la base canonique de $\RR^n$ et 
$$
F : x=\sum_{i=1}^n x_i \, e_i \; \longrightarrow \; F(x) = \sum_{i=1}^n F_i(x_1,\ldots,x_n) \, e_i
$$
La matrice jacobienne de $F$ au point $x$, not\'ee $J_F(x)$ est la matrice $n\times n$ d\'efinie par
$$
\left( J_F(x) \right)_{ij} = \frac{\partial F_i}{\partial x_j}(x_1,\ldots,x_n)
\qquad 1\le i,j \le n
$$
%
%
On a alors la formule de changement de variable :
\be
\int_K u(x)\; dx = \int_{\hat{K}} u(F(\hat{x}))\; \left| \hbox{det } J_F(\hat{x}) \right| \; d\hat{x}
\ee
%
%
{\bf Remarque :} dans la m\'ethode des \'el\'ements finis, on aura souvent \`a calculer de tels changements de variables dans des int\'egrales du type $\ds{ \int_K Hu(x)\; dx }$, o\`u $H$ est un op\'erateur aux d\'eriv\'ees partielles (gradient, laplacien, \ldots). Il faudra alors faire attention au changement de variable dans l'op\'erateur lui-m\^eme. Par exemple dans $\RR^2$ :
%
\begin{eqnarray*}
\int_K (\nabla u(x))^2\; dx & = & {\ds \int_K \left[ \left(\frac{\partial u(x,y)}{\partial x} \right)^2 + \left(\frac{\partial u(x,y)}{\partial y} \right)^2 \right]\; dx\; dy }\\
%
& = & \ds{ \int_{\hat{K}} \left[ \left(\frac{\partial u(F(\hat{x},\hat{y}))}{\partial x}  \right)^2 + 
\left(\frac{\partial u(F(\hat{x},\hat{y}))}{\partial y} \right)^2 \right] \left| \hbox{det} J_F(\hat{x}) \right| \; \; d\hat{x}\, d\hat{y}
}\\
%
& = & \ds{ \int_{\hat{K}} \left[ \left(\frac{\partial u(F(\hat{x},\hat{y}))}{\partial
 \hat{x}} \;  \frac{\partial \hat{x}}{\partial x} + \frac{\partial u(F(\hat{x},\hat{y}))}{\partial \hat{y}} \; \frac{\partial \hat{y}}{\partial x} \right)^2  \right.
}\\
%
& & \qquad
 \ds{ \left. +
\left(\frac{\partial u(F(\hat{x},\hat{y}))}{\partial \hat{x}} \;  \frac{\partial \hat{x}}{\partial y} + \frac{\partial u(F(\hat{x},\hat{y}))}{\partial \hat{y}} \; \frac{\partial \hat{y}}{\partial y} \right)^2 \right] \left| \hbox{det} J_F(\hat{x}) \right| \; \; d\hat{x}\, d\hat{y}
}
\end{eqnarray*}
%
$$
\quad
= \int_{\hat{K}} \left[ \left(\frac{\partial \hat{u}(\hat{x},\hat{y})}{\partial
 \hat{x}} \;  \frac{\partial \hat{x}}{\partial x} + \frac{\partial \hat{u}(\hat{x},\hat{y})}{\partial \hat{y}} \; \frac{\partial \hat{y}}{\partial x} \right)^2 + 
\left(\frac{\partial \hat{u}(\hat{x},\hat{y})}{\partial \hat{x}} \;  \frac{\partial \hat{x}}{\partial y} + \frac{\partial \hat{u}(\hat{x},\hat{y})}{\partial \hat{y}} \; \frac{\partial \hat{y}}{\partial y} \right)^2 \right] \left| \hbox{det} J_F(\hat{x}) \right| \; \; d\hat{x}\, d\hat{y}
\vspace*{5 mm}
$$
%
Dans le cas d'une transformation $F$ affine, not\'ee :
$$
\left\{
\begin{array}{lll}
x & = & a\hat{x} + b\hat{y} + e\\
y & = & c\hat{x} + d\hat{y} + f
\end{array}
\right.
$$
on a :
$$
\hat{x} = \frac{d(x-e)-b(y-f)}{D},\qquad 
\hat{y} = \frac{-c(x-e)+a(y-f)}{D},\quad \hbox{ et }
\left| \hbox{det} J_F(\hat{x}) \right| = D = ad-bc
\vspace*{5 mm}
$$
%
Le calcul pr\'ec\'edent devient alors :
$$
\int_K (\nabla u(x))^2\; dx = \int_{\hat{K}} \left[ \left(\frac{\partial \hat{u}(\hat{x},\hat{y})}{\partial \hat{x}} \;  \frac{d}{D} + \frac{\partial \hat{u}(\hat{x},\hat{y})}{\partial \hat{y}} \; \frac{-c}{D} \right)^2 + 
\left(\frac{\partial \hat{u}(\hat{x},\hat{y})}{\partial \hat{x}} \;  \frac{-b}{D} + \frac{\partial \hat{u}(\hat{x},\hat{y})}{\partial \hat{y}} \; \frac{a}{D} \right)^2 \right] |D| \; \; d\hat{x}\, d\hat{y}
$$
%
$$
\qquad\qquad\quad\qquad = \frac{1}{|D|}\; \int_{\hat{K}} \left[ \left( d\, \frac{\partial \hat{u}(\hat{x},\hat{y})}{\partial \hat{x}} - c\, \frac{\partial \hat{u}(\hat{x},\hat{y})}{\partial \hat{y}} \right)^2 + 
\left(-b\, \frac{\partial \hat{u}(\hat{x},\hat{y})}{\partial \hat{x}} \; + a\, \frac{\partial \hat{u}(\hat{x},\hat{y})}{\partial \hat{y}} \right)^2 \right]  \; \; d\hat{x}\, d\hat{y}
$$






\section{Feel++}
\label{sec:feel++}

\subsection{Machines et sites web}
\label{sec:machines-et-sites}

Vous devez avoir un compte sur la machine \texttt{irma-hpc2.u-strasbg.fr} pour
pouvoir utiliser Feel++. Pour vous loguer sur cette machine, vous utilisez
\texttt{ssh} (Secure Shell)
\begin{lstlisting}[language=sh]
  ssh -X -Y irma-hpc2.u-strasbg.fr -l <votre login sur irma-hpc2>
  ou
  ssh -X -Y <votre login sur irma-hpc2>@irma-hpc2.u-strasbg.fr
\end{lstlisting}
Les options \texttt{-X} \texttt{-Y} vous permettent de passer les
autorisations à \texttt{irma-hpc2} d'envoyer des fenêtres graphiques via le
protocole X11 (vous devez avoir Linux ou MacOsX pour cela).

\subsection{Utilisation de Feel++}
\label{sec:util-de-feel++}

Dans cette section, nous supposons que Feel++ est installée ce qui est le cas
sur la machine \texttt{irma-hpc2}. Il vous suffit de définir la variable
d'environnement \verb+FEELPP_DIR+}
\begin{lstlisting}[language=sh]
  export FEELPP_DIR=/opt/feelpp/0.92.1
\end{lstlisting}
ensuite créez un répertoire dans \texttt{/scratch/<votre login>/} qui s'appelle
\texttt{tutoriel}  par exemple.
\begin{lstlisting}[language=sh]
  mkdir -p /scratch/<votre login>/tutoriel
  cmake /opt/feelpp/0.92.1/share/doc/feel/examples
  # compilez la première application de Feel++
  make feelpp_qs_laplacian
\end{lstlisting}

À présent vous pouvez récupérer le manuel de Feel++ soit sur le site de ce
cours soit \url{https://feelpp.googlecode.com/files/feelpp-manual.pdf} et
étudier le tutoriel et les exemples proposés.

\subsection{Installation à partir des sources de Feel++}
\label{sec:les-sources-de}

\subsubsection{Cloner Feel++}
\label{sec:cloner-feel++}


À présent vous devez récupérer les sources de Feel++. Elles sont gérées par le
système de gestion de version \texttt{Git}\footnote{\htmladdnormallink{http://git-scm.com/}}
sur Google code\footnote{\htmladdnormallink{http://code.google.com/p/feelpp}}. Des explications
complètes sont disponibles dans le livre \texttt{Pro Git}\footnote{\htmladdnormallink{http://git-scm.com/book}}. Sur
\texttt{irma-hpc2} vous effectuez les commandes suivantes:
\begin{lstlisting}[language=sh]
  # creation d'un repertoire de stockage des sources
  mkdir Devel
  cd Devel
  # recuperation des sources de Feel++
  git clone https://code.google.com/p/feelpp/
\end{lstlisting}
À la suite de la dernière commande qui prend un peu de temps un repertoire
\texttt{feelpp} a été crée dans \texttt{Devel}.

\subsubsection{Mettre a jour Feel++}
\label{sec:mettre-jour-feel++}

Feel++ est un logiciel qui évolue constamment. Pendant le cours j'apporte des
modifications et corrections qui, je l'espère, simplifie l'utilisation du
logiciel. Afin de mettre à jour le logiciel, vous effectuez la commande
suivante:
\begin{lstlisting}
  git pull origin master
\end{lstlisting}
Cette commande va d'abord synchoniser la branche locale
\lstinline!origin/master! (voir le résultat de \lstinline!git branch -a!) avec
celui de Google Code puis dans un deuxième fusionner la branche locale
\lstinline!master! avec \lstinline!origin/master!.


\subsection{Compilation de Feel++}
\label{sec:comp-de-feel++}

Il vous faut à présent compiler Feel++. Feel++ utilise
\texttt{cmake}\footnote{\htmladdnormallink{http://www.cmake.org}}. \texttt{cmake} permet, à
partir de fichier \texttt{CMakeLists.txt} de générer des fichiers
\texttt{Makefile}.
\begin{lstlisting}[language=sh]
  # nous sommes dans le repertoire Devel
  mkdir compile
  cd compile
  cmake -DCMAKE_CXX_COMPILER=/usr/bin/g++-4.6 ../feelpp
\end{lstlisting}
Par défaut Feel++ est compilé avec les options \texttt{-g -O2} et avec
l'option \lstinline!-DCMAKE_CXX_COMPILER=/usr/bin/g++-4.6! Feel++ sera compilé
avec le compilation Gcc version 4.6.
À la fin de l'execution de la commande \texttt{cmake} vous devez voir un
message de ce type (sans erreur)
\begin{lstlisting}
...
...
-- =====================================================================

-- Some things you can do now with Feel++:
-- =====================================================================
-- Command        |   Description
-- ===============|=====================================================
-- make install   | Install to /usr/local. To change that:
--                |     cmake . -DCMAKE_INSTALL_PREFIX=yourpath
--                |   Feel++ headers will then be installed to:
--                |     /usr/local/include/feel
-- make doc       | Generate the manual applications, manual and the API
--                | documentation, requires Doxygen & LaTeX
-- make benchmarks| Generate the benchmarks(Warning: takes a long time!)
-- make check     | Build and run the unit-tests.
-- =====================================================================
--
-- Configuring done
-- Generating done
-- Build files have been written to: /home/prudhomm/Devel/compile
\end{lstlisting}

Feel++ contient de nombreux exemples et applications, nous n'allons travailler
qu'avec un petit nombre:
\begin{lstlisting}[language=sh]
  # une premiere application
  feelpp_doc_myapp
  # generer un maillage
  feelpp_doc_mymesh
  # calculer des integrales
  feelpp_doc_myintegrals
  # manipuler les espaces de fonctions
  # polynomiales par morceaux
  feelpp_doc_myfunctionspace
  # resoudre $-\Delta u=f$ en 1D
  feelpp_m1cssi_laplacian
\end{lstlisting}
Afin de compiler ces exemples rendez vous dans le répertoire
\texttt{doc/manual/tutorial/}, puis compilez-les de la fa\c {c}on suivante:
\begin{lstlisting}[language=sh]
  cd doc/manual/tutorial
  make -j3 feelpp_doc_myapp feelpp_doc_mymesh \
           feelpp_doc_myintegrals feelpp_doc_myfunctionspace\
           feelpp_m1cssi_laplacian
\end{lstlisting}

L'option \lstinline{-j3} permet de compiler les fichiers C++ 3 par 3. Dans un
premier temps \lstinline{make} compile la librairie Feel++ qui est nécessaire
aux programmes ci-dessus, puis la commande \lstinline{make} compile les
applications.


%%% Local Variables:
%%% coding: utf-8
%%% mode: latex
%%% TeX-PDF-mode: t
%%% TeX-parse-self: t
%%% x-symbol-8bits: nil
%%% TeX-auto-regexp-list: TeX-auto-full-regexp-list
%%% TeX-master: "mef-intro"
%%% ispell-local-dictionary: "french"
%%% End:


% \chapter*{Bibliographie}
% %
% %
% ~\\
% %
% {\bf Br\'ezis P.}, 1999: {\it Analyse fonctionnelle}. Dunod.\vspace*{3mm}\\
% %
% {\bf Ciarlet P.}, 1990: {\it Introduction \`a l'analyse num\'erique matricielle et \`a l'optimisation}. Masson.\vspace*{3mm}\\
% %
% {\bf Dhatt G., G. Touzot et E. Lefran\c{c}ois}, 2005: {\it M\'ethode des \'el\'ements finis}. Hermes - Lavoisier.\vspace*{3mm}\\
% %
% {\bf Joly P.}, 1990: {\it Mise en oeuvre de la m\'ethode des \'el\'ements finis}. Ellipses.\vspace*{3mm}\\
% %
% {\bf Larrouturou B. et P.-L. Lions}, 1992: {\it M\'ethodes math\'ematiques pour les sciences de l'ing\'enieur: optimisation et analyse num\'erique}. Cours de l'Ecole Polytechnique.\vspace*{3mm}\\
% %
% {\bf Lascaux P. et R. Th\'eodor}, 1986: {\it Analyse num\'erique matricielle appliqu\'ee \`a l'art de l'ing\'enieur}. Masson.\vspace*{3mm}\\
% %
% {\bf Lucquin B. et O. Pironneau}, 1996: {\it Introduction au calcul scientifique}. Masson.\vspace*{3mm}\\
% %
% {\bf Raviart P.-A. et J.-M. Thomas}, 1983: {\it Introduction \`a l'analyse num\'erique
% des \'equations aux d\'erive\'ees partielles}. Masson.\vspace*{3mm}\\
% %
% {\bf Sainsaulieu L.}, 1996: {\it Calcul scientifique}. Masson.\vspace*{3mm}\\
% %
% %
% {\bf Thomas P.}, 2006: {\it El\'ements finis pour l'ing\'enieur}. Lavoisier, Tec et Doc.\vspace*{3mm}\\

\bibliographystyle{alpha}
\bibliography{mesh,navier_stokes}

%%% Local Variables:
%%% coding: utf-8
%%% mode: latex
%%% TeX-PDF-mode: t
%%% TeX-parse-self: t
%%% TeX-auto-save: t
%%% x-symbol-8bits: nil
%%% TeX-auto-regexp-list: TeX-auto-full-regexp-list
%%% TeX-master: "mef-intro"
%%% ispell-local-dictionary: "american"
%%% End:

%
%
\end{document}
%%% Local Variables:
%%% coding: utf-8
%%% mode: latex
%%% TeX-PDF-mode: t
%%% TeX-parse-self: t
%%% TeX-auto-save: t
%%% x-symbol-8bits: nil
%%% TeX-auto-regexp-list: TeX-auto-full-regexp-list
%%% TeX-master: t
%%% ispell-local-dictionary: "francais"
%%% End:
