\section{Feel++}
\label{sec:feel++}

\subsection{Machines et sites web}
\label{sec:machines-et-sites}

Vous devez avoir un compte sur la machine \texttt{irma-hpc2.u-strasbg.fr} pour
pouvoir utiliser Feel++. Pour vous loguer sur cette machine, vous utilisez
\texttt{ssh} (Secure Shell)
\begin{lstlisting}[language=sh]
  ssh -X -Y irma-hpc2.u-strasbg.fr -l <votre login sur irma-hpc2>
  ou
  ssh -X -Y <votre login sur irma-hpc2>@irma-hpc2.u-strasbg.fr
\end{lstlisting}
Les options \texttt{-X} \texttt{-Y} vous permettent de passer les
autorisations à \texttt{irma-hpc2} d'envoyer des fenêtres graphiques via le
protocole X11 (vous devez avoir Linux ou MacOsX pour cela).

\subsection{Utilisation de Feel++}
\label{sec:util-de-feel++}

Dans cette section, nous supposons que Feel++ est installée ce qui est le cas
sur la machine \texttt{irma-hpc2}. Il vous suffit de définir la variable
d'environnement \verb+FEELPP_DIR+}
\begin{lstlisting}[language=sh]
  export FEELPP_DIR=/opt/feelpp/0.92.1
\end{lstlisting}
ensuite créez un répertoire dans \texttt{/scratch/<votre login>/} qui s'appelle
\texttt{tutoriel}  par exemple.
\begin{lstlisting}[language=sh]
  mkdir -p /scratch/<votre login>/tutoriel
  cmake /opt/feelpp/0.92.1/share/doc/feel/examples
  # compilez la première application de Feel++
  make feelpp_qs_laplacian
\end{lstlisting}

À présent vous pouvez récupérer le manuel de Feel++ soit sur le site de ce
cours soit \url{https://feelpp.googlecode.com/files/feelpp-manual.pdf} et
étudier le tutoriel et les exemples proposés.

\subsection{Installation à partir des sources de Feel++}
\label{sec:les-sources-de}

\subsubsection{Cloner Feel++}
\label{sec:cloner-feel++}


À présent vous devez récupérer les sources de Feel++. Elles sont gérées par le
système de gestion de version \texttt{Git}\footnote{\htmladdnormallink{http://git-scm.com/}}
sur Google code\footnote{\htmladdnormallink{http://code.google.com/p/feelpp}}. Des explications
complètes sont disponibles dans le livre \texttt{Pro Git}\footnote{\htmladdnormallink{http://git-scm.com/book}}. Sur
\texttt{irma-hpc2} vous effectuez les commandes suivantes:
\begin{lstlisting}[language=sh]
  # creation d'un repertoire de stockage des sources
  mkdir Devel
  cd Devel
  # recuperation des sources de Feel++
  git clone https://code.google.com/p/feelpp/
\end{lstlisting}
À la suite de la dernière commande qui prend un peu de temps un repertoire
\texttt{feelpp} a été crée dans \texttt{Devel}.

\subsubsection{Mettre a jour Feel++}
\label{sec:mettre-jour-feel++}

Feel++ est un logiciel qui évolue constamment. Pendant le cours j'apporte des
modifications et corrections qui, je l'espère, simplifie l'utilisation du
logiciel. Afin de mettre à jour le logiciel, vous effectuez la commande
suivante:
\begin{lstlisting}
  git pull origin master
\end{lstlisting}
Cette commande va d'abord synchoniser la branche locale
\lstinline!origin/master! (voir le résultat de \lstinline!git branch -a!) avec
celui de Google Code puis dans un deuxième fusionner la branche locale
\lstinline!master! avec \lstinline!origin/master!.


\subsection{Compilation de Feel++}
\label{sec:comp-de-feel++}

Il vous faut à présent compiler Feel++. Feel++ utilise
\texttt{cmake}\footnote{\htmladdnormallink{http://www.cmake.org}}. \texttt{cmake} permet, à
partir de fichier \texttt{CMakeLists.txt} de générer des fichiers
\texttt{Makefile}.
\begin{lstlisting}[language=sh]
  # nous sommes dans le repertoire Devel
  mkdir compile
  cd compile
  cmake -DCMAKE_CXX_COMPILER=/usr/bin/g++-4.6 ../feelpp
\end{lstlisting}
Par défaut Feel++ est compilé avec les options \texttt{-g -O2} et avec
l'option \lstinline!-DCMAKE_CXX_COMPILER=/usr/bin/g++-4.6! Feel++ sera compilé
avec le compilation Gcc version 4.6.
À la fin de l'execution de la commande \texttt{cmake} vous devez voir un
message de ce type (sans erreur)
\begin{lstlisting}
...
...
-- =====================================================================

-- Some things you can do now with Feel++:
-- =====================================================================
-- Command        |   Description
-- ===============|=====================================================
-- make install   | Install to /usr/local. To change that:
--                |     cmake . -DCMAKE_INSTALL_PREFIX=yourpath
--                |   Feel++ headers will then be installed to:
--                |     /usr/local/include/feel
-- make doc       | Generate the manual applications, manual and the API
--                | documentation, requires Doxygen & LaTeX
-- make benchmarks| Generate the benchmarks(Warning: takes a long time!)
-- make check     | Build and run the unit-tests.
-- =====================================================================
--
-- Configuring done
-- Generating done
-- Build files have been written to: /home/prudhomm/Devel/compile
\end{lstlisting}

Feel++ contient de nombreux exemples et applications, nous n'allons travailler
qu'avec un petit nombre:
\begin{lstlisting}[language=sh]
  # une premiere application
  feelpp_doc_myapp
  # generer un maillage
  feelpp_doc_mymesh
  # calculer des integrales
  feelpp_doc_myintegrals
  # manipuler les espaces de fonctions
  # polynomiales par morceaux
  feelpp_doc_myfunctionspace
  # resoudre $-\Delta u=f$ en 1D
  feelpp_m1cssi_laplacian
\end{lstlisting}
Afin de compiler ces exemples rendez vous dans le répertoire
\texttt{doc/manual/tutorial/}, puis compilez-les de la fa\c {c}on suivante:
\begin{lstlisting}[language=sh]
  cd doc/manual/tutorial
  make -j3 feelpp_doc_myapp feelpp_doc_mymesh \
           feelpp_doc_myintegrals feelpp_doc_myfunctionspace\
           feelpp_m1cssi_laplacian
\end{lstlisting}

L'option \lstinline{-j3} permet de compiler les fichiers C++ 3 par 3. Dans un
premier temps \lstinline{make} compile la librairie Feel++ qui est nécessaire
aux programmes ci-dessus, puis la commande \lstinline{make} compile les
applications.


%%% Local Variables:
%%% coding: utf-8
%%% mode: latex
%%% TeX-PDF-mode: t
%%% TeX-parse-self: t
%%% x-symbol-8bits: nil
%%% TeX-auto-regexp-list: TeX-auto-full-regexp-list
%%% TeX-master: "mef-intro"
%%% ispell-local-dictionary: "french"
%%% End:
